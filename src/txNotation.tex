\section{Notation}
  We introduce the following notation to formally express Bitcoin transactions.

  \paragraph{Basic building blocks}
    \begin{itemize}
      \item signature (needed to spend): $\mathrm{player}_{\mathrm{sigName}}$,
      e.g. $\alice_{\mathrm{rev}}$
      \item value (in bitcoin): $x\bitcoin$
    \end{itemize}

  \paragraph{Spending method -- in transaction output (possibly named)}
    \begin{itemize}
      \item $n$ out of $n$ multisig: \\
      $\mathtt{AND}(\mathrm{sig}_1, \dots, \mathrm{sig}_n)$, alternatively
      $\mathrm{sig}_1 \wedge \dots \wedge \mathrm{sig}_n$
      \item relative delay -- minimum blocks between current and spending
      transaction: \\
      $\rdel(\msig, \mathit{blocks})$, e.g. $\rdel(\alice_{F} \wedge \bob_{F},
      3)$
      \item absolute delay -- minimum block where current transaction can be
      spent: \\
      $\adel(\msig, \mathit{block})$, e.g. $\mathtt{delayed} =
      \adel(\alice_{\mathrm{htlc}}, 1005)$
      \item hashlock -- a hash is provided here, its preimage must be provided
      by the spending transaction. Can be nested: \TODO{remove nesting if
      unneeded} \\
      $\hlock(\msig, h)$, e.g. $\hlock(\alice_{\mathrm{htlc}} \wedge
      \charlie_{\mathrm{htlc}}, \mathtt{0x9b4f})$
    \end{itemize}

  \paragraph{Spending methods set -- each output contains one such set} \ \\
    $\mathtt{OR}(\mathit{method}_1, \dots, \mathit{method}_m)$,
    alternatively $\mathit{method}_1 \vee \dots \vee \mathit{method}_m$, \\
    e.g. $(\mathtt{fulfill} = \hlock(\alice, \mathtt{0x1bc6})) \vee
    (\mathtt{refund} = \adel(\bob, 1007))$

  \paragraph{Output -- each transaction contains one or more (possibly named)} \
  \\
    $\txout(\text{\it set of methods}, \mathit{value})$, \\
    e.g. $\mathtt{coins}_{\alice} = \txout(\mathtt{normal} = \rdel(\alice, 10)
    \vee \mathtt{revocation} = \bob_{\mathrm{rev}})$

  \paragraph{Input -- each transaction contains one or more, unambiguous
  arguments can be omitted. Signatures/preimages are implied} \ \\
    $\txin(\text{\it previous tx name}, \text{\it previous output name},
    \text{\it previous method name})$, \\
    e.g. $\txin(\mathtt{comm}_{\alice}, \mathtt{coins}_{\alice},
    \mathtt{revocation})$

  \paragraph{Transaction} \ \\
    $\tx((\txin_1, \dots, \txin_n), (\txout_1, \dots, \txout_m))$, \\
    e.g. $\mathtt{rev}_{\bob} = \tx((\txin(\mathtt{comm}_{\alice},
    \mathtt{coins}_{\alice}, \mathtt{revocation})), (\txout(\bob)))$
