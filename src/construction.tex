\section{Model \& Construction}
  In this section we will examine the architecture and the details of our model,
  along with possible attacks and their mitigations. Following the UC
  framework~\cite{uc}, we define an ideal-world functionality \fchan and a
  simulator \simulator, along with a real-world protocol \pchan that UC-realizes
  \fchan (Theorem~\ref{theorem:security}).

  Our ideal world functionality \fchan
  (Figures~\ref{code:functionality:rules}-\ref{code:functionality:close})
  represents a single channel, either simple or virtual. It acts as a relay
  between \adversary and \environment, leaking all messages. This simplifies the
  functionality and facilitates the indistinguishability argument by having
  \simulator simply running internally the real world protocols of the channel
  parties \pchan with no modifications. \fchan internally maintains a state
  machine (c.f. Figure~\TODO{state machine}) that keeps track of which internal
  parties are corrupted or negligent, whether the channel has opened, whether a
  payment is underway, which external parties are to be considered trusted (as
  they correspond to other channels owned by the same player) and whether the
  channel has closed. The single security check performed is whether the
  on-chain coins are at least equal to the expected balance once the channel
  closes. If this check fails, \fchan halts.
