Consider a sequence of parties $A_1, \dots, A_n$. We say that $i$ is left of
$i+1$ and $i+1$ is right of $i$. $\forall i \in \{2, \dots, n-1\}$, party $A_i$
has a channel with $A_{i-1}$ of total value $x_{i-1, i}$ and a channel with
$A_{i+1}$ of total value $x_{i, i+1}$. $A_1$ only has a channel with $A_2$ (of
value $x_{1, 2}$), likewise $A_n$ only has a channel with $A_{n-1}$ (of value
$x_{n-1, n}$).

After following a specific protocol that does not involve any new on-chain
transactions, each party holds off-chain a number of transactions and signatures
that imply the existence of a new channel between $A_1$ and $A_n$ with value
$x'$, funded by $A_1$. At a high level, these transactions are as follows:
\begin{itemize}
  \item Each edge party has a transaction that consumes the funding output of
  its only channel and produces two outputs: one for the preexisting channel,
  where the left party has $x'$ coins less and one that carries the $x'$ coins
  for the virtual channel (read: the left party pays for the virtual channel).
  Call the latter ``virtual output''.
  \item Each intermediate party $A_i$ has three types of transactions:
  \begin{itemize}
    \item An ``initiator'' transaction, which consumes both its channel outputs
    and produces four: one for the left channel where the left party $A_{i-1}$
    has $x'$ less coins, one for the right channel where $A_i$ has $x'$ less
    coins, one that pays $A_i$ directly $x'$ coins and one virtual output with
    $x'$ coins.
    \item Several ``extend-interval'' transactions which may be used if exactly
    one of the two adjacent parties has consumed the funding output of the
    shared channel. Wlog, assume that the party to the left has consumed the
    funding output $A_{i-1} A_i$ whereas the party to the right has not consumed
    $A_i A_{i+1}$. $A_i$'s suitable extend-interval tx consumes $A_i A_{i+1}$
    and the virtual output produced by $A_{i-1}$'s transaction. In turn it
    produces one $A_i A_{i+1}$ funding output where $A_i$ has $x'$ less coins,
    one output with $x'$ coins for $A_i$ and a new virtual output with $x'$
    coins.
    \item Several ``merge-intervals'' transactions which can be used if both
    adjacent parties have consumed their respective funding output. The suitable
    merge-intervals tx consumes both virtual outputs from left and right and
    produces a new virtual output with $x'$ coins and an output that pays $A_i$
    directly $x'$ coins.
  \end{itemize}

  \item Each party has one ``commitment'' transaction for each channel in which
  it takes part. This transaction can spend the latest funding output and
  produce one output for each party, each carrying the rightful amount. It also
  holds one ``revocation'' transaction per channel update which can be used to
  punish its counterparty if it publishes an old commitment transaction, by
  confiscating the entire channel value. To perform a payment, the two parties
  first create new commitment transactions with the new balance and then create
  revocation transactions for the old commitment transactions. This is in effect
  a simpler version of Lightning.
\end{itemize}

\section*{Q\&A}
\begin{itemize}
  \item \emph{Why are there many extend-interval and merge-intervals
  transactions?}
  \item A virtual output produced by a tx of $A_i$ specifies exactly the
  interval of parties around $A_i$ that have already made their move (i.e. the
  maximal set of successive hops that have made their move and that includes
  $A_i$). For example in the case of 10 hops, if the spending condition
  ``$\mathit{all}^8 \wedge \texttt{"4"}$'' is on-chain, it signifies that
  parties 4, 5, 6 and 7 have moved and that (if everyone is honest) parties 8 and 3
  have not moved -- it does not say anything about party 1, 2, 9 or 10.

  $A_i$ can only spend a virtual output of which the interval ends just before
  or just after $i$ and the single newly produced virtual output has an interval
  that is the union of the intervals of the consumed virtual outputs with $i$
  added.  Therefore $A_i$ has multiple extend-interval and merge-intervals
  transactions because each one corresponds to different previous interval(s).

  As a result, each intermediate party can only publish exactly one transaction.
  This transaction always generates exactly one new virtual output. If it is an
  initiator tx, it does not consume a virtual output. If it is an
  extend-interval, it consumes one and if it is a merge-intervals it consumes
  two. A merge-interval tx can be published only if the publishing party is
  surrounded by two initiators (its two adjacent parties, two non-adjacent
  parties one per side, or one adjacent party on one side and a non-adjacent one
  on the other), therefore eventually only one virtual output will remain, as
  intended.
  \item \emph{What if a malicious intermediary creates a new virtual output and
  consumes it together with an honest virtual output using its merge-intervals
  transaction?}
  \item As the merge-intervals tx has a virtual output with a wider interval,
  the same party cannot repeat the same trick. Since every new move widens the
  interval (it adds the mover to the previous interval), even if only one edge
  party is honest, the attack cannot carry on for ever, therefore eventually the
  edge party will be able to consume the virtual output as intended. Similar
  reasoning applies to extend-interval malicious transactions, where the
  malicious party fabricates the funding output. Regarding the case where a
  malicious party fabricates a virtual output and then publishes an
  extend-interval transaction that consumes this fabricated output and a valid
  funding output, we observe that the valid intervals of the aforementioned
  virtual output may include only parties that are not towards the direction of
  the honest counterparty. This means that the counterparty has the same view as
  if the malicious party was indeed an extend-interval, which causes it no
  financial loss. This fact does not change if more parties are malicious: the
  only possible difference for any honest party is the ability to spend more
  than one (extend-interval or merge-intervals) transactions and therefore gain
  more coins than if everyone were honest. Intuitively, any malicious party that
  fabricates a malicious output in order to spend an honest one just introduces
  more coins to the protocol in a way that does not allow it to gain value.
  \item \emph{What if a malicious party publishes an old commitment transaction
  (i.e. consumes a funding output without using any of the initiator,
  extend-interval or merge-intervals txs)?}
  \item Its counterparty $A_i$ won't be able to close honestly its other
  adjacent channel, but it will be able to punish the malicious party with the
  revocation transaction, thus confiscating all its funds. Therefore, to ensure
  no monetary loss is possible, $A_i$ must always enforce that $x_{i-1, i,
  \mathrm{right}} \leq x_{i, i+1, \mathrm{right}}$ and $x_{i, i+1,
  \mathrm{left}} \leq x_{i-1, i, \mathrm{left}}$ (where $x_{i, j,
  \mathrm{left/right}}$ is the value owned by the left/right party of channel
  $A_i A_j$ respectively). This balance check is performed on every payment and
  new virtual channel. NB: This is not too restrictive to not allow payments,
  but it is conjectured that this limitation can be lifted if an eltoo-based
  channel update method is used instead of the current, lightning-based method.
  \item \emph{What about timelocks?}
  \item Virtual outputs can be consumed by extend-interval or merge-intervals as
  soon as they enter the ledger state, but if such a party does not publish its
  transaction after a while, then the virtual channel parties should be able to
  use this output as funding output for their virtual channel -- this prevents
  griefing attacks. Therefore we need to put a timelocked spending condition on
  each virtual output spendable by the two parties that own the new virtual
  channel.

  Each such timelock should be long enough for each of the entitled
  intermediaries to have enough time to consume the virtual output, plus give a
  little leeway in case the party goes offline for a short period. Our
  construction allows the creation of ``recursive'' virtual channels, i.e.
  virtual channels that are built on top of other virtual channels. The funding
  outputs of the virtual channels exist off-chain and they need some time to
  reach the chain. The deeper an intermediary's channel is nested and the larger
  the number of hops that enabled this intermediary's channels, the longer has
  to be the timelock for the virtual outputs it is able consume.
  \item \emph{What is the timelock value of a channel?}
  \item
  \begin{equation}
    t =
      \begin{cases}
        p + s & \mbox{if funding output on-chain}, \\
        p + \sum\limits_{i = 2}^{n-1}(s - 1 + t_i) & \mbox{else}
      \end{cases} \enspace,
  \end{equation}
  where $t_i$ is the timelock of the $i$-th underlying intermediary party, $s$
  is the upper bound of $\eta$ as in Lemma~7.19 of~\cite{BMTZ17} and we
  arbitrarily choose $p = 3$ globally. This arises as the worst case delay,
  where a virtual channel owner submits its transaction and then every
  intermediary submits its second-mover transaction at the latest possible
  moment, one after the other.
  \item \emph{What is the protocol followed by channel parties to establish the
  necessary keys and signatures for the virtual channel transactions?}
  \item At a high level, the protocol consists of three roundtrips, each
  starting from the virtual channel funder to the first intermediary and then
  from each intermediary to the next, up to the virtual channel fundee. The
  first roundtrip is for key and timelocks distribution, where each party
  obtains all the necessary keys for all its required transactions, in the
  second roundtrip all signatures except for those needed to consume funding
  outputs are distributed and finally in the third roundtrip the parties
  exchange signatures that consume the funding outputs. This structure ensures
  that parties only commit to the new channel state only after they have locally
  all the signatures necessary to enforce this state unilaterally.
\end{itemize}
