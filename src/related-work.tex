\section{Related Work}
  Due to massive replication and pervasive synchronisation requirements,
  blockchains have inherently low throughput and high latency~\cite{scaling}.
  The most prominent solution is Payment Channel Networks
  (PCNs)~\cite{DBLP:conf/fc/GudgeonMRMG20}. A number of constructions have been
  proposed, each with its own features and limitations. The first such
  proposal~\cite{spilman} only enabled unidirectional payment channels. One of
  the oldest PCNs is Duplex Micropayment Channels (DMC)~\cite{decker}, which
  uses a decrementing timelock for each new channel state, ensuring that only
  the latest state can be put on-chain first. The Lightning
  Network~\cite{lightning} was proposed to allow channels to stay open for an
  arbitrary length of time, leveraging the punishment mechanism that is used in
  the current work as well. Lightning is now implemented and functional for
  Bitcoin. It has also been adapted for Ethereum~\cite{wood2014ethereum}, where
  it is known as the Raiden Network~\cite{raiden}.

  The design comes with some weaknesses. The wormhole
  attack~\cite{DBLP:conf/ndss/MalavoltaMSKM19} against Lightning allows
  colluding parties in a multi-hop payment to steal the fees of the
  intermediaries between them and Flood \& Loot~\cite{10.1145/3419614.3423248}
  analyses the feasibility of an attack in which too many channels are forced to
  close in a short amount of time, reducing the blockchain liveness and enabling
  a malicious party to steal off-chain funds.

  Payment routing~\cite{spider,prihodko2016flare,lee2020routee} is another
  research area that aims to improve the network efficiency without sacrificing
  privacy. Actively rebalancing channels~\cite{DBLP:conf/ccs/KhalilG17} can
  further increase network efficiency by preventing routes from becoming
  unavailable due to lack of well-balanced funds.

  Generalized Bitcoin-Compatible
  Channels~\cite{cryptoeprint:2020:476} enable the creation of state channels on
  Bitcoin, extending channel functionality from simple payments to arbitrary
  Bitcoin scripts.

  Sprites~\cite{sprites} leverages the scripting language of Ethereum to
  decrease the time collateral is locked up compared to Lightning.
  Perun~\cite{perun} and GSCN~\cite{DBLP:conf/ccs/DziembowskiFH18} exploit the
  Turing-complete scripting language of Ethereum to provide virtual state
  channels, i.e. channels that can open without an on-chain transaction and that
  allow for arbitrary scripts to be executed off-chain. Similar features are
  provided by Celer~\cite{dong2018celer}. Hydra~\cite{cryptoeprint:2020:299}
  provides state channels for the Cardano~\cite{cardano} blockchain.

  BDW~\cite{scalable-funding} shows how pairwise channels over Bitcoin can be
  funded with no on-chain transactions by allowing parties to form groups that
  can pool their funds together off-chain and then use those funds to open
  channels. ACMU~\cite{10.1145/3319535.3345666} allows for multi-path atomic
  payments with reduced collateral, enabling new applications such as
  crowdfunding conditional on reaching a funding target.

  TEE-based~\cite{zhao2019sok}
  solutions~\cite{teechan,10.1145/3341301.3359627,liao2021speedster,lee2020routee}
  improve the throughput and efficiency of PCNs by an order of magnitude or
  more, at the cost of having to trust TEEs. Brick~\cite{avarikioti2020brick}
  uses a partially trusted committee to extend PCNs to fully asynchronous
  networks.

  Solutions alternative to PCNs include sidechains~\cite{cryptoeprint:2020:175}
  and partially centralised payment networks that entirely avoid using a
  blockchain~\cite{DBLP:conf/trust/ArmknechtKMYZ15,stellar,silentwhispers,DBLP:conf/ndss/RoosMKG18}.

  Last but not least, a number of works propose virtual channel constructions
  for Bitcoin. Lightweight Virtual Payment
  Channels~\cite{10.1007/978-3-030-65411-5_18} enables a virtual channel to be
  opened on top of two preexisting channels and uses a technique similar to DMC.
  Bitcoin-Compatible Virtual Channels~\cite{cryptoeprint:2020:554} also enables
  virtual channels on top of two preexisting simple (i.e. non-virtual) channels
  and offers two protocols, the first of which guarantees that the channel will
  stay off-chain for an agreed period, while the second allows the intermediary
  to turn the virtual into a simple channel.

  \cite{DBLP:conf/fc/GudgeonMRMG20}, \cite{spider}, \cite{lightning},
  \cite{raiden}, \cite{bitcoin}, \cite{scaling}, \cite{decker},
  \cite{DBLP:conf/trust/ArmknechtKMYZ15}, \cite{stellar}, \cite{silentwhispers},
  \cite{DBLP:conf/ndss/RoosMKG18}, \cite{DBLP:conf/ccs/DziembowskiFH18},
  \cite{perun}, \cite{teechan}, \cite{sprites}, \cite{prihodko2016flare},
  \cite{DBLP:conf/ndss/MalavoltaMSKM19}, \cite{cryptoeprint:2020:175},
  \cite{cryptoeprint:2020:554}, \cite{cryptoeprint:2020:476},
  \cite{avarikioti2020brick}, \cite{10.1145/3319535.3345666},
  \cite{10.1007/978-3-030-65411-5_18}, \cite{cryptoeprint:2020:299},
  \cite{dong2018celer}, \cite{10.1145/3341301.3359627}, \cite{liao2021speedster}
