% TODO: Consider merging part of this section with \ref{section:comparison}
\paragraph{Related work}
 The first proposal for PCNs was due to~\cite{spilman} which only enabled
 unidirectional payment channels. As mentioned previously, DMCs~\cite{decker}
 with their decrementing timelocks have the shortcoming of limited channel
 lifetime. This was ameliorated by LN~\cite{lightning} which has become the dominant paradigm for designing PCNs for Bitcoin-compatible systems. 
 LN is currently implemented and operational for
  Bitcoin. It has also been adapted for Ethereum~\cite{wood2014ethereum}, where
  it is known as the Raiden Network~\cite{raiden}.

A number of attacks have been identified against LN. The wormhole
  attack~\cite{DBLP:conf/ndss/MalavoltaMSKM19} against LN allows
  colluding parties in a multi-hop payment to steal the fees of the
  intermediaries between them and Flood \& Loot~\cite{10.1145/3419614.3423248}
  analyses the feasibility of an attack in which too many channels are forced to
  close in a short amount of time, reducing the blockchain liveness and enabling
  a malicious party to steal off-chain funds.

  Payment routing~\cite{spider,prihodko2016flare,lee2020routee} is another research area that aims to improve the network efficiency without sacrificing  privacy. Actively rebalancing channels~\cite{DBLP:conf/ccs/KhalilG17} can
  further increase network efficiency by preventing routes from becoming   unavailable due to lack of well-balanced funds.

  An alternantive payment channel construction that aspires to be the successor
  of Lightning is eltoo~\cite{eltoo}. It has a conceptually simpler
  construction, smaller on-chain footprint and a more forgiving attitude towards
  submitting an old channel state than Lightning, but it needs the
  \texttt{ANYPREVOUT} sighash flag to be added to Bitcoin. Generalized
  Bitcoin-Compatible Channels~\cite{cryptoeprint:2020:476} enable the creation
  of state channels on Bitcoin, extending channel functionality from simple
  payments to arbitrary Bitcoin scripts.

  Sprites~\cite{sprites} leverages the scripting language of Ethereum to
  decrease the time collateral is locked up compared to Lightning.
  Perun~\cite{perun} and GSCN~\cite{DBLP:conf/ccs/DziembowskiFH18} exploit the
  Turing-complete scripting language of Ethereum to provide virtual state
  channels, i.e.\ channels that can open without an on-chain transaction and that
  allow for arbitrary scripts to be executed off-chain. Similar features are
  provided by Celer~\cite{dong2018celer}. Hydra~\cite{cryptoeprint:2020:299}
  provides state channels for the Cardano~\cite{cardano} blockchain which
  combines a UTXO type of model with general purpose smart contract
  functionality that are also isomorphic, i.e.\ Hydra channels can accommodate
  any script that is compatible with the underlying blockchain.

  BDW~\cite{scalable-funding} shows how pairwise channels over Bitcoin can be   funded with no on-chain transactions by allowing parties to form groups that   can pool their funds together off-chain and then use those funds to open   channels. ACMU~\cite{10.1145/3319535.3345666} allows for multi-path atomic   payments with reduced collateral, enabling new applications such as   crowdfunding conditional on reaching a funding target.

  TEE-based~\cite{zhao2019sok}
solutions~\cite{teechan,10.1145/3341301.3359627,liao2021speedster,lee2020routee}
  improve the throughput and efficiency of PCNs by an order of magnitude or
  more, at the cost of having to trust TEEs. Brick~\cite{avarikioti2020brick}
  uses a partially trusted committee to extend PCNs to fully asynchronous
  networks.

  Solutions alternative to PCNs include 
side\-chains (e.g.,~\cite{BCDF+14,sidechains,KiaZin18}),
non-custo\-dial chains (e.g.,~\cite{plasma,konstantopoulos2019plasma,plasma-lower-bounds,rollup}),
  and partially centralised payment networks that entirely avoid using a blockchain~\cite{DBLP:conf/trust/ArmknechtKMYZ15,stellar,silentwhispers,DBLP:conf/ndss/RoosMKG18}. 

  Last but not least, a number of works propose virtual channel constructions
  for Bitcoin. Lightweight Virtual Payment
  Channels~\cite{10.1007/978-3-030-65411-5_18} enables a virtual channel to be
  opened on top of two preexisting channels and uses a technique similar to DMC.
  Let ``simple channels'' be those built directly on-chain, i.e.\ channels that are not
  virtual.
  Bitcoin-Compatible Virtual Channels~\cite{cryptoeprint:2020:554} also enables
  virtual channels on top of two preexisting simple channels
  and offers two protocols, the first of which guarantees that the channel will
  stay off-chain for an agreed period, while the second allows the single intermediary
  to turn the virtual into a simple channel.
  We remark that the above strategy has the shortcoming that even if it is made
  recursive (a direction left open in~\cite{cryptoeprint:2020:554}) after $k$
  applications of the constructor the virtual channel participant will have to
  publish on-chain $k$ transactions in order to close the channel if all
  intermediaries actively monitor the blockchain.

  Furthermore, Donner~\cite{donner} is the first work to achieve variadic
  virtual channels without the need for recursion nor features that are not yet
  available in Bitcoin. This is achieved by having the funder use funds that are
  external to the ``base channels'' (i.e.\ the channels that the virtual channel is
  based on), so a party that has all its coins in channels cannot fund a Donner
  channel; additionally, we conjecture that using external coins precludes
  variadic virtual channel designs that are not encumbered with limited
  lifetime. Donner also relies on placeholder outputs which, due to the minimum
  coins they need to carry to exceed Bitcoin's ``dust limit'', may skew the
  incentives of rational players and adds to the opportunity cost of maintaining
  a channel. Furthermore, its design complicates future iterations that lift its
  current restriction that only one of the two channel parties can fund the
  virtual channel. Donner is more efficient than the present work in terms of
  storage, computation and communication complexity, and boasts a simpler
  design, but has less room for optimisations.

  We refer the reader to Table~\ref{table:comparison-features} for a comparison of the
  features and limitations of virtual channel protocols, including the one put
  forth in the current work.

  \begin{table*}
    \caption{Features \& requirements comparison of virtual channel protocols}
    \label{table:comparison-features}
    \begin{minipage}{\textwidth}
    \begin{center}
    \begin{tabular}{|l|c|c|c|c|}
    \hline
              & Unlimited lifetime & Recursive & Variadic & Script requirements \\
    \hline
    LVPC~\cite{10.1007/978-3-030-65411-5_18}
              & ✗                  & \LEFTcircle\footnote{lacks security analysis}
                                               & ✗         & Bitcoin \\
    \hline
    BCVC~\cite{cryptoeprint:2020:554}
              & ✓                  & ✗         & ✗         & Bitcoin \\
    \hline
    Perun~\cite{perun}
              & ✓                  & ✗         & ✗         & Ethereum \\
    \hline
    GSCN~\cite{DBLP:conf/ccs/DziembowskiFH18}
              & ✓                  & ✓         & ✗         & Ethereum \\
    \hline
    Donner~\cite{donner}
              & ✗                  & ✗         & ✓         & Bitcoin \\
    \hline
    this work & ✓                  & ✓         & ✓         & Bitcoin + \texttt{ANYPREVOUT} \\
    \hline
    \end{tabular}
    \end{center}
    \end{minipage}
  \end{table*}
