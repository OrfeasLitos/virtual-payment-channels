\section{Protocol Description}
  Conceptually, Elmo is split into four actions: channel opening,
  payments, cooperative closing and unilateral closing. Parties $P_1$ and $P_n$
  may open a channel $(P_1, P_n)$ between them directly on-chain, in which case
  they follow an opening procedure like that of LN; such a channel is called
  \emph{simple} and is explaned in more detail below.
  Otherwise they can open it on top of a path
  of preexisting \emph{base} channels $(P_1, P_2)$, $(P_2, P_3)$, $\dots$,
  $(P_{n-1}, P_{n})$, in which case $(P_1, P_n)$ is a \emph{virtual}
  channel, also explained later. A channel is either simple or virtual, not both. Since Elmo is
  recursive, each base channel may itself be simple or
  virtual. To open a virtual
  channel, all parties on the
  path set aside funds in their channels as
  collateral; they do this
  by creating so-called \emph{virtual} transactions (txs) that essentially
  tie the spending of two adjacent base channels into a single atomic action.
  Once intermediaries are done, they have created a special \emph{funding}
  output off-chain with the sum of $P_1$ and $P_n$'s channel
  balance. $P_1$ and $P_n$
  finally create the channel, applying the logic of simple channels on top of
  the funding output: their channel is now open. LN demands that the funding
  output is on-chain, but we lift this requirement. We instead guarantee that
  either endpoint can put the funding output on-chain unilaterally.
  A payment over an established channel (described later for simple channels) follows a procedure inspired by
  LN.

  Parties $P_1, \dots, P_n$ can optimistically close a virtual channel
  completely off-chain. At a high
  level, parties controlling the base channels \emph{revoke} their \emph{virtual}
  txs and the related \emph{commitment} txs, i.e., they cannot use them
  anymore. This effectively ``peels'' one \emph{virtualisation layer}.
  In the process, they redistribute coins so that intermediaries ``break even'',
  while $P_1$
  and $P_n$ get their rightful coins (as per the last virtual channel
  state) in their base channels ($(P_1, P_2)$ and $(P_{n-1}, P_n)$).

  Finally, the unilateral closing procedure of a virtual channel $(P_1, P_n)$
  consists of either $P_1$ or $P_n$ signing and publishing a number of
  txs on-chain. In the simplest case, $P_1$ publishes her virtual tx. This
  prompts $P_2$ to publish her
  virtual tx as well and so on up to $P_{n-1}$, at which point the
  funding output of $(P_1, P_n)$ is automatically on-chain and closing can
  proceed as for simple channels. If instead any intermediary stays inactive, then a timelock
  expires and causes a suitable output to become the funding output for $(P_1,
  P_n)$, at
  the expense of the inactive party. As discussed later, the funding output
  employs \texttt{ANYPREVOUT} so that the channel needs only a
  single commitment tx per endpoint, avoiding an exponential state blowup in the
  recursion depth and making off-chain payments efficient.

  In more detail, during the channel opening procedure (cf.
  Fig.~\ref{code:ln:open}) the two counterparties (i) create new keypairs and
  exchange the resulting public keys, then (ii) if the channel is
  virtual, prepare the underlying base channels, next (iii) they exchange
  signatures for their initial commitment txs
  and lastly, (iv) if the channel is simple, the \emph{funder} (who is the only
  party that provides coins for a new channel, like in LN) signs and publishes
  the \emph{funding} tx to the ledger.

  In order to build intuition, let us present examples of the
  lifecycles of a simple and a virtual channel. Consider $5$ parties, $A, B,
  \dots, E$ and $4$ channels, $(A, B)$, $\dots,(D, E)$, that will act as
  the base of the virtual channel $(A, E)$. We first follow the operations of
  the simple channel $(A, B)$ and then those of $(A, E)$. We simplify some parts
  of the protocol to aid comprehension --- see Appx.~\ref{sec:description} for
  more details.

  \makeatletter%
  \@ifclassloaded{IEEEtran}%
    {\paragraph{Simple channel}}%
    {\paragraph{Simple channel.}}%
  \makeatother%
  % splncs
  \addtolength{\intextsep}{-15pt}
  \begin{figure}[!htbp]
    \centering
    \subimport{./figures/manual-tikz/}{payment-layer-simple.tex}
    \caption{Left to right: funding ($F$), $A$'s commitment ($C_A$) and $B$'s
    revocation ($R_B$) txs.
    The symmetric commitment and
    revocation txs of $B$ and $A$ respectively are not shown.
    %$A \wedge B$ needs a signature by both $A$ and $B$,
    $A_F$ is $A$'s funding key, $A_R$ her revocation key. $A+t$ needs a signature
    by $A$ after relative timelock $t$. The first $C_A$ output is
    spendable either by both $A_R$ and $B_R$, or by $A + t$. Inputs and outputs are
    represented by separate arrows: The input of $C_A$ spending the output of
    $F$ is shown as two connected arrows.}%, with the ``or'' denoted with $\vee$.}
    \label{figure:payment-layer-simple}
  \end{figure}
  % splncs
  \addtolength{\intextsep}{15pt}

  First $A$ and $B$ generate funding and revocation keypairs and exchange
  public keys ($A_F, B_F, A_R, B_R$ in Fig.~\ref{figure:payment-layer-simple} --
  $2$ messages). Each then
  locally generates the \emph{funding} and the two \emph{commitment} txs
  ($F, C_A$ in Fig.~\ref{figure:payment-layer-simple}). They
  sign $C_A$ and exchange the
  signatures ($2$ messages). $A$ then publishes $F$ on-chain. Once it
  is finalised, the channel is open.

  The funding tx $F$ moves $A$'s initial coins to a $2$-of-$2$ multisig, i.e., an
  output that needs signatures from both $A_F$ and $B_F$ to be spent. There is one
  commitment tx per party, stored locally off-chain. The one held by $A$
  ($C_A$ in Fig.~\ref{figure:payment-layer-simple}) spends the funding tx
  and has one output for $A$ (initially with all coins) and one for $B$
  (initially with $0$ coins). $A$'s output can be spent by either a multisig, or
  by $A$ after a \emph{relative timelock} of $t$ (relative means that the countdown starts at the moment of
  publication). This is, as we will promptly see, so
  that $B$ has time to \emph{punish} $A$ if she cheats. $B$'s commitment tx
  $C_B$ is symmetric.

  When $A$ pays $c$ coins to $B$ in the channel, $A$ and $B$ create two new
  commitment txs.
  They have the same outputs and scripts as their previous ones, save for the
  coins: $A$'s outputs have $c$ coins less, $B$'s outputs have $c$ coins more.
  They sign them and swap signatures. In order to ensure only one set of
  commitment txs is valid at a time, they then revoke their previous ones. To
  that end, they build and sign revocation txs for the previous
  commitment txs. A payment needs $3$ messages in total. $B$'s revocation tx ($R_B$ in
  Fig.~\ref{figure:payment-layer-simple}) gives $B$ all coins that belonged
  to $A$ and vice versa. This way both parties are
  disincentivised from publishing an old commitment tx under the threat of
  losing all their channel coins.

  $A$ or $B$ can now unilaterally close $(A, B)$ by simply publishing the latest
  commitment tx on-chain and waiting for the timelock to expire. Since the last
  commitment tx is not revoked, punishment is impossible. The
  mechanics of simple channels are essentially a simplification of LN.

  \makeatletter%
  \@ifclassloaded{IEEEtran}%
    {\paragraph{Virtual channel}}%
    {\paragraph{Virtual channel.}}%
  \makeatother%
  % splncs
  \addtolength{\intextsep}{-15pt}
  \begin{figure*}[!htbp]
    \begin{minipage}{.47\textwidth}
    \subimport{./figures/manual-tikz/}{intermediary-initiator-simple.tex}
    \caption{$A$--$E$ virtual channel: $D$'s initiator transaction. Spends the
    funding outputs of the $C$--$D$ and $D$--$E$ channels. $D$ can use it if neither
    $C$ nor $E$ have published a virtual transaction yet. $A_V$: $A$'s
    ``virtual'' key. $A_b$: $A$'s ``bridge'' key.}
    \label{figure:virtual-layer-initiator-simple}
    \end{minipage}\hspace{23pt}%
    \begin{minipage}{.48\textwidth}
    \subimport{./figures/manual-tikz/}{intermediary-extend-interval-simple.tex}
    \caption{$A$--$E$ virtual channel: One of $C$'s extend-interval
    transactions. Spends the virtual output of $D$'s
    initiator transaction and the funding output of the $B$--$C$ channel. $C$ can
    use it if $D$ has already published its initiator transaction and $B$ has
    not published a virtual transaction yet.}
    \label{figure:virtual-layer-extend-interval-simple}
    \end{minipage}
  \end{figure*}
  % splncs
  \addtolength{\intextsep}{15pt}

  Assume now that channels $(A, B)$, $\dots,(D, E)$, are open and the ``left''
  party of each owns at least $c$ coins in it. These channels can be either
  simple or virtual -- in the latter case, the virtual channel $(A, E)$ will
  leverage the recursive property --
  thanks to the similarity of all layers, the description below is identical in
  both cases. In order for $(A, E)$ to open using $(A, B)$,
  $\dots,(D, E)$, as base channels, initially with $A$ having $c$ coins, the
  $5$ parties act as follows. First, they generate and exchange a number of new
  keys (i.e., all keys that appear in the outputs of
  Figs.~\ref{figure:virtual-layer-initiator-simple}--\ref{figure:virtual-layer-merge-intervals-simple}). Then
  each base pair removes $c$ coins from the ``left'' party in
  their base channel. The updated commitment txs use some of the new
  keys for the multisig in their input, since, as we will see, so-called \emph{virtual}
  txs will stand between the funding and the commitment txs from now on. These
  virtual txs will form the \emph{virtual layer} (Fig.~\ref{figure:example-start-end-simple}).

  Next, parties generate and sign the virtual txs and send the signatures among
  them. These txs sit at
  the core of
  Elmo. Intuitively, they force each intermediary to interact with both of its
  base channels instead of one at a time, ensuring that information relevant to
  the virtual channel ``flows'' along the base channels. There are $3$ virtual
  tx types -- looking ahead, we now explain them via example of their use during
  channel closing. The \emph{initiator} virtual tx spends the two funding outputs of an
  intermediary (say $D$) and thus needs to be signed by $C_F$, $D_F$, and $E_F$.
  It produces $4$ outputs: one new funding
  output for each of $(C, D)$ and $(D, E)$
  (Fig.~\ref{figure:virtual-layer-initiator-simple}, top \& bottom outputs), one
  output that refunds the collateral to $D$ ($2$nd from top) and,
  crucially, a so-called \emph{virtual} output ($3$rd from top). The latter
  can be spent by $C$ with an \emph{extend-interval} tx, the second type of virtual
  tx (Fig.~\ref{figure:virtual-layer-extend-interval-simple}), which needs
  signatures from all $5$ parties (top input). This tx also
  spends the other, as-of-yet unspent, $C$'s funding output, namely that of
  $(B,C)$ (bottom
  input). It has $3$ outputs: one refunding the collateral to $C$
  (top), another virtual output (middle), and a funding output that replaces the
  one just spent (bottom). In our example, $A$ also uses its initiator tx, which
  is different since $A$ is an endpoint (Fig.~\ref{figure:virtual-layer-endpoint-simple}). It spends only the funding output of
  $(A, B)$ and produces $2$ outputs: a new funding output for $(A, B)$ (top) and
  a virtual output (bottom) -- this is the only virtual tx $A$ needs. $B$ can in
  turn spend $A$'s and $C$'s virtual outputs with a
  \emph{merge-intervals} tx
  (Fig.~\ref{figure:virtual-layer-merge-intervals-simple}), the last virtual tx
  type, which also needs signatures from the \emph{virtual keys} $A_V \dots E_V$
  of all $5$ parties. It has $2$ outputs: one refunds $B$ and
  the other produces a new virtual
  output. Now
  all base funding outputs are spent, all intermediaries are refunded and the
  virtual layer has ``collapsed''. Additionally, the
  virtual output of $B$'s tx plays the role of the funding output of the virtual
  channel $(A, E)$.

  The virtual txs are designed around two axes: First, each intermediary can
  publish only one virtual tx, which refunds its collateral
  exactly once. We will see how this is enforced below. Second, if the chain of
  virtual txs is at any point broken by,
  e.g., an inactive intermediary that does not publish its virtual tx, the
  virtual channel will still be funded correctly and the inactive party will
  lose its collateral. This is guaranteed because
  the unclaimed virtual output automatically turns into the funding output of
  the virtual channel
  after a timelock. See, e.g., $2$nd spending method of the $3$rd output of
  Fig.~\ref{figure:virtual-layer-initiator-simple}. Keys $A_b \dots E_b$ can be
  spent by \emph{bridge} txs, the output of which funds the $(A, E)$ channel.
  Bridge txs are discussed in Appx.~\ref{subsec:description-virtual-channels}.

  % splncs
  \addtolength{\intextsep}{-15pt}
  \begin{figure}[!htbp]
    \subimport{./figures/manual-tikz/}{virtual-layer-endpoint-simple.tex}
    \caption{$A$--$E$ virtual channel: $A$'s initiator transaction. Spends the
    funding output of the $A$--$B$ channel. $A$ can use it if $B$ has not published
    a virtual transaction.}
    \label{figure:virtual-layer-endpoint-simple}
  \end{figure}
  % splncs
  \addtolength{\intextsep}{15pt}

  Some considerations remain before we can ascertain the security of the
  scheme. Firstly, we must ensure that no intermediary can publish more than
  one virtual tx to protect the endpoints from an unbounded sequence of virtual
  txs preventing them from accessing their funding output indefinitely -- note
  that malicious parties can fabricate arbitrarily many virtual outputs using
  their own, external to the protocol, coins, therefore if all virtual outputs
  were identical, an adversary could publish a perpetual stream of
  merge-intervals txs, spending one valid and one fabricated virtual output.
  This is safeguarded by
  specifying on each virtual output the exact sequence of parties that have
  already published a virtual tx and only allowing the parties at the two edges
  of the sequence
  to extend it with their virtual tx. If all intermediaries publish a virtual
  tx, then the last virtual output that was published is not spendable by
  another virtual
  transaction. This ensures that the endpoints will eventually obtain a funding
  output. Preventing this attack means that intermediaries need to store
  $O(n^3)$ virtual transactions for a virtual channel over $n$ parties. Secondly,
  we have to carefully select the exact values of timelocks to ensure that
  each party has enough time to act. The timelocks increase linearly
  with the depth of the recursion. The exact values are shown in
  Sec.~\ref{sec:pseudocode} and Appx.~\ref{sec:protocol}.

  % splncs
  \addtolength{\intextsep}{-15pt}
  \begin{figure}[!htbp]
    \subimport{./figures/manual-tikz/}{intermediary-merge-intervals-simple.tex}
    \caption{$A$--$E$ virtual channel: One of $B$'s merge intervals
    transactions. Spends the virtual outputs of $A$'s and $C$'s virtual
    transactions. $B$ can use it if both $A$ and $C$ have already published
    their initiator or extend-interval transactions.}
    \label{figure:virtual-layer-merge-intervals-simple}
  \end{figure}
  % splncs
  \addtolength{\intextsep}{15pt}

  We now return to the opening procedure. After the $5$ parties set up the virtual txs, they revoke their previous
  commitment
  txs. They do this by signing the relevant revocation txs, just like
  for a simple channel. It takes $12 \cdot (n-1)$ messages, i.e., $6$ messages
  per endpoint and $12$ messages per intermediary, to set up the virtual layer.

  % splncs
  \addtolength{\intextsep}{-15pt}
  \begin{figure*}
    \centering
    \subimport{./figures/manual-tikz/}{example-start-end-simple.tex}
    \caption{$4$ simple channels supporting a virtual one. $A$ and $D$ start
    closing by publishing their initiator virtual txs, then $C$ publishes its
    suitable extend-interval virtual tx, after which $B$ publishes its suitable
    merge-intervals virtual tx. No party stays inactive. The virtual
    transactions $A$--$D$ form the virtual layer. Virtual outputs are marked
    with the set (interval) of parties that have already published a tx.
    \emph{Bridge} txs like $b$ are used by $A$ and $E$ to convert the various
    virtual outputs into the same funding output, as \texttt{ANYPREVOUT} only
    works across identical outputs.}
    \label{figure:example-start-end-simple}
  \end{figure*}
  % splncs
  \addtolength{\intextsep}{15pt}

  At last, the $5$ parties have set up virtual layer: Both $A$ and $E$ can
  unilaterally
  force the funding output of their virtual channel on-chain, irrespective of
  the actions of the rest of the parties. Likewise, honest intermediaries can
  unilaterally retrieve their funds.

  $A$ and $E$ finally exchange
  commitment transactions for their new channel, thus concluding its opening.
  $A$ and $E$ can pay each other over their virtual channel exactly like
  they would over a simple channel; we refer the reader to the relevant
  description above.

  Note that funding outputs use the \texttt{ANYPREVOUT} flag, thus ensuring
  that a single pair of commitment txs can spend any of the funding outputs. If
  \texttt{ANYPREVOUT} were not used, each virtual layer would need a copy of the
  entire set of discussed txs for each possible funding output of its base
  layer, resulting in exponential storage requirements. To make matters worse,
  a payment over $(A, E)$ would need $A$ and $E$ to renegotiate exponentially
  many commitment txs, as well as recalculate all their downstream txs, which
  would in turn need interaction with intermediaries of all virtual channels
  built over $(A, E)$, completely defeating the essence of payment channels.

  \makeatletter%
  \@ifclassloaded{IEEEtran}%
    {\paragraph{Cooperative closing}}%
    {\paragraph{Cooperative closing.}}%
  \makeatother%
  To enhance usability, our protocol allows the virtual channel to be closed
  off-chain, given that all parties cooperate. To do this, the endpoints first
  let the intermediaries know their final virtual channel balance. Then the
  parties of each base channel create new commitment txs for their channels,
  moving the collateral back into the channel: the ``left'' party gets $A$'s
  coins and the ``right'' one gets $E$'s. Thus all intermediaries ``break even''
  across their two channels. Once this is done, the parties revoke all virtual
  txs, using a logic similar to the revocation procedure of simple channels but
  scaled up to all parties. This is why all virtual tx outputs
  (Figs.~\ref{figure:virtual-layer-initiator-simple}--\ref{figure:virtual-layer-merge-intervals-simple})
  have a spending method with $A_R \dots E_R$ keys.

  What if one party does not cooperate? Then one or more of the other parties must close
  unilaterally on-chain. Fig.~\ref{figure:example-start-end-simple} shows how
  this would play out if $A$ and $D$ initiated this procedure.

  Our protocol is recursive because both simple and virtual channels are
  ultimately represented by a funding output that either already is or can be put
  on-chain, therefore new virtual channels can be built on either.
  Both simple and virtual channels avoid key reuse on-chain, thus ensuring party
  privacy from on-chain observers.
