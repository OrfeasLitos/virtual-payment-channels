\section{High Level Explanation}
  Conceptually, our protocol is split into three main actions: channel opening,
  payments and closing. A channel $(P_1, P_n)$ between parties $P_1$ and $P_n$
  may be opened directly on-chain, in which case the two parties follow an
  opening procedure similar to that of LN, or it can be opened on top of a path
  of preexisting channels $(P_2, P_3), (P_3, P_4), \dots, (P_{n-3}, P_{n-2}),
  (P_{n-2}, P_{n-1})$. In the latter case all parties $P_i$ on the path follow
  our novel protocol, setting aside funds in their channels as collateral for
  the new virtual channel that is being opened. Once all intermediaries are
  committed, $P_1$ and $P_n$ finally create (and keep off-chain) their
  ``commitment'' transaction, following a logic similar to Lightning.

  A payment over an established channel follows a procedure heavily inspired by
  LN, but without the use of HTLCs. To be completed, a payment needs three
  messages to be exchanged by the two parties.

  Finally, the closing procedure can be completed unilaterally and consists of
  signing and publishing a number of transactions on-chain. As we will discuss
  later, the exact transactions that a party will publish vary depending on the
  exact actions of the other on-path parties. Our protocol can be augmented with
  a more efficient optimistic collaborative closing procedure, which however is
  left as future work.

  In more detail, to open a channel (c.f.~\ref{code:ln:open}) the two
  counterparties (a.k.a. ``endpoints'') first create new keypairs and exchange
  the resulting public keys ($2$ messages), then prepare the underlying base
  channels if the new channel is virtual ($10(n-1)$ total messages, i.e. $5$
  outgoing messages per endpoint and $10$ outgoing messages per intermediary
  with the number of intermediaries being $n-2$), next they exchange signatures
  for their respective initial commitment transactions ($2$ messages) and
  lastly, if the channel is simple (i.e. not virtual), the ``funder'' signs and
  publishes the ``funding'' transaction on-chain.  We here note that like LN,
  only one of the two parties, the funder, provides coins for a new channel.
  This limitation simplifies the execution model and the analysis, but can be
  lifted at the cost of additional protocol complexity.

  In a similar vein to earlier PCN proposals, having an open channel essentially
  means having very specific keys, transactions and signatures at hand, as well
  as checking the ledger periodically and being ready to take action if
  misbehaviour is detected. Let us first consider a simple channel that has been
  established between \alice and \bob where the former owns $c_A$ and the latter
  $c_B$ coins. There are three sets of transactions at play: A ``funding''
  transaction that is put on-chain, off-chain ``commitment'' transactions that
  spend the funding output and off-chain ``revocation'' transactions that spend
  commitment outputs in case of misbehaviour. \TODO{add figure}

  In particular, there is a single on-chain funding transaction with a single
  output that is encumbered with a $2/\{\pk{A}, \pk{B}\}$ multisig and carries
  $c_A + c_B$ coins.

  Next, there are two commitment transactions, each of which can spend the
  funding tx and produce two outputs with $c_A$ and $c_B$ coins each. The two
  txs differ in the outputs' spending conditions: The $c_A$ output in \alice's
  commitment tx can be spent either by \alice after it has been on-chain for a
  pre-agreed period (i.e. it is encumbered with a ``timelock''), or by a
  ``revocation'' transaction (discussed below) via a 2-of-2 multisig between the
  counterparties, whereas the $c_B$ output can be spent only by \bob without a
  timelock. \bob's commitment tx is symmetric: the $c_A$ output can be spent
  only by \alice without timelock and the $c_B$ output can be spent either by
  \bob after the timelock expiration or by a revocation tx.  When a new pair of
  commitment txs are created (either during channel opening or on each update)
  \alice signs \bob's commitment tx and sends him the signature (and
  vice-versa), therefore \alice can unilaterally sign and publish her commitment
  tx but \bob's (and vice-versa).

  Last, there are $2m$ revocation transactions, where $m$ is the total number of
  updates of the channel. The $j$th revocation tx held by an endpoint spends the
  output carrying the counterparty's funds in the counterparty's $j$th
  commitment tx. It has a single output spendable immediately by the
  aforementioned endpoint. Each endpoint stores $m$ revocation txs, one for each
  superseded commitment tx. This creates a disincentive for an endpoint to cheat
  by using any other commitment transaction instead of its most recent one to
  spend the funding output: the timelock on the output permits its counterparty
  to use the corresponding revocation transaction and thus claim the cheater's
  funds. Endpoints do not have a revocation tx for the last commitment
  transaction, therefore these can be safely published. For a channel update to
  be completed, the endpoints must exchange the signatures for the revocation
  txs that spend the commitment txs that just became obsolete.

  Observe that the above logic is essentially a simplification of LN.
