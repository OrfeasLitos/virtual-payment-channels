\section{Protocol Pseudocode}
\label{sec:pseudocode}

We here present a simplified version of the protocol in pseudocode form. We omit
complications imposed by UC. We refer the reader to Appendix,
Section~\ref{sec:protocol} for the complete protocol and to
Section~\ref{sec:construction} for an in-depth description of the protocol in
prose.
\ \\

\begin{center}
  \begin{processbox}{\pchan -- self is $P$}
    \begin{itemize}
      \item Before handling each message:
      \begin{algorithmic}[0]
      % TODO: change 0 to 1 if line numbers needed in all algorithm snippets.
      % Due to number duplication, we may need to scrap itemize and have one big
      % algorithmic instead
        \If{we have not been activated since more than $p$ blocks}
          \State mark ourselves as negligent \Comment{no balance security
          guarantees anymore}
        \EndIf
      \end{algorithmic}

      \item Initialisation:
      \begin{algorithmic}[0]
        \State Receive $\pk{P, \mathrm{out}}$ from \environment{} \Comment{all
        outputs owned by $P$ pay $\pk{P, \mathrm{out}}$}
        \State generate own keypair
        \State wait for \environment to give own keypair some starting coins
      \end{algorithmic}

      \item Opening:
      \begin{algorithmic}[0]
        \State generate funding and revocation keypairs
        \State exchange funding, revocation and out public keys with
        counterparty
        \If{opening virtual (off-chain) channel}
          \State ask our host channel to prepare, passing them our funding keys
          \Comment{c.f.\ ``Hosting a virtual channel''}
          \State get $t_P$ from host \Comment{timelock to ensure our balance
          security}
        \EndIf
        \State exchange and verify signatures on commitment transactions with
        counterparty
        \If{opening simple (on-chain) channel}
          \State prepare and submit funding transaction to ledger and wait for
          its inclusion \Comment{only one party funds the channel, so the
          funding transaction needs only the funder's signature}
          \State $t_P \gets s + p$ \Comment{timelock to ensure balance security
          for simple channels}
        \EndIf
      \end{algorithmic}

      \item Hosting a virtual channel of $c_{\mathrm{virt}}$ coins:
      \begin{algorithmic}[0]
        \State ensure we have enough coins to host such a virtual channel
        \State generate one new funding keypair, $O(n^2)$ virtual keypairs
        ($O(n)$ per hop) and one virtual revocation keypair \Comment{all
        keypairs are generated normally, using \textsc{keyGen}()}
        \State exchange generated public keys among all base channel parties
        \State generate and sign new commitment transactions with our
        counterparties. The new funding keys and the latest revocation keys are
        used and the balance of the party ``closer'' to the funder is reduced by
        $c_{\mathrm{virt}}$ \Comment{$1$ counterparty if we are endpoint, $2$
        counterparties if we are intermediary}
        \State exchange signatures with counterparties and verify them
        \State generate and sign all $O(n^3)$ virtual transactions \Comment{one
        signature for each virtual input -- each virtual input needs one
        signature from each party. Only ``extend-interval'' and
        ``merge-intervals'' transactions need these signatures}
        \State exchange all signatures among all base channel parties and verify
        that all our virtual transactions have fully signed virtual inputs
        \State exchange with counterparties and verify signatures for the
        funding inputs of our virtual transactions \Comment{only ``initiator''
        and ``extend-interval'' transactions need these signatures}
        \State exchange with counterparties and verify signatures for the
        revocation transactions of the previous channel state
        \If{$P$ is intermediary}
          \State $t_P \gets \max\{t \text{ of left channel}, t \text{ of right
          channel}\}$
        \Else \: \Comment{$P$ is endpoint}
          \State $t_P \gets p + \sum\limits_{j = 2}^{n-1}(s - 1 + t_j)$
          \Comment{worst case delay is if counterparty uses initiator tx and
          every intermediary uses its extend-interval tx sequentially -- the
          maximum possible delay is $O(\text{sum of intermediaries' delays})$}
        \EndIf
      \end{algorithmic}

      \item Reacting if counterparty publishes virtual transaction:
      \begin{algorithmic}[0]
        \If{both our counterparties have published a virtual transaction}
          \State publish our merge-intervals transaction that has an interval
          equal to the union of the intervals of the two virtual transactions
          plus ourselves
        \Else \: \Comment{exactly one of our counterparties has published a
        virtual transaction}
          \State publish our extend-interval transaction that has an interval
          equal to the interval of the virtual transaction plus ourselves
        \EndIf
      \end{algorithmic}

      \item Paying $x$ coins:
      \begin{algorithmic}[0]
        \State ensure we have enough coins to pay
        \If{we host a virtual channel}
          \State ensure balance after payment will not allow griefing attack
          \Comment{c.f.\ Subsubsection~\ref{construction:ln}}
        \EndIf
        \State generate and sign new commitment transactions, with $x$ coins
        less for the payer and $x$ coins more for the payee and using the latest
        revocation keys
        \State exchange and verify signatures
        \State sign revocation transactions that correspond to the old
        commitment transactions
        \State generate next revocation keypairs
        \State exchange and verify signatures and revocation public keys
      \end{algorithmic}

      \item Unilaterally closing:
      \begin{algorithmic}[0]
        \State publish all initiator transactions that are needed to put our
        funding output on the ledger
        \State publish our latest commitment transaction to the ledger
      \end{algorithmic}

      \item Cooperatively closing:
      \begin{algorithmic}[0]
        \State \Comment{only a virtual channel which does not host any further
        virtual channel may close cooperatively}
        \State both endpoints sign and broadcast the final virtual channel
        balance $(c_1, c_2)$
        \State every party verifies both signatures, ensures that the two
        opinions agree and that the balance sum is equal to $c_{\mathrm{virt}}$
        \State generate and sign new commitment transactions with:
        \begin{itemize}
          \item the most recent old funding keys (the ones used before hosting
          the virtual channel)
          \item the new revocation keys
          \item $c_1$ additional coins for the party closest to the virtual
          channel funder and $c_2$ for the counterparty
        \end{itemize}
        \State generate new revocation keypairs
        \State exchange and verify signatures and revocation public keys
        \State generate and sign revocation transactions for both the old
        virtual transactions (with the virtual revocation keys) and the old
        commitment transactions (with the normal revocation keys)
        \State exchange and verify signatures \Comment{if a party publishes a
        revoked virtual transaction, its various outputs can be spent by
        revocation transactions so that its ($1$ or $2$) counterparties can
        claim all base channel funds}
      \end{algorithmic}

      \item Punishing malicious counterparties:
      \begin{algorithmic}[0]
        \State \Comment{executed at least every $p$ blocks}
        \If{the ledger contains an old commitment transaction}
          \State sign and publish the corresponding revocation transaction
        \EndIf
        \If{the ledger contains an old virtual transaction}
          \State sign and publish the corresponding revocation transaction(s)
        \EndIf
      \end{algorithmic}
    \end{itemize}
  \end{processbox}
  \captionof{figure}{High level pseudocode of the Elmo protocol}
  \label{code:pseudocode}
\end{center} \ \\
