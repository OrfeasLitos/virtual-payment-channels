\begin{proof}[Proof of Lemma~\ref{lemma:no-halt}]
  We prove the Lemma in two steps. We first show that if the conditions of
  Lemma~\ref{lemma:ideal-balance} hold, then the conditions of
  Lemma~\ref{lemma:real-balance-security} for the real world execution with
  protocol \textsc{ln} and the same \environment and \adversary hold as well for
  the same $k, m, n$ and $l$ values.

  For $\itistate_P$ to become $\textsc{ignored}$, either \simulator has to send
  (\textsc{became corrupted or negligent}, $P$) or $\texttt{host}_P$ must output
  (\textsc{enabler used revocation}) to \fchan
  (Fig.~\ref{code:functionality:open},
  l.~\ref{code:functionality:open:when-ignored}). The first case only happens
  when either $P$ receives (\textsc{corrupt}) by \adversary
  (Fig.~\ref{code:simulator}, l.~\ref{code:simulator:when-corrupted}), which
  means that the simulated $P$ is not honest anymore, or when $P$ becomes
  \texttt{negligent} (Fig.~\ref{code:simulator},
  l.~\ref{code:simulator:when-negligent}), which means that the first condition
  of Lemma~\ref{lemma:real-balance-security} is violated. In the second case, it
  is $\texttt{host}_P \neq \ledger$ and the state of $\texttt{host}_P$ is
  \textsc{guest punished} (Fig.~\ref{code:virtual-layer:punishment},
  ll.~\ref{code:virtual-layer:punishment:when-punished-1}
  or~\ref{code:virtual-layer:punishment:when-punished-2}), so in case $P$ receives
  (\textsc{forceClose}) by \environment the output of $\texttt{host}_P$ will be
  (\textsc{guest punished}) (Fig.~\ref{code:virtual-layer:close},
  l.~\ref{code:virtual-layer:close:output-guest-punished}). In all cases, some
  condition of Lemma~\ref{lemma:real-balance-security} is violated.

  For $\itistate_P$ to become \textsc{open} at least once, the following
  sequence of events must take place (Fig.~\ref{code:functionality:open}): If $P
  = \alice$, it must receive (\textsc{init}, $\pk{}$) by \environment when
  $\itistate_P = \textsc{uninit}$, then either receive (\textsc{open}, $c$,
  \ledger, $\dots$) by \environment and (\textsc{base open}) by \simulator or
  (\textsc{open}, $c$, \texttt{hops} ($\neq \ledger$), $\dots$) by \environment,
  (\textsc{funded}, \textsc{host}, $\dots$) by \texttt{hops}[0].\texttt{left}
  and (\textsc{virtual open}) by \simulator. In either case, \simulator only
  sends its message only if all its simulated honest parties move to the
  \textsc{open} state (Fig.~\ref{code:simulator},
  l.~\ref{code:simulator:when-open}), therefore if the second condition of
  Lemma~\ref{lemma:ideal-balance} holds and $P = \alice$, then the second
  condition of Lemma~\ref{lemma:real-balance-security} holds as well. The same
  line of reasoning can be used to deduce that if $P = \bob$, then $\itistate_P$
  will become \textsc{open} for the first time only if all honest simulated
  parties move to the \textsc{open} \textit{state}, therefore once more the
  second condition of Lemma~\ref{lemma:ideal-balance} holds only if the second
  condition of Lemma~\ref{lemma:real-balance-security} holds as well. We also
  observe that, if both parties are honest, they will transition to the
  \textsc{open} state simultaneously.

  Regarding the third Lemma~\ref{lemma:ideal-balance} condition, we assume (and
  will later show) that if both parties are honest and the state of one is
  \textsc{open}, then the state of the other is also \textsc{open}. Each time
  $P$ receives input (\textsc{fund me}, $f$, $\dots$) by $R \in \{\fchan,
  \textsc{ln}\}$, $\itistate_P$ transitions to \textsc{pending fund},
  subsequently when a command to define a new \textsc{virt} ITI through $P$ is
  intercepted by \fchan, $\itistate_P$ transitions to \textsc{tentative fund}
  and afterwards when \simulator sends (\textsc{fund}) to \fchan, $\itistate_P$
  transitions to \textsc{sync fund}. In parallel, if $\itistate_{\bar{P}} =
  \textsc{ignored}$, then $\itistate_P$ transitions directly back to
  \textsc{open}. If on the other hand $\itistate_{\bar{P}} = \textsc{open}$ and
  \fchan intercepts a similar \textsc{virt} ITI definition command through
  $\bar{P}$, $\itistate_{\bar{P}}$ transitions to \textsc{tentative help fund}.
  On receiving the aforementioned (\textsc{fund}) message by \simulator and
  given that $\itistate_{\bar{P}} = \textsc{tentative help fund}$, \fchan also
  sets $\itistate_{\bar{P}}$ to \textsc{sync help fund}. Then both
  $\itistate_{\bar{P}}$ and $\itistate_P$ transition simultaneously to
  \textsc{open} (Fig.~\ref{code:functionality:fund}). This sequence of events
  may repeat any $n \geq 0$ times. We observe that throughout these steps,
  honest simulated $P$ has received (\textsc{fund me}, $f$, $\dots$) and that
  \simulator only sends (\textsc{fund}) when all honest simulated parties have
  transitioned to the \textsc{open} state (Fig.~\ref{code:simulator},
  l.~\ref{code:simulator:when-fund} and Fig.~\ref{code:ln:virtualise:start-end},
  l.~\ref{code:ln:virtualise:start-end:hosts-ready}), so the third condition of
  Lemma~\ref{lemma:real-balance-security} holds with the same $n$ as that of
  Lemma~\ref{lemma:ideal-balance}.

  Moving on to the fourth Lemma~\ref{lemma:ideal-balance} condition, we again
  assume that if both parties are honest and the state of one is \textsc{open},
  then the state of the other is also \textsc{open}. Each time \fchan receives
  (\textsc{coop closing}, $P$, $r$) by \simulator, $\itistate_P$ transitions to
  \textsc{coop closing} and subsequently when \simulator sends (\textsc{coop
  closed}, $P$) to \fchan, if $\sublayer{P} = 0$ then $\itistate_P$ transitions
  to \textsc{coop closed}, else $\itistate_P$ transitions to \textsc{open}. This
  sequence of events may repeat any $k \geq 0$ times. We observe that throughout
  these steps, honest simulated $P$ has transitioned to the \textsc{coop
  closing} state and that \simulator only sends (\textsc{coop closed}, $P$) when
  honest simulated $P$ transitions to either \textsc{open} or \textsc{coop
  closed} state, so the sum of $j$ (from the fourth condition of
  Lemma~\ref{lemma:real-balance-security}) plus $k$ (from the fifth condition of
  Lemma~\ref{lemma:real-balance-security}) is equal to the $k$ of
  Lemma~\ref{lemma:ideal-balance}.

  Regarding the sixth Lemma~\ref{lemma:ideal-balance} condition, we again
  assume that if both parties are honest and the state of one is \textsc{open},
  then the state of the other is also \textsc{open}. Each time $P$ receives
  input (\textsc{pay}, $d$) by \environment, $\itistate_P$ tranisitions to
  \textsc{tentative pay} and subsequently when \simulator sends (\textsc{pay})
  to \fchan, $\itistate_P$ transitions to (\textsc{sync pay}, $d$). In parallel,
  if $\itistate_{\bar{P}} = \textsc{ignored}$, then $\itistate_P$ transitions
  directly back to \textsc{open}. If on the other hand $\itistate_{\bar{P}} =
  \textsc{open}$ and \fchan receives (\textsc{get paid}, $d$) by \environment
  addressed to $\bar{P}$, $\itistate_{\bar{P}}$ transitions to \textsc{tentative
  get paid}. On receiving the aforementioned (\textsc{pay}) message by
  \simulator and given that $\itistate_{\bar{P}} = \textsc{tentative get paid}$,
  \fchan also sets $\itistate_{\bar{P}}$ to \textsc{sync get paid}. Then both
  $\itistate_P$ and $\itistate_{\bar{P}}$ transition simultaneously to
  \textsc{open} (Fig.~\ref{code:functionality:pay}). This sequence of events may
  repeat any $m \geq 0$ times. We observe that throughout these steps, honest
  simulated $P$ has received (\textsc{pay}, $d$) and that \simulator only sends
  (\textsc{pay}) when all honest simulated parties have completed sending or
  receiving the payment (Fig.~\ref{code:simulator},
  l.~\ref{code:simulator:when-pay}), so the sixth condition of
  Lemma~\ref{lemma:real-balance-security} holds with the same $m$ as that of
  Lemma~\ref{lemma:ideal-balance}. As far as the seventh condition of
  Lemma~\ref{lemma:ideal-balance} goes, we observe that this case is symmetric
  to the one discussed for its sixth condition above if we swap $P$ and
  $\bar{P}$, therefore we deduce that if Lemma~\ref{lemma:ideal-balance} holds
  with some $l$, then Lemma~\ref{lemma:real-balance-security} holds with the
  same $l$.

  As promised, we here argue that if both parties are honest and one party moves
  to the \textsc{open} state, then the other party will move to the
  \textsc{open} state as well. We already saw that the first time one party
  moves to the \textsc{open} state, it will happen simultaneously with the same
  transition for the other party. We also saw that, when a party transitions
  from the \textsc{sync help fund} or the \textsc{sync fund} state to the
  \textsc{open} state, then the other party will also transition to the
  \textsc{open} state simultaneously. Additionally, we saw that if one party
  transitions from the \textsc{coop closing} state to the \textsc{open} state,
  the other party will also transition to the \textsc{open} state
  simultaneously. Furthermore, we saw that if one party transitions from the
  \textsc{sync pay} or the \textsc{sync get paid} state to the \textsc{open}
  state, the other party will also transition to the \textsc{open} state
  simultaneously. Lastly we notice that we have exhausted all manners in which a
  party can transition to the \textsc{open} state, therefore we have proven that
  transitions of honest parties to the \textsc{open} state happen
  simultaneously.

  Now, given that \simulator internally simulates faithfully both \textsc{ln}
  parties and that \fchan relinquishes to \simulator complete control of the
  external communication of the parties as long as it does not halt, we deduce
  that \simulator replicates the behaviour of the aforementioned real world. By
  combining these facts with the consequences of the two Lemmas and the check
  that leads \fchan to halt if it fails (Fig.~\ref{code:functionality:close},
  l.~\ref{code:functionality:close:check:if}), we deduce that if the conditions
  of Lemma~\ref{lemma:ideal-balance} hold for the honest parties of \fchan and
  their kindred parties, then the functionality halts only with negligible
  probability.

  In the second proof step, we show that if the conditions of
  Lemma~\ref{lemma:ideal-balance} do not hold, then the check of
  Fig.~\ref{code:functionality:close},
  l.~\ref{code:functionality:close:check:if} never takes place. We first discuss
  the $\itistate_P = \textsc{ignored}$ case. We observe that the
  \textsc{ignored} \textit{State} is a sink state, as there is no way to leave
  it once in. Additionally, for the balance check to happen, \fchan must receive
  (\textsc{closed}, $P$) by \simulator when $\itistate_P \neq \textsc{ignored}$
  (Fig.~\ref{code:functionality:close},
  l.~\ref{code:functionality:close:check}). We deduce that, once $\itistate_P =
  \textsc{ignored}$, the balance check will not happen. Moving to the case where
  $\itistate_P$ has never been \textsc{open}, we observe that it is impossible
  to move to any of the states required by
  l.~\ref{code:functionality:close:check} of Fig.~\ref{code:functionality:close}
  without first having been in the \textsc{open} state. Moreover if $P =
  \alice$, it is impossible to reach the \textsc{open} state without receiving
  input (\textsc{open}, $c$, $\dots$) by \environment. Lastly, as we have
  observed already, the three last conditions of Lemma~\ref{lemma:ideal-balance}
  are always satisfied. We conclude that if the conditions to
  Lemma~\ref{lemma:ideal-balance} do not hold, then the check of
  Fig.~\ref{code:functionality:close},
  l.~\ref{code:functionality:close:check:if} does not happen and therefore
  \fchan does not halt.

  On aggregate, \fchan may only halt with negligible probability in the security
  parameter.
\end{proof}
