\section{Payment Simulation details}
\label{sec:simulation-details}
  Due to the privacy guarantees of LN, we are unable to obtain real-world
  off-chain payment data. We therefore generate payments randomly. More
  specifically, we provide three different payment topologies to mimic
  different usage schemes: First, each party has a preferred receiver, chosen
  uniformly at the beginning, which it pays half the time, the other half
  choosing the payee uniformly at random. Each payment value is chosen
  uniformly at random from the $[0, \max]$ range, for $\max =
  \frac{(\text{initial coins}) \cdot \text{\#players}}{\text{\#payments}}$. We
  employ $1000$ parties, with a knowledge function disclosing to each party its
  next $m=100$ payments, as it appeared this is a realistic knowledge function
  for this case. This scenario occurs when new users are onboarded with
  the intent to primarily pay a single counterparty, but sporadically pay
  others as well. Second, in an attempt to emulate real-world payment
  distributions, the value and number of incoming payments of each player are
  drawn from the zipf~\cite{powers-1998-applications} distibution with
  parameter $2$, which corresponds to real-world power-law distributions with a
  heavy tail~\cite{DBLP:journals/cn/BroderKMRRSTW00}. Each payment value is
  chosen according to the
  zipf$(2.16)$ distribution which corresponds to the $80/20$
  rule~\cite{pareto}, moved to have a mean equal to $\frac{\max}{2}$. We
  consider $500$ parties, and a knowledge function with $m=10$, as this is more
  aligned with real-world scenarios. Third, all choices are made uniformly at
  random, with each payment chosen uniformly from $[0, \max]$, employing a
  total of $3000$ parties, again with each knowing its next $m=10$ payments.
  For all scenarios the payer of each payment is chosen uniformly at
  random, no channels exist initially, and all parties initially own the same
  amount of coins on-chain. A payer funds a new channel with the minimum of all
  the on-chain funds of the payer and the sum of the known future payments to
  the same payee plus $10$ times the current payment value. The
  number of parties is chosen to ensure the simulation completes within a
  reasonable length of time.

  In order to avoid bias, we simulate each
  protocol with the same payments. We simulate each scenario with $20$
  distinct sets of payments and keep the average.
  In Figs.~\ref{graph:delays} and~\ref{graph:fees}, scale does not begin at zero
  for better visibility. Payment
  delays are calculated based on which protocol is used and how the payment is
  performed. Average
  latency is high as it describes the whole run, including slow on-chain
  payments and channel openings. Total fees are calculated by summing the fee
  of each ``basic'' event (e.g., paying an intermediary for its service). None
  of the $3$ protocols provide fee recommendations, so we use the same baseline
  fees for the same events in all $3$ to avoid bias. These
  fees are not systematically chosen, therefore Fig.~\ref{graph:fees} provides
  relative, not absolute, fees.

  As Fig.~\ref{graph:delays} shows, delays are primarily influenced by the
  payment distribution and only secondarily by the protocol: The preferred
  receiver is the fastest and the uniform is the slowest. This is reasonable:
  In the preferred receiver scenario at least half of each party's payments can
  be performed over a single channel, thus on-chain actions are reduced. On the
  other hand, in the uniform scenario payments are spread over all parties
  evenly, so channels are not as well utilised.

