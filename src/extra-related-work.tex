\section{Further Related Work}
\label{sec:further-related-work}
  Solutions alternative to PCNs include side\-chains
  (e.g.,~\cite{BCDF+14,sidechains,KiaZin18}), commit-chains
  (e.g.,~\cite{plasma,10.1007/978-3-031-54776-8_2}) and non-cu\-sto\-di\-al chains
  (e.g.,~\cite{plasma,konstantopoulos2019plasma,plasma-lower-bounds}). These
  approaches offer more efficient payment methods, at the cost of
  requiring a distinguished mediator, additional tust, or on-chain
  checkpointing. Furthermore, they do not enable payments between different instances
  of the same protocol.
  Due to their conceptual and interface differences as
  well as differing levels of software maturity, dedicated user studies need to
  be carried out in order to compare the usability and overall costs of each
  approach under various usage patterns. Rollups~\cite{ZKRollup,Optimism} are
  incompatible with Bitcoin, as they only optimise computation, not storage,
  whereas Bitcoin has by design minimal computation needs.

Various attacks have been identified against LN. The wormhole
  attack~\cite{DBLP:conf/ndss/MalavoltaMSKM19} against LN allows
  colluding parties in a multi-hop payment to steal the fees of the
  intermediaries between them and Flood \& Loot attacks~\cite{10.1145/3419614.3423248}
  analyses an attack in which too many channels are forced to
  close in a short amount of time, harming blockchain liveness and enabling
  a malicious party to steal off-chain funds.

  To the best of our knowledge, no formal treatment of the privacy of LN exists.
  Nevertheless, it intuitively improves upon the privacy of on-chain Bitcoin
  transactions, as LN payments do not leave a permanent record: only
  intermediaries of each payment are informed. It can be argued that Elmo
  further improves privacy, as payments are hidden from
  the intermediaries of a virtual channel.

  Payment routing~\cite{spider,prihodko2016flare,lee2020routee} is another research area that aims to improve network efficiency without sacrificing  privacy. Actively rebalancing channels~\cite{DBLP:conf/ccs/KhalilG17} can
  further increase network efficiency by reducing unavailable routes due to lack of well-balanced funds.

  Bolt~\cite{10.1145/3133956.3134093} constructs privacy-preserving payment
  channels enabling both direct payments and payments with a single untrusted
  intermediary. Sprites~\cite{sprites} leverages the scripting language of
  E\-the\-re\-um to decrease the time collateral is locked compared to LN.

  State channels are a generalisation of payment channels, which enable
  off-chain execution of any smart contract supported by the underlying
  blockchain, not just payments. Generalized Bitcoin-Compatible
  Channels~\cite{DBLP:journals/iacr/AumayrEEFHMMR20} enable the creation of
  state channels on Bitcoin, extending channel functionality from simple
  payments to arbitrary Bitcoin scripts. Since Elmo only pertains to payment,
  not state, channels, we choose not to build it on top
  of~\cite{DBLP:journals/iacr/AumayrEEFHMMR20}. State channels can also be
  extended to more than two
  parties~\cite{DBLP:conf/asiaccs/LiaoZSS22,DBLP:conf/eurocrypt/DziembowskiEFHH19}.

  BDW~\cite{scalable-funding} shows how pairwise channels over Bitcoin can be
  funded with no on-chain transactions by allowing parties to form groups that
  can pool their funds together off-chain and then use those funds to open
  channels. Such proposals are complementary to virtual channels and, depending
  on the use case, could be more efficient. In comparison to Elmo, BDW is less
  flexible: coins in a BDW pool can only be exchanged with members of that
  pool. ACMU~\cite{10.1145/3319535.3345666} allows for multi-path atomic
  payments with reduced collateral, enabling new applications such as
  crowdfunding conditional on reaching a funding target.

  TEE-based~\cite{zhao2019sok}
solutions~\cite{teechan,10.1145/3341301.3359627,DBLP:conf/asiaccs/LiaoZSS22,lee2020routee}
  improve the throughput and efficiency of PCNs by an order of magnitude or
  more, at the cost of having to trust TEEs. Brick~\cite{avarikioti2020brick}
  uses a partially trusted committee to extend PCNs to fully asynchronous
  networks.

  %and partially centralised payment networks that entirely avoid using a
  %blockchain~\cite{DBLP:conf/trust/ArmknechtKMYZ15,stellar,silentwhispers,DBLP:conf/ndss/RoosMKG18}.

  Donner~\cite{donner} is technically insecure since any state update to a base
  channel invalidates the corresponding $\mathsf{tx}^r$. There is a
  straightforward fix, which however adds an overhead to each payment over
  a base channel: On every payment, the two base channel parties must update
  their $\mathsf{tx}^r$ to spend the $\alpha$ output of the new state. Potential
  intermediaries must consider this overhead and possibly increase the fees they
  require from the endpoints. This per-payment overhead can be avoided by using
  \texttt{ANYPREVOUT} in the $\alpha$ output.
