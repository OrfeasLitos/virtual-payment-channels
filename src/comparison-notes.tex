\section{Details on Simulations}
\label{sec:simulation-details}
  The communication rounds for a party is calculated as its [\#incoming messages +
  \#outgoing messages]/2. The size of outgoing messages and the stored data are
  measured in raw bytes. The data is counted as the sum of the relevant channel
  identifiers ($8$ bytes each, as defined by the Lightning Network
  specification\footnote{\url{https://github.com/lightning/bolts/blob/master/07-routing-gossip.md\#definition-of-short_channel_id}}),
  transaction output identifiers ($36$ bytes), secret keys ($32$ bytes each),
  public keys ($32$ bytes each, compressed form -- these double as party
  identifiers), Schnorr signatures ($64$ bytes each), coins ($8$ bytes each),
  times and timelocks (both $4$ bytes each). UC-specific data is ignored.

  For LVPC, multiple different topologies can support a virtual channel between
  $P_1$ and $P_n$ (all of which need $n-1$ base channels). We here consider the
  case in which the funder $P_1$ first opens one virtual channel with $P_3$ on
  top of channels $(P_1, P_2)$ and $(P_2, P_3)$, then another virtual channel
  with $P_4$ over $(P_1, P_3)$ and $(P_3, P_4)$ and so on up to the $(P_1, P_n)$
  channel, opened over $(P_1, P_{n-1})$ and $(P_{n-1}, P_n)$. We choose this
  topology as $P_1$ cannot assume that there exist any virtual channels between
  other parties (which could be used as shortcuts).

  A subtle byproduct of the above topology is that during the opening phase of
  LVPC every intermediary $P_i$ acts both as a fundee in its virtual channel
  with the funder $P_1$ and as an intermediary in the virtual channel of $P_1$
  with the next party $P_{i+1}$. The above does not apply to the first
  intermediary $P_2$, since it already has a channel with $P_1$ before the
  protocol starts. Table~\ref{table:comparison:overhead:n-parties:open} shows
  the total cost of intermediaries $P_3, \dots, P_{n-1}$. The first intermediary
  $P_2$ incurs instead [intermediary's costs - fundee's costs] for all three
  measured quantities.

  For Elmo, the data are derived assuming a virtual channel opens directly on
  top of $n-1$ base channels. In other words the channel considered is opened
  without the help of recursion and only leverages the variadic property of
  Elmo. In Table~\ref{table:comparison:overhead:n-parties:open} the resources
  calculated for Elmo are exact for $n \geq 4$ parties, whereas for $n = 3$ they
  slightly overestimate.

  For the closing comparison, we measure on-chain transactions' size in
  vbytes\footnote{\url{https://en.bitcoin.it/wiki/Weight_units}}, which map
  directly to on-chain fees and thus are preferable to raw bytes. Using vbytes
  also ensures our comparison remains up-to-date irrespective of the network
  congestion and bitcoin-to-fiat currency exchange rate at the time of reading.
  We use the tool found in
  \url{https://jlopp.github.io/bitcoin-transaction-size-calculator/} to aid size
  calculation. For the case of intermediaries, in order to only show
  the costs incurred due to supporting a virtual channel, we subtract the cost
  the intermediary would pay to close its channel if it was not supporting any
  virtual channel.

  The on-chain number of transactions to close a virtual channel in the case of
  LVPC is calculated as follows: One ``split'' transaction is needed for each
  base channel ($n-1$ in total), plus one ``merge'' transaction per virtual
  channel ($n-2$ in total), plus a single ``refund'' transaction for the virtual
  channel, for a total of $2n-2$ transactions.
