Consider a sequence of parties $A_1, \dots, A_n$. We say that $i$ is left of
$i+1$ and $i+1$ is right of $i$. $\forall i \in \{2, \dots, n-1\}$, party $A_i$
has a channel with $A_{i-1}$ of total value $x_{i-1, i}$ and a channel with
$A_{i+1}$ of total value $x_{i, i+1}$. $A_1$ only has a channel with $A_2$ (of
value $x_{1, 2}$), likewise $A_n$ only has a channel with $A_{n-1}$ (of value
$x_{n-1, n}$).

After following a specific protocol that does not involve any new on-chain
transactions, each party holds off-chain a number of transactions and signatures
that imply the existence of a new channel between $A_1$ and $A_n$ with value
$x'$, funded by $A_1$. At a high level, these transactions are as follows:
\begin{itemize}
  \item Each edge party has a transaction that consumes the funding output of
  its only channel and produces two outputs: one for the preexisting channel,
  where the left party has $x'$ coins less and one that carries the $x'$ coins
  for the virtual channel (read: the left party pays for the virtual channel).
  Call the latter ``virtual output''.
  \item Each intermediate party $A_i$ has three types of transactions:
  \begin{itemize}
    \item A ``first-mover'' transaction, which consumes both its channel outputs
    and produces four: one for the left channel where the left party $A_{i-1}$
    has $x'$ less coins, one for the right channel where $A_i$ has $x'$ less
    coins, one that pays $A_i$ directly $x'$ coins and one virtual output with
    $x'$ coins.
    \item Several ``second-mover'' transactions which may be used if exactly one
    of the two adjacent parties has consumed the funding output of the shared
    channel. Wlog, assume that the party to the left has consumed the funding
    output $A_{i-1} A_i$ whereas the party to the right has not consumed $A_i
    A_{i+1}$. $A_i$'s suitable second-mover tx consumes $A_i A_{i+1}$ and
    the virtual output produced by $A_{i-1}$'s transaction. In turn it produces
    one $A_i A_{i+1}$ funding output where $A_i$ has $x'$ less coins, one output
    with $x'$ coins for $A_i$ and a new virtual output with $x'$ coins.
    \item Several third-mover transactions which can be used if both
    adjacent parties have consumed their respective funding output. The suitable
    ``third mover'' tx consumes both virtual outputs from left and right and
    produces a new virtual output with $x'$ coins and an output that pays $A_i$
    directly $x'$ coins.
  \end{itemize}
\end{itemize}

\section*{Q\&A}
\begin{itemize}
  \item \emph{Why are there many second- and third-mover transactions?}
  \item A virtual output produced by a tx of $A_i$ specifies exactly the
  interval of parties around $A_i$ that have already made their move. $A_i$ can
  only spend a virtual output of which the interval ends just before or just
  after $i$ and the single newly produced virtual output has an interval that is
  the union of the intervals of the consumed virtual outputs with $i$ added.
  Therefore $A_i$ has multiple second- and third-mover transactions because each
  one corresponds to different previous interval(s).

  As a result, each intermediate party can only publish exactly one transaction.
  This transaction always generates exactly one new virtual output. If it is a
  first-mover tx, it does not consume a virtual output. If it is a second-mover,
  it consumes one and if it is a third-mover it consumes two. A third-mover tx
  can be published only if the publishing party is surrounded (directly or
  indirectly) by two first-movers, therefore eventually only one virtual output
  will remain, as intended.
  \item \emph{What if a malicious intermediary creates a new virtual output and
  consumes it together with an honest virtual output using its third-mover
  transaction?}
  \item As the third-mover tx has a virtual output with a wider interval, the
  same party cannot repeat the same trick. Since the interval is always
  widening, even if only one edge party is honest, the attack cannot carry for
  ever, therefore eventually the edge party will be able to consume the virtual
  output as intended. Furthermore, \TODO{check if some other party cannot take
  its coins.} Similar reasoning applies to second-mover malicious transactions.
  \item \emph{What if a malicious party publishes an old commitment transaction
  (i.e. consumes a funding output without using any of the first-, second- or
  third-mover txs)?}
  \item Its counterparty $A_i$ won't be able to close honestly its other
  adjacent channel, but it will be able to punish the malicious party with the
  revocation transaction, thus confiscating all its funds. Therefore, to ensure
  no monetary loss is possible, $A_i$ must always enforce that $x_{i-1, i,
  \mathrm{right}} \leq x_{i, i+1, \mathrm{right}}$ and $x_{i, i+1,
  \mathrm{left}} \leq x_{i-1, i, \mathrm{left}}$ (where $x_{i, j,
  \mathrm{left/right}}$ is the coins owned by the left/right party of channel
  $A_i A_j$ respectively). This balance check is performed on every payment and
  new virtual channel.
\end{itemize}
