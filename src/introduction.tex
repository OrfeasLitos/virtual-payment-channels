\section{Introduction}
  The popularity of blockchain protocols 
  in recent years has stretched their
  performance  exposing a number of scalability considerations. 
In particular, 
Bitcoin and related blockchain protocols exhibit very high
  latency (e.g. Bitcoin has a latency of 1h~\cite{bitcoin}) 
  and a very low throughput (e.g., Bitcoin can handle at most 7
  transactions per second~\cite{scaling}), 
both significant shortcomings that jeopardize wider 
use and adoption and are to a certain extend inherent~\cite{scaling}. 
To address these considerations
a prominent approach is to optimistically handle
  transactions off-chain via a ``Payment Channel Network'' 
  (PCN) (see, e.g.,~\cite{DBLP:conf/fc/GudgeonMRMG20} for a survey)
  and only use the underlying blockchain protocol as an
  arbiter in case of dispute.

The key primitive of PCN protocols is a 
payment (or more generally state) channel. 
Two parties initiate the channel by locking
some funds on-chain and subsequently
exchange transactions to update the state of the channel. 
The key feature is that state updates are not posted on-chain
and hence they remain unencumbered by the performance limitations
of the underlying blockchain protocol. 
Given this primitive, multiple overlapping payment
channels can be combined and form the PCN. 

Closing a channel is an operation that involves posting the 
state of the channel on-chain; it is essential that any party
individually can close a channel as otherwise a malicious counterparty
could prevent an honest party  from
accessing their funds. This functionality
however raises an important design consideration: how to prevent malicious parties from posting old states of the channel.
%
Addressing this issue can be done with some suitable use
of  transaction ``time-locks'', a feature that prevents a transaction
or a specific script from being processed on-chain
prior to a specific time. 
%
For instance, diminishing transaction time-locks 
facilitated the Duplex Micropayment Channels (DMC)~\cite{decker}
at the expense of bounding the overall lifetime of a channel. 
Using script time-locks, the Lightning Network (LN)~\cite{lightning}
provided a better solution that enabled channels staying open
for an arbitrary length of time: the key idea was to duplicate the
state of the channel between the two counterparties, say Alice and Bob, and facilitate a punishment mechanism that can be triggered by Bob whenever Alice posts an old state update and vice-versa. The script time-locking is essential to allow an honest counterparty some time to act. 

Interconnecting state channels in LN enables any two parties to transmit funds to each other as long as they can find a route of payment channels that connects them. The downside of this mechanism is that it requires the direct involvement of all the parties along the path for each payment. Instead, ``virtual payment channels'', suggest the more attractive approach of putting a one-time initialization step to setup a virtual payment channel, which subsequently can be used for direct payments with complexity  ---in the optimistic case---  independent of the length of the path. 
Initial constructions for virtual channels essentially capitalized on the extended functionality of Ethereum, e.g., 
Perun~\cite{perun} and GSCN~\cite{DBLP:conf/ccs/DziembowskiFH18}, while more
recent work,~\cite{cryptoeprint:2020:554}  brought them closer to
Bitcoin-compatibility (by leveraging adaptor
signatures~\cite{cryptoeprint:2020:476}).

A virtual channel constructor can be thought of as an  {\em operator} over the underlying primitive of a state channel. We can identify three   natural desiderata for this operator. 

\begin{itemize}
\item Recursive. A recursive virtual channel constructor can operate over channels that themselves could be the results of previous applications of the operator. This is important in the context of PCNs since it allows building virtual channels on top of pre-existing virtual channels.
\item Variadic. A variadic virtual channel constructor can virtualize any number of input state channels. This is important in the context of PCNs since it enables applying the operator to build virtual channels of arbitrary length.
\item Symmetric. A symmetric virtual channel constructor offers operations that are symmetric in terms of cost between the two endpoints or the intermediaries (but not a mix of both) during setup (for the optimistic and pessimistic execution paths). This is important in the context of PCNs since it ensures that no party is worse-off or better-off after an application of the operator in terms of accessing the basic functionality of the channel. 
\end{itemize}

We note that recursiveness, while identified already as an important design property (e.g., see~\cite{DBLP:conf/ccs/DziembowskiFH18}) it has not been achieved in the context of Bitcoin-compatible channels 
(it was achieved only for DCN-like fixed lifetime channels in~\cite{10.1007/978-3-030-65411-5_18} and left as an open question for LN-type channels in~\cite{cryptoeprint:2020:554}). 
The reason behind this are the severe limitations imposed in the design by the scripting language of Bitcoin-compatible systems. 
%
With respect to the other two properties, observe that successive applications of a recursive {\em binary} virtual channel operator to make it variadic will break symmetry (since the sequence of operator applications will impact the participants' functions with respect to the resulting channel). This is of particular concern since all previous virtual channel constructors proposed are binary, cf.~\cite{DBLP:conf/ccs/DziembowskiFH18,cryptoeprint:2020:554,10.1007/978-3-030-65411-5_18}. 

\paragraph{Our Contributions.}  We present  the first  Bitcoin-suitable
recursive virtual channel constructor that is recursive and supports channels
with an indefinite lifetime. In addition, our constructor, Elmo (named after St.
Elmo's fire), is variadic and symmetric. In our constructor, both optimistic and pessimistic execution paths are optimal in terms of round complexity: issuing payments between the two endpoints requires just two messages of size independent of the length of the channel while
closing the channel requires a single transaction for any involved party (endpoint or intermediary) also independent of the channel's length. 

We achieve the above by leveraging on a sophisticated virtual channel setup protocol which, on the one hand, enables endpoints to use an interface that is invariant between base and virtual channels, 
while on the other, intermediaries can act following any arbitrary activation sequence when the channel is closed. The latter is achieved by making it feasible for anyone becoming an initiator towards closing the channel, while subsequent respondents, following the activation sequence, can choose the right action to successfully complete the closure process by posting a single transaction each. 

  We formally prove the security of the constructor protocol in the  UC~\cite{uc} setting. The construction relies on the \texttt{ANYPREVOUT}   signature type, which does not sign the hash of the transaction it spends, therefore allowing for a single pre-signed transaction to spend any output
  with a suitable script. We further discuss the limitations of any constructor
  primitive that does not rely on \texttt{ANYPREVOUT} in
  Section~\ref{section:anyprevout}. In particular in Theorem~\ref{theorem:anyprevout},
  we prove that any virtual channel constructor protocol that
  has participants store transactions in their local state
  and offers an efficient closing operation via $O(1)$ transactions
  will have an exponentially large state in the number of intermediaries, unless
  \texttt{ANYPREVOUT} is available. 
 
\ignore{  %%%%%%% --------- OLD TEXT INTRO -----------
  The most popular PCN is the Lightning Network (LN)~\cite{lightning}, which   works on top of Bitcoin. With LN, parties can open a pairwise channel with a   single on-chain transaction and subsequently pay each other an unlimited   number of times, only limited by the speed of their internet connection. What   is more, a party can pay another even if they do not have a direct channel.   They can instead leverage a path of channels for a fee and perform a so-called   multi-hop payment in an atomic manner. Unfortunately a multi-hop payment needs   active cooperation by all intermediaries, therefore increasing the latency and the probability of failure of the payment.


  To mitigate this issue, virtual payment channels have been proposed
  \TODO{cite}. These enable two parties, say Alice and Bob, to open a payment channel over two preexisting channels, one between Alice and Charlie and   another between Charlie and Bob. \TODO{check if recursive channels exist}

  However, due to the limited scripting language of Bitcoin, it has proved   challenging to build a secure protocol that allows virtual channels to be   opened over more than two underlying channels, \TODO{delete following phrase   if the previous's TODO answer is affirmative} as well as to make this   construction recursive in the sense that further virtual channels can be opened on top of other virtual channels.

  This work fills this gap by providing a concrete protocol that allows for   arbitrarily many channels to be opened on top of arbitrarily long channel   paths, where the underlying channels may themselves be virtual. This is   achieved using standard Bitcoin script and an elaborate transaction
  configuration. 
} %% END OF IGNORE 
  
  
