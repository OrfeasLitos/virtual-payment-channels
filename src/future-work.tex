\section{Discussion and Future work}
\label{sec:future-work}
\makeatletter%
\@ifclassloaded{IEEEtran}%
  {\paragraph{Domino attack}}%
  {\paragraph{Domino attack.}}%
\makeatother%
In~\cite{donner} the Domino attack is presented. Briefly, it claims that a
malicious virtual channel member can force other channels to close. To
illustrate this, suppose $A$ and $E$ of
Fig.~\ref{figure:example-start-end-simple} open
channels
$(A, B)$, $(D, E)$ and $(A, E)$ with the sole intent of forcing channels $(B,
C)$ and $(C, D)$ to close. We observe that, contrary to the attack goal, honest
base parties are only
forced to publish a single virtual transaction each, which places their funding
outputs on-chain but does not cause the base channels to close. There is still a
small downside:
the channel capacity is reduced by the collateral, which
is paid directly to one of the two base channel parties. Since no coins are
stolen, the only cost to $B$, $C$ and $D$ is the on-chain fees
of one transaction. This is an inherent but small risk of
recursive channels, which must be taken into account when making one's
channel the base of another. This risk can be eliminated by making an attacker
pay the fees for the others' on-chain transactions. This fee need not apply in
case of cooperative closing nor
during normal operation and can be reduced for reputable parties.
Suitable reputation systems, as well as mechanisms for assigning inactivity
blame (i.e., proving which parties tried collaborative closing before
closing unilaterally), which are needed to determine who must pay the fees,
can be designed but are beyond the scope of this work. A possible solution to
the recuded channel capacity is discussed in Appx.~\ref{sec:domino-prevention}.

\makeatletter%
\@ifclassloaded{IEEEtran}%
  {\paragraph{Future work}}%
  {\paragraph{Future work.}}%
\makeatother%
Further usability enhancements are possible: Firstly, we can support many
virtual channels in the same virtual layer, reducing on-chain and timelock
demands. Then, the maximum time between activations can be turned from the
currently global constant into a per-channel configurable parameter. Next,
various non-malicious mishaps such as dropped messages can be handled without
causing unilateral channel closure. What is more, LN features like one-off
multi-hop payments and cooperative on-chain closing of simple channels can be
easily incorporated. Lastly, an explicit fee system is needed. Periodic payments
by the endpoints to the intermediares can compensate for the opportunity cost of
the collateral for the latter. For additional future work, see
Appx.~\ref{sec:extra-future-work}.
