\begin{lemma}[No halt]
  In an ideal execution with \fchan and \simulator, the functionality never
  halts (i.e. l.~\ref{code:functionality:state-machine-2:halt} of
  Fig.~\ref{code:functionality:state-machine-2} is never executed).
\end{lemma}

\begin{proof}
  The only way for \fchan to halt is if either $\texttt{check}_A$ or
  $\texttt{check}_B$ fails. For these checks to be run, \fchan must have
  received (\textsc{close}) by \simulator and must have been in the
  \textsc{open} \textit{State} (as any other state of
  Fig.~\ref{code:functionality:state-machine-2},
  l.~\ref{code:functionality:state-machine-2:close:ensure} can only be reached
  after \fchan has been in the \textsc{open} \textit{State}). Additionally,
  \fchan can only reach the \textsc{open} \textit{State} if all honest simulated
  parties transition to the \textsc{open} \textit{State} as well
  (Fig.~\ref{code:simulator}, l.~\ref{code:simulator:open}), which in turn
  happens for simulated \alice only if she has received (\textsc{open}, \dots)
  by \environment. Observe further that \simulator notifies \fchan right away
  when either simulated party becomes corrupted or negligent
  (Fig.~\ref{code:simulator}, ll.~\ref{code:simulator:corrupted}
  and~\ref{code:simulator:negligent} respectively) and the balance of each party
  is checked by \fchan only if this party is not corrupted nor negligent. These
  facts in combination mean that for each party, whenever the prerequisites for
  Lemma~\ref{lemma:ideal-balance} are true the prerequisites for
  Lemma~\ref{lemma:real-balance-security} are also true and therefore the check
  for this party will succeed (c.f.
  Fig~\ref{code:functionality:state-machine-2},
  l.~\ref{code:functionality:state-machine-2:close:coins}). Furthermore, when
  the prerequisites for Lemma~\ref{lemma:ideal-balance} do not hold, the check
  for the respective party will be true. On aggregate, when an ideal execution
  of \fchan and \simulator take place, in no case will \fchan halt.
\end{proof}
