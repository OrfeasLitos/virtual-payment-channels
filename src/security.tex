\section{Security}
  The first step to formally arguing about the security of our scheme is to
  clearly delineate the exact security guarantees it provides. To that end, we
  first prove two similar claims regarding the conservation of funds in the real
  and ideal world, Lemmas~\ref{lemma:real-balance-security}
  and~\ref{lemma:ideal-balance} respectively. Informally, the first claims that
  an honest, non-negligent party which was implicated in an already closed
  channel on which a number of payments took place will have at least the
  expected funds on-chain. The second lemma states that for an ideal party in a
  similar situation, the balance that \fchan has stored for it is at least equal
  to the expected funds. In both cases the expected funds are (initial balance -
  funds for supported virtuals - outbound payments + inbound payments). Note
  that the funds for supported virtuals only refer to those funds used by the
  funder of the virtual channel, not the rest of the base parties.

  Both proofs follow the various possible execution paths, keeping track of the
  resulting balance in each case and coming to the conclusion that balance is
  secure in all cases, except if signatures are forged.

  It is important to note that in fact \pchan provides a stronger guarantee,
  namely that an honest, non-negligent party with an open channel can
  unilaterally close it and obtain the expected funds on-chain within a known
  time frame, given that \environment sends the necessary ``daemon'' messages.
  This stronger guarantee is sufficient to make this construction reliable
  enough for real-world applications. However a corresponding ideal world
  functionality with such guarantees would have to be aware of the specific
  transactions and signatures, therefore it would be essentially as complicated
  as the protocol, thus violating the spirit of the simulation-based security
  paradigm.

  Subsequently we prove Lemma~\ref{lemma:no-halt}, which informally states that
  if an ideal party and all its trusted parties are honest, then \fchan does not
  halt with overwhelming probability. This is proven by first arguing that if
  the conditions of Lemma~\ref{lemma:ideal-balance} for the ideal world hold,
  then the conditions of Lemma~\ref{lemma:real-balance-security} also hold for
  the equivalent real world execution, therefore in this case \fchan does not
  halt. We then argue that also in case the conditions of
  Lemma~\ref{lemma:ideal-balance} do not hold, \fchan may never halt as well,
  therefore concluding the proof.

  We then formulate and prove Theorem~\ref{theorem:security}, which states that
  \pchan UC-realises \fchan. The corresponding proof is a simple application of
  Lemma~\ref{lemma:no-halt}, the fact that \fchan is a simple relay and that
  \simulator faithfully simulates \pchan internally.

  Lastly we construct a ``multi-session
  extension''~\cite{DBLP:conf/crypto/CanettiR03} of \fchan and of \pchan and
  prove Theorem~\ref{theorem:multi-session-security}, which claims that the
  real-world multi-session extension protocol UC-realises the ideal-world
  multi-session extension functionality. The proof is straightforward and
  utilises the transitivity of UC-emulation.

  All formal proofs can be found in the Appendix.
