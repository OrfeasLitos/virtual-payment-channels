\section{Efficiency Evaluation \& Simulations}
  \label{section:comparison}
  We offer here a cost and efficiency comparison of this work with
  LVPC~\cite{10.1007/978-3-030-65411-5_18} and Donner~\cite{donner}. We focus on
  these because they are the only ones that enable
  virtual channels over any number of base channels. We remind that LVPC
  achieves this via its recursive property, while Donner
  because it is variadic (c.f.\ Table~\ref{table:comparison-features}).

  We first count the communication, storage and on-chain cost of a virtual
  channel under each protocol. We then simulate the execution of a large number
  of payments among many parties and derive payment latency and fees. We thus
  obtain an end-to-end understanding of both the requirements and the benefits
  each protocol provides.

  \paragraph{Cost calculation} We consider the general $n$-party case: $1$
  funder ($P_1$), $1$ fundee ($P_n$) and $n-2$ intermediaries ($P_2, \dots,
  P_{n-1}$) where each party has one base channel with each adjacent party. Our
  comparison regards the off-chain cost of opening
  (Table~\ref{table:comparison:overhead:n-parties:open}) and the on-chain cost
  of unilaterally closing
  (Table~\ref{table:comparison:overhead:n-parties:close}).

  Regarding channel opening, in
  Table~\ref{table:comparison:overhead:n-parties:open} we measure for each of
  the three protocols the number of communication rounds required, the total
  size of the outgoing messages as well as the amount of space for storing
  channel data. We measure from the perspective of the funder, the fundee
  and an intermediary, along with the aggregate for all parties. The
  communication rounds for a party is calculated as its [\#incoming messages +
  \#outgoing messages]/2. The size of outgoing messages and the stored data are
  measured in raw bytes. The data is counted as the sum of the relevant channel
  identifiers ($8$ bytes each, as defined by the Lightning Network
  specification\footnote{\url{https://github.com/lightning/bolts/blob/master/07-routing-gossip.md\#definition-of-short_channel_id}}),
  transaction output identifiers ($36$ bytes), secret keys ($32$ bytes each),
  public keys ($32$ bytes each, compressed form -- these double as party
  identifiers), Schnorr signatures ($64$ bytes each), coins ($8$ bytes each),
  times and timelocks (both $4$ bytes each). UC-specific data is ignored.

  For LVPC, multiple different topologies can support a virtual channel between
  $P_1$ and $P_n$ (all of which need $n-1$ base channels). We here consider the
  case in which the funder $P_1$ first opens one virtual channel with $P_3$ on
  top of channels $(P_1, P_2)$ and $(P_2, P_3)$, then another virtual channel
  with $P_4$ over $(P_1, P_3)$ and $(P_3, P_4)$ and so on up to the $(P_1, P_n)$
  channel, opened over $(P_1, P_{n-1})$ and $(P_{n-1}, P_n)$. We choose this
  topology as $P_1$ cannot assume that there exist any virtual channels between
  other parties (which could be used as shortcuts).

  A subtle byproduct of the above topology is that during the opening phase of
  LVPC every intermediary $P_i$ acts both as a fundee in its virtual channel
  with the funder $P_1$ and as an intermediary in the virtual channel of $P_1$
  with the next party $P_{i+1}$. The above does not apply to the first
  intermediary $P_2$, since it already has a channel with $P_1$ before the
  protocol starts. Table~\ref{table:comparison:overhead:n-parties:open} shows
  the total cost of intermediaries $P_3, \dots, P_{n-1}$. The first intermediary
  $P_2$ incurs instead [intermediary's costs - fundee's costs] for all three
  measured quantities.

  For Elmo, the data are derived assuming a virtual channel opens directly on
  top of $n-1$ base channels. In other words the channel considered is opened
  without the help of recursion and only leverages the variadic property of
  Elmo. In Table~\ref{table:comparison:overhead:n-parties:open} the resources
  calculated for Elmo are exact for $n \geq 4$ parties, whereas for $n = 3$ they
  slightly overestimate.

  \addtolength{\intextsep}{-15pt}
  \begin{table}[h!]
    \resizebox{\textwidth}{!}{%
    \begin{tabular}{|l|c|c|c|c|c|c|c|c|c|c|c|}
    \hline
    \multicolumn{12}{|c|}{Open} \\
    \hline
    \multirow{3}{*}{}
              & \multicolumn{3}{|c|}{Funder} & \multicolumn{3}{|c|}{Fundee}
              & \multicolumn{3}{|c|}{Intermediary}
              & \multicolumn{2}{|c|}{Total} \\
    \cline{2-12}
              & \multirow{2}{*}{\shortstack{party \\ rounds}}
              & \multicolumn{2}{|c|}{size} & \multirow{2}{*}{\shortstack{party
              \\ rounds}} & \multicolumn{2}{|c|}{size}
              & \multirow{2}{*}{\shortstack{party \\ rounds}}
              & \multicolumn{2}{|c|}{size} & \multicolumn{2}{|c|}{size} \\
    \cline{3-4} \cline{6-7} \cline{9-12}
              & & sent & stored & & sent & stored & & sent & stored & sent &
              stored \\
    \hline
    LVPC      & $8(n-2)$ & $1381(n-2)$ & $3005(n-2)$ & $7$ & $1254$ & $2936$
              & $16$ & $2989$ & $6385$ & $4370n-8740$ & $9390n-18780$ \\
    \hline
    % my count
    %Donner     & $2$ & $164n + 1934$ & $108n + 2150$ & $1$ & $44n+128$
    %           & $176n+496$ & $1$ & $76n + 2010$ & $132n+2370$
    %           & $132n^2+2390n-2094$ & $76n^2+2066n-1858$ \\
    %\hline
    \multirow{2}{*}{Donner}
              & \multirow{2}{*}{$2$} & \multirow{2}{*}{$184n + 829$}
              & \multirow{2}{*}{\shortstack{$1332.5k+$ \\ $43n+125.5$}}
              & \multirow{2}{*}{$1$} & \multirow{2}{*}{$43n+192.5$}
              & \multirow{2}{*}{\shortstack{$1332.5k+$ \\ $43n+125.5$}}
              & \multirow{2}{*}{$1$} & \multirow{2}{*}{$547$}
              & \multirow{2}{*}{\shortstack{$1332.5k+$ \\ $43n+125.5$}}
              & \multirow{2}{*}{$774n-71$}
              & \multirow{2}{*}{\shortstack{$1332.5kn +$ \\ $43n^2 + 125.5n$}}
              \\
              & & & & & & & & & & & \\
              % For Donner, I drew the storage numbers from
              % https://eprint.iacr.org/2021/855.pdf, p. 22. I'm not sure what
              % pid is, so these numbers may have to be revised.
    \hline
    \multirow{3}{*}{Elmo}
              & \multirow{3}{*}{$6$} &
              \multirow{3}{*}{\shortstack{$32n^3-128n^2$ \\
              $+544n-276$}} &
              \multirow{3}{*}{\shortstack{$\frac{128}{3}n^3-128n^2$ \\
              $+\frac{1276}{3}n+220$}} &
              \multirow{3}{*}{$6$}
              & \multirow{3}{*}{\shortstack{$32n^3-128n^2$ \\
              $+544n-340$}} &
              \multirow{3}{*}{\shortstack{$\frac{128}{3}n^3-128n^2$ \\
              $+\frac{1276}{3}n+220$}} &
              \multirow{3}{*}{$12$}
              & \multirow{3}{*}{\shortstack{$96n^3-256n^2$ \\
              $+404n-40$}}
              & \multirow{3}{*}{\shortstack{$96n^3-256n^2$ \\
              $+468n+88$}}
              & \multirow{3}{*}{\shortstack{$96n^4-384n^3+$ \\
              $724n^2+240n-792$}} &
              \multirow{3}{*}{\shortstack{$96n^4-\frac{1088}{3}n^3+$ \\
              $660n^2+\frac{8}{3}n+520$}}\\
              & & & & & & & & & & & \\
              & & & & & & & & & & & \\
    \hline
    \end{tabular}}
    \caption{Open efficiency comparison of virtual channel protocols with $n$
    parties and $k$ payments}
    \label{table:comparison:overhead:n-parties:open}
  \end{table}
  \addtolength{\intextsep}{15pt}

  Regarding closing, in Table~\ref{table:comparison:overhead:n-parties:close} we
  measure for each of the three protocols the worst-case on-chain cost a party
  would need to incur in order to unilaterally close its channel. The cost is
  measured both in the number of transactions and in their total size. We
  measure size in
  vbytes\footnote{\url{https://en.bitcoin.it/wiki/Weight_units}}, which map
  directly to on-chain fees and thus are preferable to raw bytes. Using vbytes
  also ensures our comparison remains up-to-date irrespective of the network
  congestion and bitcoin-to-fiat currency exchange rate at the time of reading.
  We use the tool found in
  \url{https://jlopp.github.io/bitcoin-transaction-size-calculator/} to aid size
  calculation.

  For the two endpoints (funder and fundee), the cost of unilaterally closing
  the virtual channel is reported. On the other hand, for each intermediary we
  report the cost of closing a base channel. Note that, in order to only show
  the costs incurred due to supporting a virtual channel, we subtract the cost
  the intermediary would pay to close its channel if it was not supporting any
  virtual channel. We also present the worst-case total on-chain cost,
  aggregated over all parties. Note that the latter cost is not simply the sum
  of the worst-case costs of all parties, as one party's worst case is not
  necessarily the worst case of another. This cost rather represents the maximum
  possible load an instantiation of each protocol could add to the blockchain
  when closing.

  The on-chain number of transactions to close a virtual channel in the case of
  LVPC is calculated as follows: One ``split'' transaction is needed for each
  base channel ($n-1$ in total), plus one ``merge'' transaction per virtual
  channel ($n-2$ in total), plus a single ``refund'' transaction for the virtual
  channel, for a total of $2n-2$ transactions.

  \addtolength{\intextsep}{-15pt}
  \begin{table}[h!]
    \begin{minipage}{\textwidth}
    \centering
    \begin{tabular}{|l|c|c|c|c|c|c|c|c|}
    \hline
    \multicolumn{9}{|c|}{Unilateral Close} \\
    \hline
              & \multicolumn{2}{|c|}{Intermediary}
              & \multicolumn{2}{|c|}{Funder} & \multicolumn{2}{|c|}{Fundee}
              & \multicolumn{2}{|c|}{Total} \\
    \hline
              & \#txs & size & \#txs & size & \#txs & size & \#txs & size \\
    \hline
    LVPC      & $3$ & $627$ & $2$ & $383$ & $2$ & $359$ & $2n-2$ & $435n -
              510.5$ \\
    \hline
    Donner    & $1$ & $204.5$ & $4$ & $704 + 43n$ & $1$ & $136.5$ & $2n$ & $458n
              - 26$ \\
    \hline
    Elmo      & $1$ & $297.5$ & $3$ & $376$ & $3$ & $376$
              & $n+1$ & $254.5n-133$ \\
    \hline
    \end{tabular}
    \end{minipage}
    \caption{On-chain worst-case closing efficiency comparison of virtual
    channel protocols with $n$ parties}
    \label{table:comparison:overhead:n-parties:close}
  \end{table}
  \addtolength{\intextsep}{15pt}

  We note that Elmo takes advantage of Schorr signatures to reduce both its
  on-chain and storage footprint. In particular, the $n$ signatures that are
  needed to spend each virtual and bridge output can be reduced to a single
  aggregate signature without compromising security. The same cannot be said for
  Donner, since Schnorr signatures cannot optimise away the $n$ outputs of the
  funder's transaction $\tx^{\mathtt{vc}}$. Likewise LVPC cannot gain a linear
  improvement with this optimisation, since each of its relevant transactions
  (``split'', ``merge'' and ``refund'') needs constant signatures.
