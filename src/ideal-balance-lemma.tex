\begin{lemma}[Ideal world balance]
\label{lemma:ideal-balance}
  Consider an ideal world execution with functionality \fchan and simulator
  \simulator. Let $P \in \{\alice, \bob\}$ one of the two parties of \fchan.
  Assume that all of the following are true:
  \begin{itemize}
    \item $\itistate_P \neq \textsc{ignored}$,
    \item $P$ has transitioned to the \textsc{open} \textit{State} at least
    once. Additionally, if $P = \alice$, it has received $(\textsc{open}, c,
    \dots)$ by \environment prior to transitioning to the \textsc{open}
    \textit{State},
    \item $P$ [has received $(\textsc{fund me}, f_i, \dots)$ as input by another
    \fchan/\textsc{ln} ITI while $\itistate_P = \textsc{open}$ and $P$
    subsequently transitioned to \textsc{open} \textit{State}] $n \geq 0$ times,
    \item $P$ [has received $(\textsc{pay}, d_i)$ by \environment while
    $\itistate_P = \textsc{open}$ and $P$ subsequently transitioned to
    \textsc{open} \textit{State}] $m \geq 0$ times,
    \item $P$ [has received $(\textsc{get paid}, e_i)$ by \environment while
    $\itistate_P = \textsc{open}$ and $P$ subsequently transitioned to
    \textsc{open} \textit{State}] $l \geq 0$ times.
  \end{itemize}
  Let $\phi = 1$ if $P = \alice$, or $\phi = 0$ if $P = \bob$. If \fchan
  receives $(\textsc{close}, P)$ by \simulator, then the following holds with
  overwhelming probability on the security parameter:
  \begin{equation}
    \texttt{balance}_P = \phi \cdot c - \sum\limits_{i=1}^n f_i -
    \sum\limits_{i=1}^m d_i + \sum\limits_{i=1}^l e_i
  \end{equation}
\end{lemma}

\begin{proof}[Proof of Lemma~\ref{lemma:ideal-balance}]
  We will prove the Lemma by following the evolution of the $\subbalance{P}$
  variable.
  \begin{itemize}
    \item When \fchan is activated for the first time, it sets $\subbalance{P}
    \gets 0$ (Fig.~\ref{code:functionality:open},
    l.~\ref{code:functionality:open:boot}). \item If $P = \alice$ and it
    receives (\textsc{open}, $c$, $\dots$) by \environment, it stores $c$
    (Fig.~\ref{code:functionality:open},
    l.~\ref{code:functionality:open:store}). If later $\itistate_P$ becomes
    \textsc{open}, \fchan sets $\subbalance{P} \gets c$
    (Fig.~\ref{code:functionality:open},
    ll.~\ref{code:functionality:open:base-balance}
    or~\ref{code:functionality:open:virtual-balance}). In contrast, if $P =
    \bob$, it is $\subbalance{P} = 0$ until at least the first transition of
    $\itistate_P$ to \textsc{open} (Fig.~\ref{code:functionality:open}).
    \item Every time $P$ receives input $(\textsc{fund me}, f_i, \dots)$ by
    another party while $\itistate_P = \textsc{open}$, $P$ stores $f_i$
    (Fig.~\ref{code:functionality:fund},
    l.~\ref{code:functionality:fund:when-fund}). The next time $\itistate_P$
    transitions to \textsc{open} (if such a transition happens), $\subbalance{P}$
    is decremented by $f_i$ (Fig.~\ref{code:functionality:fund},
    l.~\ref{code:functionality:fund:complete-fund}). Therefore, if this cycle
    happens $n \geq 0$ times, $\subbalance{P}$ will be decremented by
    $\sum\limits_{i=1}^n f_i$ in total.
    \item Every time $P$ receives input $(\textsc{pay}, d_i)$ by \environment
    while $\itistate_P = \textsc{open}$, $d_i$ is stored
    (Fig.~\ref{code:functionality:pay},
    l.~\ref{code:functionality:pay:store-pay}). The next time
    $\itistate_P$ transitions to \textsc{open} (if such a transition happens),
    $\subbalance{P}$ is decremented by $d_i$
    (Fig.~\ref{code:functionality:pay},
    l.~\ref{code:functionality:pay:decrement}). Therefore, if this cycle
    happens $m \geq 0$ times, $\subbalance{P}$ will be decremented by
    $\sum\limits_{i=1}^m d_i$ in total.
    \item Every time $P$ receives input $(\textsc{get paid}, e_i)$ by
    \environment while $\itistate_P = \textsc{open}$, $e_i$ is stored
    (Fig.~\ref{code:functionality:pay},
    l.~\ref{code:functionality:pay:store-get-paid}). The next time
    $\itistate_P$ transitions to \textsc{open} (if such a transition happens)
    $\subbalance{P}$ is incremented by $e_i$
    (Fig.~\ref{code:functionality:pay},
    l.~\ref{code:functionality:pay:increment}). Therefore, if this cycle
    happens $l \geq 0$ times, $\subbalance{P}$ will be incremented by
    $\sum\limits_{i=1}^l e_i$ in total.
  \end{itemize}
  On aggregate, after the above are completed and then \fchan receives
  (\textsc{close}, $P$) by \simulator, it is $\subbalance{P} = c -
  \sum\limits_{i=1}^n f_i - \sum\limits_{i=1}^m d_i + \sum\limits_{i=1}^l e_i$
  if $P = \alice$, or else if $P = \bob$, $\subbalance{P} = - \sum\limits_{i=1}^n
  f_i - \sum\limits_{i=1}^m d_i + \sum\limits_{i=1}^l e_i$.
\end{proof}
