\section{Protocol Description}
  Conceptually, Elmo is split into four main actions: channel opening,
  payments, cooperative closing and unilateral closing. A channel $(P_1, P_n)$
  between parties $P_1$ and $P_n$
  may be opened directly on-chain, in which case the two parties follow an
  opening procedure similar to that of LN; such a channel is called ``simple''.
  Otherwise it can be opened on top of a path
  of preexisting ``base'' channels $(P_1, P_2)$, $(P_2, P_3)$, $\dots$, $(P_{n-2},
  P_{n-1})$, $(P_{n-1}, P_{n})$, in which case $(P_1, P_n)$ is a ``virtual''
  channel (each base channel may itself be simple or virtual). For a virtual
  channel, all parties $P_i$ on the
  path follow our novel protocol, setting aside funds in their channels as
  collateral for the new virtual channel that is being opened; this is done
  by creating so-called ``virtual'' transactions that essentially
  tie the spending of two adjacent base channels into a single atomic action.
  Once all intermediaries are done, a special ``funding'' output has been
  created off-chain which carries the sum of $P_1$ and $P_n$'s channel
  balance and which either one can at any time put on-chain. LN demands that the
  funding output is on-chain, but we lift this requirement. $P_1$ and $P_n$
  finally create their channel, applying a logic similar to LN on top of
  the funding output: their channel is now open.

  A payment over an established channel follows a procedure heavily inspired by
  LN as well. To be completed, a payment needs three messages to be exchanged by
  the two parties.

  A virtual channel can be optimistically closed completely off-chain. At a high
  level, the parties that control the base channels ``revoke'' their virtual
  transactions and the related commitment transactions. Revoked transactions
  cannot be used anymore. This effectively peels one layer of virtualisation.
  Balances are redistributed so that intermediaries ``break even'', while $P_1$
  and $P_n$ each gets its rightful coins (as reflected in the last state of the
  virtual channel) in its base channel ($(P_1, P_2)$ and $(P_{n-1}, P_n)$
  respectively).

  Finally, the unilateral closing procedure of a channel $C$ does not need
  cooperation and consists of signing and publishing a number of transactions
  on-chain. As
  we will discuss later, the exact transactions that a party will publish vary
  depending on the actions of the parties that control the channels which form
  the base of $C$, as well as the channels that are based on $C$.

  In a nutshell, a virtual channel is built on top of two or more ``base''
  channels, which, due to the recursive property, may themselves be simple or
  virtual. The parties that control the base channels are called ``base
  parties''. The fact that more than two base channels can be used by a single
  virtual channel is ensured by the variadic property.

  In more detail, to open a channel (c.f.\ Figure~\ref{code:ln:open}) the two
  counterparties (a.k.a.\ ``endpoints'') (i) create new keypairs and exchange
  the resulting public keys ($2$ messages), then (ii) prepare the underlying
  base channels if the new channel is virtual ($12 \cdot (n-1)$ total messages,
  i.e.\ $6$ outgoing messages per endpoint and $12$ outgoing messages per
  intermediary, for $n-2$ intermediaries), next (iii) they exchange signatures
  for their respective initial commitment transactions ($2$ messages) and
  lastly, (iv) if the channel is to be opened directly on-chain (i.e.\ is
  simple), the ``funder'' signs and publishes the ``funding'' transaction to the
  ledger. As we alluded to earlier, a channel with its funding transaction
  on-chain is called ``simple''. A channel is either simple or virtual, not
  both. We here note that like LN, only one of the two parties, the funder,
  provides coins for a new channel. This limitation simplifies the execution
  model and the analysis, but can be lifted at the cost of additional protocol
  complexity.

  Let us now introduce some notation used in figures with transactions.
  Reflecting the UTXO model, each transaction is represented by a circular,
  named node with one incoming edge per input and one outgoing edge per output.
  Each output can be connected with at most one input of another transaction;
  cycles are not allowed. Above an input or an output edge we note the number of
  coins it carries. In some figures the coins are omitted. Below an input we
  place the data carried and below an output its spending conditions (a.k.a\
  script). For
  a connected input-output pair, we omit the data carried by the input.
  $\sigma_K$ is a signature on the transaction by $\sk{K}$; in all cases, signatures
  are carried by inputs. An output marked
  with $\pk{K}$ needs a signature by $\sk{K}$ to be spent. $m/\{\pk{1}, \dots,
  \pk{n}\}$ is an $m$-of-$n$ multisig ($m \leq n$) that needs signatures from
  $m$ distinct keys among $\sk{1}, \dots, \sk{n}$. If $k$ is a spending
  condition, then $k + t$ is the same spending condition but with a relative timelock
  of $t$. Spending conditions or data can be combined with logical ``AND''
  ($\wedge$) and ``OR'' ($\vee$), so an output $a \vee b$ can be spent either by
  matching the condition $a$ or the condition $b$, and an input $\sigma_a \wedge
  \sigma_b$ carries signatures from $\sk{a}$ and $\sk{b}$.
  Note that all signatures for all multisig outputs make use of the
  \texttt{ANYPREVOUT} hash type.

\subsection{Simple Channels}
  In a similar vein to earlier UTXO-based PCN proposals, having an open channel essentially
  means having very specific keys, transactions and signatures at hand, as well
  as checking the ledger periodically and being ready to take action if
  misbehaviour is detected. Let us first consider a simple channel that has been
  established between \alice and \bob where the former owns $c_A$ and the latter
  $c_B$ coins. There are three sets of transactions at play: A ``funding''
  transaction that is put on-chain, ``commitment'' transactions that are stored
  off-chain and
  spend the funding output on channel closure and off-chain ``revocation''
  transactions that spend commitment outputs in case of misbehaviour (c.f.\
  Figure~\ref{figure:payment-layer}).

  \begin{figure}
    \centering
    \subimport{./figures/manual-tikz/}{payment-layer.tex}
    \caption{Funding, commitment and revocation transactions}
    \label{figure:payment-layer}
  \end{figure}

  In particular, there is a single on-chain funding transaction that spends $c_A
  + c_B$ coins (originally belonging to the funder), with a single output that is encumbered with a
  $2/\{\pk{A, F}, \pk{B, F}\}$ multisig and carries $c_A + c_B$ coins.

  Next, there are two commitment transactions, one for each party, each of which can spend the
  funding tx and produce two outputs with $c_A$ and $c_B$ coins each. The two
  txs differ in the outputs' spending conditions: The $c_A$ output in \alice's
  commitment tx can be spent either by \alice after it has been on-chain for a
  pre-agreed period (i.e.\ it is encumbered with a ``timelock''), or by a
  ``revocation'' transaction (discussed below) via a $2$-of-$2$ multisig between
  the counterparties, whereas the $c_B$ output can be spent only by \bob without
  a timelock. \bob's commitment tx is symmetric: the $c_A$ output can be spent
  only by \alice without timelock and the $c_B$ output can be spent either by
  \bob after the timelock expiration or by a revocation tx. When a new pair of
  commitment txs are created (either during channel opening or on each update)
  \alice signs \bob's commitment tx and sends him the signature (and
  vice-versa), therefore \alice can later unilaterally sign and publish her commitment
  tx but not \bob's (and vice-versa).

  Last, there are $2m$ revocation transactions, where $m$ is the total number of
  updates of the channel. The $j$th revocation tx held by an endpoint spends the
  output carrying the counterparty's funds in the counterparty's $j$th
  commitment tx. It has a single output spendable immediately by the
  aforementioned endpoint. Each endpoint stores $m$ revocation txs, one for each
  superseded commitment tx. This creates a disincentive for an endpoint to cheat
  by using any other commitment transaction than its most recent one to close
  the channel: the timelock on the commitment output permits its counterparty to
  use the corresponding revocation transaction and thus claim the cheater's
  funds.  Endpoints do not have a revocation tx for the last commitment
  transaction, therefore these can be safely published. For a channel update to
  be completed, the endpoints must exchange the signatures for the revocation
  txs that spend the commitment txs that just became obsolete.

  Observe that the above logic is essentially a simplification of LN. In
  particular, Elmo does not use Hashed TimeLocked Contracts (HTLCs), which
  enable multi-hop payments in LN.

\subsection{Virtual Channels}
  In order to gain intuition on how virtual channels work, consider $n-1$
  simple channels between $n$ honest parties as before. $P_1$, the
  funder, and $P_n$, the fundee, want to open a virtual channel over these base
  channels.
  Before opening the virtual, each base channel is entirely independent, having
  different unique keys, separate on-chain funding outputs, a possibly different
  balance and number of updates. After the $n$ parties follow our novel virtual
  channel opening protocol, they will all hold off-chain a number of new,
  ``virtual'' transactions that spend their respective funding transactions. The
  ``virtual'' transactions can be spent by ``bridge'' transactions which in turn
  are spendable by new commitment transactions in a manner that ensures fair
  funds allocation for all honest parties. ``Bridge'' transactions take
  advantage of \texttt{ANYPREVOUT} to ensure that each of $P_1, P_n$ only needs
  to maintain a single commitment transaction.

  In particular, apart from the transactions of simple channels (i.e.\
  commitment and revocation txs), each of the two
  endpoints also has an ``initiator'' transaction that spends the funding output
  of its only base channel and produces two outputs: one new funding output for
  the base channel and one ``virtual'' output (c.f.\
  Figures~\ref{figure:virtual-layer-endpoint},~\ref{code:virtual-layer:endpoint-txs}).
  If the initiator transaction ends up on-chain honestly, the latter output
  carries coins that will directly or indirectly fund the funding output of the
  virtual channel. This virtual funding output can in turn be spent by a
  commitment transaction that functions exactly in the same manner as in a
  simple channel.

  \begin{figure}
    \subimport{./figures/manual-tikz/}{virtual-layer-endpoint.tex}
    \caption{$A-E$ virtual channel: $A$'s initiator transaction. Spends the
    funding output of the $A-B$ channel. Can be used if $B$ has not published
    a virtual transaction yet.}
    \label{figure:virtual-layer-endpoint}
  \end{figure}

  Intermediaries on the other hand store three sets of virtual transactions
  (Figure~\ref{code:virtual-layer:mid-txs}): ``initiator''
  (Figure~\ref{figure:virtual-layer-initiator}), ``extend-interval''
  (Figure~\ref{figure:virtual-layer-extend-interval}) and ``merge-intervals''
  (Figure~\ref{figure:virtual-layer-merge-intervals}). Each intermediary has one
  initiator tx, which spends the party's two funding outputs and produces four:
  one funding output for each base channel, one output that directly pays the
  intermediary coins equal to the total value in the virtual channel, and one
  ``virtual output'', which carries coins that can potentially fund the virtual
  channel. If both funding outputs are still unspent, publishing its initiator
  tx is the only way for an honest intermediary to close either of its channels
  and retrieve its collateral.

  \begin{figure}
    \subimport{./figures/manual-tikz/}{intermediary-initiator.tex}
    \caption{$A-E$ virtual channel: $B$'s initiator transaction. Spends the
    funding outputs of the $A-B$ and $B-C$ channels. Can be used if neither
    $A$ nor $C$ have published a virtual transaction yet.}
    \label{figure:virtual-layer-initiator}
  \end{figure}

  \begin{figure}
    \subimport{./figures/manual-tikz/}{intermediary-extend-interval.tex}
    \caption{$A-E$ virtual channel: One of $B$'s extend interval
    transactions. $\sigma$ is the signature. Spends the virtual output of $A$'s
    initiator transaction and the funding output of the $B$-$C$ channel. Can be
    used if $A$ has already published its initiator transaction and $C$ has not
    published a virtual transaction yet.}
    \label{figure:virtual-layer-extend-interval}
  \end{figure}

  \begin{figure}
    \subimport{./figures/manual-tikz/}{intermediary-merge-intervals.tex}
    \caption{$A$ - $E$ virtual channel: One of $B$'s merge intervals
    transactions. Spends the virtual outputs of $A$'s and $C$'s virtual
    transactions. Can be used if both $A$ and $C$ have already published their
    initiator transactions. Notice that the interval of $C$'s virtual output
    only contains $C$, which can only happen if $C$ has published its initiator
    and not any other of its virtual transactions.}
    \label{figure:virtual-layer-merge-intervals}
  \end{figure}

  Furthermore, each intermediary has $O(n)$ extend-interval transactions.
  Being an intermediary, the party is involved in two base channels, each having
  its own funding output. In case exactly one of these two funding outputs has
  been spent honestly and the other is still unspent,
  publishing an extend-interval transaction is the only way for the party to
  close the base channel corresponding to the unspent output and retrieve its
  collateral.
  Such a transaction consumes two outputs: the only
  available funding output and a suitable virtual output, as discussed below. An
  extend-interval tx has three outputs: A funding output replacing the one just
  spent, one output that directly pays the intermediary coins equal to the total
  value of the virtual channel, and one virtual output.

  Last, each intermediary has $O(n^2)$ merge-intervals transactions. If both
  base channels' funding outputs of the party have been spent honestly,
  publishing a merge-intervals
  transaction is the only way for the party to retrieve its collateral. Such a
  transaction consumes two suitable virtual outputs, as discussed below. It has
  two outputs: One that directly pays the intermediary coins equal to the total
  value of the virtual channel, and one virtual output.

  Note that each output of a virtual transaction has a ``revocation'' spending
  method which needs a signature from every party that could end up owning the
  output coins: each funding output is signed by the two parties of the
  corresponding channel, each refund output is signed by the transaction owner
  and the party to the left (giving $c_{\mathrm{virt}}$ coins to the left party
  if the owner acts maliciously), whereas each virtual output is signed by the
  transaction owner, the right party and the two virtual channel parties. If the
  owner acts maliciously, $c_{\mathrm{virt}}$ are given to the right party. The
  virtual channel parties have to sign as well since this output may end up
  funding their channel -- lack of such signatures would allow two colluding
  intermediaries to claim the virtual output for themselves.

  Each virtual transaction is accompanied by a ``bridge'' transaction. Any
  virtual output may end up funding the virtual channel, but the various virtual
  outputs do not have the same script, thus there cannot be a single commitment
  transaction able to spend all of them. Without the bridge transaction, the
  parties of the virtual channel would have to keep track of $O(n^3)$ commitment
  transactions to be able to close their channel securely in every case, making
  channel updates expensive. This is fixed by the bridge transactions, which all
  have exactly the same output, unifying the interface between the
  virtualisation and the payment transactions and thus making virtual channel
  updates as cheap as simple channel updates.

  To understand why this multitude of virtual transactions is needed, we now
  zoom out from the individual party and discuss the dynamic of unilateral
  closing as a
  whole. The first party $P_i$ that wishes to close a base channel observes that
  its funding output(s) remain(s) unspent and publishes its initiator
  transaction. First, this allows $P_i$ to use its commitment transaction to
  close the base channel. Second, in case $P_i$ is an intermediary, it directly
  regains the coins it has locked for the virtual channel as collateral. Third,
  it produces a
  virtual output that can only be consumed by $P_{i-1}$ and $P_{i+1}$, the
  parties adjacent to $P_i$ (if any) with specific extend-interval transactions.
  The virtual output of this extend-interval transaction can in turn be spent by
  specific extend-interval transactions of $P_{i-2}$ or $P_{i+2}$ that have not
  published a virtual transaction yet (if any) and so on for the next
  neighbours. The
  idea is that each party only needs to publish a single virtual transaction to
  ``collapse'' the virtual layer and each virtual output uniquely defines the
  continuous interval of parties that have already published a virtual
  transaction and only allows parties at the edges of this interval to extend it.
  This extension rule prevents malicious parties from indefinitely replacing a
  virtual output with a new one. As the name suggests, merge-intervals
  transactions are published by parties that are adjacent to two parties that
  have already published their virtual transactions and in effect joins the two
  intervals into one.

  Each virtual output can also be used to fund the virtual
  channel after a timelock, to protect from unresponsive parties blocking the
  virtual channel indefinitely. This in turn means that if an intermediary
  observes either of its funding outputs being spent, it has to publish its
  suitable virtual transaction before the timelock expires to avoid losing
  funds. What is more, all virtual outputs need the signature of all parties to
  be spent before the timelock (i.e.\ they have an $n$-of-$n$ multisig) in order
  to prevent colluding parties from faking the intervals progression. Thanks to
  Schnorr signatures however, the on-chain footprint of the $n$ signatures is
  reduced to that of a single signature. To ensure
  that parties have an opportunity to react, the timelock of a virtual output is
  the maximum of the required timelocks of the intermediaries that can spend it.
  Let $p$ be a global constant representing the maximum number of blocks a party
  is allowed to stay offline between activations without becoming negligent
  (the latter term is explained in detail later), and
  $s$ the maximum number of blocks needed for an honest transaction to enter the
  blockchain after being published, as in Proposition~\ref{prop:liveness} of
  Section~\ref{sec:liveness}.
  The required timelock of a party is $p + s$ if its channel is simple,  or $p +
  \sum\limits_{j = 2}^{n - 1}(s - 1 + t_j)$ if the channel is virtual, where
  $t_j$ is the required timelock of the base channel
  of the $j$th intermediary's channel. The only exception are virtual outputs
  with an interval that includes all parties, which are just
  funding outputs for the virtual channel: an interval with all parties cannot
  be further extended, therefore one spending method and the timelock are
  dropped.

  We here note that a typical extend-interval and merge-intervals transaction
  has to be able to
  spend different outputs, depending on the order other base parties publish
  their virtual transactions. For example, $P_3$'s extend-interval tx that
  extends the interval $\{P_1, P_2\}$ to $\{P_1, P_2, P_3\}$ must be able to
  spend both the virtual output of $P_2$'s initiator transaction and $P_2$'s
  extend-interval transaction which has spent $P_1$'s initiator transaction.
  In order for the received signatures for virtual and commitment txs to be
  valid for multiple previous outputs, the previously proposed
  \texttt{ANYPREVOUT} sighash flag~\cite{anyprevout} is needed to be added to
  Bitcoin. We conjecture that variadic recursive virtual channels with $O(1)$
  on-chain and subexponential number of off-chain transactions of size $O(1)$
  for each party cannot be constructed in Bitcoin without this flag.\footnote{To
  see why, consider a virtual channel over $k+1$ players who close the channel
  non-cooperatively via on-chain interaction. Assuming the $(k+1)$-th party goes
  last, the protocol should be able to accommodate any possible activation
  sequence for the first $k$ parties. Consecutive pairs of parties $(i,i+1)$
  need to be reactive to each other's posted transactions since they share a
  base channel. It follows that for each $i$ we can assign either ``L'' or ``R''
  signifying the directionality of reaction, resulting in a total of $2^{k-1}$
  different sequences. Without \texttt{ANYPREVOUT}, the $(k+1)$-th party needs
  a different transaction to interact with the outcome of each sequence, hence
  blowing up its local storage. The formalization of this argument is outside
  the scope of the present work.} We hope this work provides additional
  motivation for this flag to be included in the future.

  Note also that the newly established virtual channel can itself act as a base
  for further virtual channels, as its funding output can be unilaterally put
  on-chain in a pre-agreed maximum number of blocks. This in turn means that, as
  we discussed above, a further virtual channel must take the delay of its
  virtual base channels into account to determine the timelocks needed for its
  own virtual outputs.

  Let a single \emph{channel round} be a series of messages starting from the
  funder and hop by hop reaching the fundee and back again. For the actual
  protocol that establishes a virtual channel $6$ channel rounds are needed
  (c.f.\ Figure~\ref{code:ln:prepare-base}). The first communicates
  parties' identities, their funding keys, revocation keys and their neighbours'
  channel balances, the second creates new commitment transactions, the third
  communicates keys for virtual transactions (a.k.a.\ virtual keys), all parties'
  coins and desired timelocks, the fourth and the
  fifth communicate signatures for the virtual transactions (signatures for
  virtual outputs and funding outputs respectively) and the sixth shares
  revocation signatures for the old channel states.

  Cooperative closing is quite intuitive (c.f.\
  Figures~\ref{code:ln:coop-close},~\ref{code:ln:coop-closed-to-initiator},~\ref{code:ln:coop-close-fundee},~\ref{code:ln:coop-close-funder}
  and~\ref{code:virtual-layer:coop-close-intermediary}). It can be initiated by
  any party, one and a half communication rounds are needed. The funder builds
  new commitment txs, which once again spend the funding outputs that the
  virtual txs spent before, just like prior to opening the virtual channel. The
  funder sends
  their signatures to the first intermediary; the latter similarly builds, signs
  and sends new commitment txs to the second intermediary and so on until the
  fundee. The fundee
  responds with its own commitment tx signatures, along with signatures revoking
  the previous commitment tx and virtual txs. This is repeated backwards until
  revocations reach the funder. Finally the funder sends its revocation to its
  neighbour and it to the next, until the revocations reach the fundee. The
  channel has now closed cooperatively.

  As for the unilateral closing, let us now turn to an example in order to
  better grasp how our construction plays out on-chain in practice
  (Figure~\ref{figure:example-start-end}). Consider an
  established virtual channel on top of $4$ preexisting simple base channels.
  Let $A$, $B$, $C$, $D$ and $E$ be the relevant parties, which control the $(A,
  B)$, $(B, C)$, $(C, D)$ and $(D, E)$ base channels, along with the $(A, E)$
  virtual channel. After carrying out some payments, $A$ decides to unilaterally
  close the virtual channel. It therefore publishes its initiator transaction,
  thus consuming the funding output of $(A, B)$ and producing (among others) a
  virtual output with the interval $\{A\}$. $B$ notices this before the timelock
  of the virtual output expires and publishes its extend-interval
  transaction that consumes the aforementioned virtual output and the funding
  output of $(B, C)$, producing a virtual output with the interval $\{A, B\}$.
  $C$ in turn publishes the corresponding extend-interval transaction, consuming
  the virtual output of $B$ and the funding output of $(C, D)$ while producing a
  virtual output with the interval $\{A, B, C\}$. Finally $D$ publishes the last
  extend-interval transaction, thus producing an interval with all players.
  No more virtual transactions can be published. Now $A$ can spend the virtual
  output of the last extend-interval transaction with the relevant bridge
  transaction, which can then be spent by $A$'s or $E$'s latest commitment
  transaction. Note that if any of $B$, $C$ or $D$ does not act within the
  timelock prescribed in their consumed virtual output, then $A$ or $E$ can
  spend the virtual output with the relevant bridge transaction and this with
  the latest commitment transaction, thus eventually $A$ can close its virtual
  channel in all cases.

  \begin{figure*}
    \subimport{./figures/manual-tikz/}{example-start-end.tex}
    \caption{$4$ simple channels supporting a virtual. $A$ starts the closing
    procedure by publishing its initiator tx, then parties $B$-$D$ each
    publishes its extend-interval tx with the relevant interval. No party is
    negligent. Virtual outputs are marked with their interval.}
    \label{figure:example-start-end}
  \end{figure*}
