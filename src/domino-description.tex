In~\cite{donner} a process called ``Domino attack'' is presented, which
erroneously claims that a malicious virtual channel member can force others'
channels to close. According to Domino, if $A$ and $E$ of
Fig.~\ref{figure:example-start-end-simple} unilaterally close virtual channel
$(A, E)$, they then force base channels $(B, C)$ and $(C, D)$ to close. This is
not the case. Contrary to Domino's goal, honest parties $B, C, D$ are only
forced to publish a single virtual tx each, which simply places their
funding outputs on-chain; channels $(B, C)$ and $(C, D)$ remain open. Contrary
to the authors' claims, Domino is not an attack against Elmo. There is still a
small downside: the channel capacity is reduced by the collateral, which
is paid directly to one of the two base channel parties. Since no coins are
stolen, the only cost to $B$, $C$ and $D$ is the on-chain fees
of one tx. This is an inherent but small risk of
recursive channels, which must be taken into account when making one's
channel the base of another. This risk can be eliminated by making an attacker
pay the fees for the others' on-chain txs. This fee need not apply in
case of cooperative closing nor
during normal operation and can be reduced for reputable parties.
Suitable reputation systems, as well as mechanisms for assigning inactivity
blame (i.e., proving which parties tried collaborative closing before
closing unilaterally), which are needed to determine who must pay the fees,
can be designed but are beyond the scope of this work.
% splncs, short
%A possible solution to
%the recuded channel capacity is discussed in Appx.~\ref{sec:domino-prevention}.

