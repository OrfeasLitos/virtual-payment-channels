\section{High Level Explanation}
  Conceptually, our protocol is split into three main actions: channel opening,
  payments and closing. A channel $(P_1, P_n)$ between parties $P_1$ and $P_n$
  may be opened directly on-chain, in which case the two parties follow an
  opening procedure similar to that of Lightning, or it can be opened
  on top of a path of preexisting channels $(P_2, P_3), (P_3, P_4), \dots,
  (P_{n-3}, P_{n-2}), (P_{n-2}, P_{n-1})$. In the latter case all parties $P_i$
  on the path follow our novel protocol, locking funds in their channels as
  collateral for the new virtual channel that is being opened. Once all
  intermediaries are committed, $P_1$ and $P_n$ finally create (and keep
  off-chain) their ``commitment'' transaction, following a logic similar to
  Lightning.

  A payment over an established channel follows a procedure heavily inspired by
  Lightning, but without the use of HTLCs. To be completed, a payment needs
  three messages to be exchanged by the two parties.

  Finally, the closing procedure can be completed unilaterally and consists of
  signing and publishing a number of transactions on-chain. As we will discuss
  later, the exact transactions that a party will publish vary depending on the
  exact actions of the other on-path parties. Our protocol can be augmented with
  a more efficient optimistic collaborative closing procedure, which however is
  left as future work.

  In more detail, to open a channel (c.f.~\ref{code:ln:open}) the two
  counterparties first create new keypairs and exchange the resulting public
  keys (2 messages), then prepare the underlying base channels if the new
  channel is virtual (\TODO{count messages} with the number of intermediaries
  being $n-2$), next they exchange signatures for their respective initial
  commitment transactions (2 messages) and lastly, if the channel is simple
  (i.e. not virtual), the ``funder'' signs and publishes the ``funding''
  transaction on-chain.  We here note that like Lightning, only one of the two
  parties, the funder, provides coins for a new channel. This limitation
  simplifies the execution model and the analysis, but can be lifted at the cost
  of additional protocol complexity.

  In a similar vein to earlier PCN proposals, having an open channel essentially
  means having very specific keys, transactions and signatures at hand, as well
  as checking the ledger periodically and being ready to take action if
  misbehaviour is detected. Let us first consider a simple channel that has been
  established between \alice and \bob where the former owns $c_A$ and the latter
  $c_B$ coins. There are three sets of transactions at play:
  \begin{itemize}
    \item a single on-chain ``funding'' transaction with a single output that
    is encumbered with a $2/\{\alice, \bob\}$ multisig and carries $c_A + c_B$
    coins,
    \item two ``commitment'' transactions, each of which can spend the funding
    tx and produce two outputs with $c_A$ and $c_B$ coins each. The two txs
    differ in the outputs' spending conditions: The $c_A$ output in \alice's
    commitment tx can be spent either by \alice after it has been on-chain for a
    pre-agreed time (i.e. it is encumbered with a ``timelock''), or by a
    ``revocation'' transaction (discussed below), whereas the $c_B$ output can
    be spent only by \bob without any timelock. \bob's commitment tx is
    symmetric: the $c_A$ output can be spent only by \alice without timelock and
    the $c_B$ output can be spent either by \bob after the timelock expiration
    or by a revocation tx. When a new pair of commitment txs are created (either
    during channel opening or on each update) \alice signs \bob's commitment tx
    and sends him the signature (and vice-versa), therefore \alice can
    unilaterally sign and publish only her commitment tx (and vice-versa).
    \item $2m$ ``revocation'' transactions where $m$ is the number of updates.
    \TODO{continue}
  \end{itemize}

  \TODO{delete/change a lot next paragraph}
  Regarding an off-chain payment (a.k.a. channel update --
  c.f.~\ref{code:ln:pay}), the two channel parties first create the new
  commitment transactions that carry the updated balance and exchange signatures
  for them (2 messages), and afterwards create the revocation transactions that
  spend the commitment transactions of the previous version and exchange their
  signatures (2 messages).
