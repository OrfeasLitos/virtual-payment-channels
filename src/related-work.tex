\makeatletter%
\@ifclassloaded{IEEEtran}%
  {\paragraph{Related work}}%
  {\paragraph{Related work.}}%
\makeatother%
Perun~\cite{perun} and GSCN~\cite{DBLP:conf/ccs/DziembowskiFH18} exploit the
  Turing-complete scripting language of Ethereum to build virtual state
  channels.
  GSCN uses a per-channel functionality and a recursive argument similar
  to that of our UC-security analysis. Their security argument is however
  flawed, as they incorrectly argue that every level is subroutine respecting
  with respect to the same environment and subroutines.
  We believe that, given the versatile Ethereum scripting, GSCN could be
  straightforwardly extended to support variadic channels.
  Similar features are provided by Celer~\cite{dong2018celer}.
  Hydra~\cite{cryptoeprint:2020:299} provides state channels for
  Cardano~\cite{cardano}.

  Closer to Elmo, several works propose virtual channel constructions
  for Bitcoin. LVPC~\cite{10.1007/978-3-030-65411-5_18} enables a virtual channel to be
  opened on top of two preexisting channels using a technique similar to DMC,
  unfortunately inheriting the fixed lifetime limitation.
  Let \emph{simple channels} be those built directly on-chain, i.e., channels that are not
  virtual.
  Bitcoin-Compatible Virtual Channels~\cite{9519487} enables
  virtual channels on top of two preexisting simple channels
  and offers two protocols, the first of which guarantees that the channel will
  stay off-chain for an agreed period, while the second allows the single intermediary
  to turn the virtual into a simple channel.
  This strategy has the shortcoming that even if it is made
  recursive (a direction left open in~\cite{9519487}) after $k$
  applications of the constructor the virtual channel participant will have to
  publish on-chain $k$ transactions in order to close the channel if all
  intermediaries actively monitor the blockchain.

  % splncs
  %\addtolength{\intextsep}{-23pt}
  \begin{table*}
    \caption{Features \& requirements comparison of virtual channel protocols}
    \label{table:comparison-features}
    \begin{minipage}{\textwidth}
    \begin{center}
    \begin{tabular}{|l|c|c|c|c|c|}
    \hline
              & Unlimited lifetime & Recursive & Variadic & Symmetric & Script requirements \\
    \hline
    LVPC~\cite{10.1007/978-3-030-65411-5_18}
              & ✗                  & \LEFTcircle\footnote{lacks security analysis}
                                               & ✗         & ✓         & Bitcoin \\
    \hline
    BCVC~\cite{9519487}
              & ✓                  & ✗         & ✗         & ✓         & Bitcoin \\
    \hline
    Perun~\cite{perun}
              & ✓                  & ✗         & ✗         & ✓         & Ethereum \\
    \hline
    GSCN~\cite{DBLP:conf/ccs/DziembowskiFH18}
              & ✓                  & ✓         & ✗         & ✓        & Ethereum \\
    \hline
    Donner~\cite{donner}
              & ✗                  & ✗         & ✓         & ✗         & Bitcoin \\
    \hline
    this work & ✓                  & ✓         & ✓         & ✓         & Bitcoin + \texttt{ANYPREVOUT} \\
    \hline
    \end{tabular}
    \end{center}
    \end{minipage}
  \end{table*}
  % splncs
  %\addtolength{\intextsep}{23pt}

  Donner~\cite{donner}
    (released originally concurrently with the first technical report of Elmo) 
%  is the first work 
%  to 
achieves variadic, but not recursive
  virtual channels. This is
  done by having the funder lock as
  collateral twice the amount of the desired channel funds: once on-chain with
  funds that are external to the \emph{base channels} (i.e., the channels that the
  virtual channel is based on) and once off-chain within its base channel. Thus
  the funder's collateral is double that of LVPC and Elmo. The
  collateral for all other parties is the same
  across LVPC, Donner, and Elmo.
  Additionally, a Donner channel needs active periodic collaboration of the
  endpoints and all base channel parties to refresh
  its lifetime, therefore a Donner channel does not have a truly unlimited
  lifetime. We conjecture that using external coins precludes variadic
  virtual channels with unlimited lifetime. This
  design choice further means that Donner is not symmetric. Donner also uses
  placeholder outputs which, due to the minimum coins they need to
  exceed Bitcoin's \emph{dust limit}, may skew the incentives of rational players
  and add to the channel
  opportunity cost. The aforementioned incentives
  together with its lack of recursiveness mean that if a party with coins in a
  Donner channel decides to use them with another party, it first has to close
  its channel either off-chain, which needs cooperation of all intermediaries, or
  else on-chain, with all the delays and fees this entails.
  Further, its design complicates
  future iterations that lift its current restriction that only one of the two
  channel parties can fund the virtual channel. On the positive side, Donner is
  more efficient than Elmo in terms of storage, computation and communication
  complexity, and boasts a simpler design. 
  %, but has less room for optimisations and is not recursive. 
  Their work also introduces the \emph{Domino attack},
  which we address in Section~\ref{sec:future-work}.

  We refer the reader to Appx.~\ref{sec:further-related-work} for further
  related work, including a technical issue in Donner and its resolution.
  Table~\ref{table:comparison-features} contains a comparison of the
  features and limitations of virtual channel protocols, including the current
  work.
