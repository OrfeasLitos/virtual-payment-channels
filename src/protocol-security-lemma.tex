\begin{lemma}[Real world balance security]
\label{lemma:real-balance-security}
  Consider a real world execution with $P \in \{\alice, \bob\}$ honest
  \textsc{ln} ITI and $\bar{P}$ the counterparty ITI. Assume that all of the
  following are true:
  \begin{itemize}
    \item the internal variable \texttt{negligent} of $P$ has value ``False'',
    \item $P$ has transitioned to the \textsc{open} \textit{State} for the first
    time after having received $(\textsc{open}, c, \dots)$ by either
    \environment or $\bar{P}$,
    \item $P$ [has received $(\textsc{fund me}, f_i, \dots)$ as input by another
    \textsc{ln} ITI while \textit{State} was \textsc{open} and subsequently $P$
    transitioned to \textsc{open} \textit{State}] $n$ times,
    \item $P$ [has received $(\textsc{pay}, d_i)$ by \environment while
    \textit{State} was \textsc{open} and $P$ subsequently transitioned to
    \textsc{open} \textit{State}] $m$ times,
    \item $P$ [has received $(\textsc{get paid}, e_i)$ by \environment while
    \textit{State} was \textsc{open} and $P$ subsequently transitioned to
    \textsc{open} \textit{State}] $l$ times.
  \end{itemize}
  Let $\phi = 1$ if $P = \alice$, or $\phi = 0$ if $P = \bob$. If $P$ receives
  $(\textsc{close})$ by \environment and, if $\texttt{host}_P \neq \ledger$
  the output of $\texttt{host}_P$ is (\textsc{closed}), then eventually the
  state obtained when $P$ inputs $(\textsc{read})$ to \ledger will contain $h$
  $(c_i, \pk{P, \mathrm{out}})$ outputs such that
  \begin{equation}
  \label{lemma:real-balance-security:ineq}
    \sum\limits_{i=1}^h c_i \geq \phi \cdot c - \sum\limits_{i=1}^n f_i -
    \sum\limits_{i=1}^m d_i + \sum\limits_{i=1}^l e_i \enspace
  \end{equation}
  with overwhelming probability in the security parameter.
\end{lemma}

\begin{proof}
  We first note that, as signature forgeries only happen with negligible
  probability and only a polynomial number of signatures are verified by honest
  parties throughout an execution, the event in which any forged signature
  passes the verification of an honest party or of \ledger happens only with
  negligible probability. We can therefore ignore this event throughout this
  proof and simply add a computationally negligible distance between
  \environment's outputs in the real and the ideal world at the end.

  Define the \emph{history} of a channel as $H = (F, C)$, where each of $F, C$
  is a list of lists of integers. A party $P$ which satisfies the Lemma
  conditions has a unique, unambiguously and recursively defined history: If the
  value \texttt{hops} in the (\textsc{open}, $c$, $\texttt{hops}$, $\dots$)
  message was equal to \ledger, then $F$ is the empty list, otherwise $F$ is the
  concatenation of the $F$ and $C$ lists of the party that sent
  (\textsc{funded}, $\dots$) to $P$, as they were at the moment the latter
  message was sent. After initialised, $F$ remains immutable. Observe that, if
  $\texttt{hops} \neq \ledger$, both aforementioned messages must have been
  received before $P$ transitions to the \textsc{open} state.

  The list $C$ of party $P$ is initialised to $[[g]]$ when $P$'s $\itistate$
  transitions for the first time to \textsc{open}, where $g = c$ if $P =
  \alice$, or $g = 0$ if $P = \bob$; this represents the initial channel
  balance. The value $x$ or $-x$ is appended to the last list in $C$ when a
  payment is received (Fig.~\ref{code:ln:pay:revocations},
  l.~\ref{code:ln:pay:revocations:paid-in}) or sent
  (Fig.~\ref{code:ln:pay:revocations},
  l.~\ref{code:ln:pay:revocations:paid-out}) respectively by $P$. Moving on to
  the funding of new virtual channels, whenever $P$ funds a new virtual channel
  (Fig.~\ref{code:ln:virtualise:start-end},
  l.~\ref{code:ln:virtualise:start-end:reduce-coins}), $[-c_{\mathrm{virt}}]$
  is appended to $C$ and whenever $P$ helps with the opening of a new virutal
  channel, but does not fund it (Fig.~\ref{code:ln:virtualise:start-end},
  l.~\ref{code:ln:virtualise:start-end:reply}), $[0]$ is appended to $C$.
  Therefore $C$ consists of one list of integers for each sequence of inbound
  and outbound payments that have not been interrupted by a virtualisation step
  and a new list is added for every new virtual layer. We also observe that a
  non-negligent party with history $(F, C)$ satisfies the Lemma conditions and
  that the value of the right hand side of the
  inequality~(\ref{lemma:real-balance-security:ineq}) is equal to
  $\sum\limits_{s \in C} \sum\limits_{x \in s} x$, as all inbound and outbound
  payment values and new channel funding values that appear in the Lemma
  conditions are recorded in $C$.

  Let party $P$ with a particular history. We will inductively prove that $P$
  satisfies the Lemma. The base case is when a channel is opened with
  $\texttt{hops} = \ledger$ and is closed right away, therefore $H = ([], [g])$,
  where $g = c$ if $P = \alice$ and $g = 0$ if $P = \bob$.  $P$ can transition
  to the \textsc{open} \textit{State} for the first time only if all of the
  following have taken place:
  \begin{itemize}
    \item It has received (\textsc{open}, $c$, $\dots$) while in the
    \textsc{init} \textit{State}. In case $P = \alice$, this message must have
    been received as input by \environment (Fig.~\ref{code:ln:open},
    l.~\ref{code:ln:open:alice-open}), or in case $P = \bob$, this message must
    have been received via the network by $\bar{P}$
    (Fig.~\ref{code:ln:exchange-open-keys},
    l.~\ref{code:ln:exchange-open-keys:bob-open}).
    \item It has received $\pk{\bar{P}, F}$. In case $P = \bob$, $\pk{\bar{P},
    F}$ must have been contained in the (\textsc{open}, $\dots$) message by
    $\bar{P}$ (Fig.~\ref{code:ln:exchange-open-keys},
    l.~\ref{code:ln:exchange-open-keys:bob-open}), otherwise if $P = \alice$
    $\pk{\bar{P}, F}$ must have been contained in the (\textsc{accept channel},
    $\dots$) message by $\bar{P}$ (Fig.~\ref{code:ln:exchange-open-keys},
    l.~\ref{code:ln:exchange-open-keys:accept-channel}).
    \item It internally holds a signature on the commitment transaction $C_{P,
    0}$ that is valid when verified with public key $\pk{\bar{P}, F}$
    (Fig.~\ref{code:ln:exchange-open-sigs},
    ll.~\ref{code:ln:exchange-open-sigs:b-verify}
    and~\ref{code:ln:exchange-open-sigs:a-verify}).
    \item It has the transaction $F$ in the \ledger state
    (Fig.~\ref{code:ln:commit-base}, l.~\ref{code:ln:commit-base:f-in-state} or
    Fig.~\ref{code:ln:bob}, l.~\ref{code:ln:bob:state-open}).
  \end{itemize}

  We observe that $P$ satisfies the Lemma conditions with $m = n = l = 0$.
  Before transitioning to the \textsc{open} \textit{State}, $P$ has produced
  only one valid signature for the ``funding'' output $(c, 2/\{\pk{P, F},
  \pk{\bar{P}, F}\})$ of $F$ with $\sk{P, F}$, namely for $C_{\bar{P}, 0}$
  (Fig.~\ref{code:ln:exchange-open-sigs},
  ll.~\ref{code:ln:exchange-open-sigs:a-sign}
  or~\ref{code:ln:exchange-open-sigs:b-sign}), and sent it to $\bar{P}$
  (Fig.~\ref{code:ln:exchange-open-sigs},
  ll.~\ref{code:ln:exchange-open-sigs:a-send}
  or~\ref{code:ln:exchange-open-sigs:b-send}), therefore the only two ways to
  spend $(c, 2/\{\pk{P, F}, \pk{\bar{P}, F}\})$ are by either publishing $C_{P,
  0}$ or $C_{\bar{P}, 0}$. We observe that $C_{P, 0}$ has a ($g$, ($\pk{P,
  \mathrm{out}} + (t + s)$) $\vee$ $2/\{\pk{P, R}, \pk{\bar{P}, R}\}$) output
  (Fig.~\ref{code:ln:exchange-open-sigs},
  l.~\ref{code:ln:exchange-open-sigs:a-tx}
  or~\ref{code:ln:exchange-open-sigs:b-tx}). The spending method $2/\{\pk{P, R},
  \pk{\bar{P}, R}\}$ cannot be used since $P$ has not produced a signature for
  it with $\sk{P, R}$, therefore the alternative spending method, $\pk{P,
  \mathrm{out}} + (t + s)$, is the only one that will be spendable if $C_{P, 0}$
  is included in \ledger, thus contributing $g$ to the sum of outputs that
  contribute to inequality~(\ref{lemma:real-balance-security:ineq}). Likewise,
  if $C_{\bar{P}, 0}$ is included in \ledger, it will contribute at least one
  ($g$, $\pk{P, \mathrm{out}}$) output to this inequality, as $C_{\bar{P}, 0}$
  has a ($g$, $\pk{P, \mathrm{out}}$) output
  (Fig.~\ref{code:ln:exchange-open-sigs},
  l.~\ref{code:ln:exchange-open-sigs:a-tx}
  or~\ref{code:ln:exchange-open-sigs:b-tx}). Additionally, if $P$ receives
  (\textsc{close}) by \environment while $H = ([], [g])$, it attempts to publish
  $C_{P, 0}$ (Fig.~\ref{code:ln:close}, l.~\ref{code:ln:close:submit}), and will
  either succeed or $C_{\bar{P}, 0}$ will be published instead. We therefore
  conclude that in every case \ledger will eventually have a state $\Sigma$ that
  contains at least one $(g, \pk{P, \mathrm{out}})$ output, therefore satisfying
  the Lemma consequence.

  Let $P$ with history $H = (F, C)$. The induction hypothesis is that the Lemma
  holds for $P$. Let $c_P$ the sum in the right hand side of
  inequality~(\ref{lemma:real-balance-security:ineq}). In order to perform the
  induction step, assume that $P$ is in the \textsc{open} state. We will prove
  all the following (the facts to be proven are shown with emphasis for
  clarity):
  \begin{itemize}
    \item If $P$ receives (\textsc{fund me}, $f$, $\dots$) by a (local, trusted)
    \textsc{ln} ITI $R$, subsequently transitions back to the \textsc{open}
    state (therefore moving to history $(F, C')$ where $C'$ is $C + [-f]$) and
    finally receives (\textsc{close}) by \environment and (\textsc{closed}) by
    $\texttt{host}_P$ before any further change to its history, then
    \emph{eventually $P$'s \ledger state will contain $h$ $(c_i, \pk{P,
    \mathrm{out}})$ transaction outputs such that $\sum\limits_{i=1}^h c_i \geq
    \sum\limits_{s \in C'} \sum\limits_{x \in s} x$}. Furthermore, given that
    $P$ moves to the \textsc{open} state after the (\textsc{fund me}, $\dots$)
    message, it also sends (\textsc{funded}, $\dots$) to $R$
    (Fig.~\ref{code:ln:virtualise:start-end},
    l.~\ref{code:ln:virtualise:start-end:funder-funded}). If subsequently the
    state of $R$ transitions to \textsc{open} (therefore obtaining history
    $(F_R, C_R)$ where $F_R = F + C$ and $C_R = [[f]]$), and finally receives
    (\textsc{close}) by \environment and (\textsc{closed}) by $\texttt{host}_R$
    ($\texttt{host}_R = \texttt{host}_P$ -- Fig.~\ref{code:ln:bob},
    l.~\ref{code:ln:bob:host}) before any further change to its history, then
    \emph{eventually $R$'s \ledger state will contain $k$ $(c^R_i, \pk{R,
    \mathrm{out}})$ transaction outputs such that $\sum\limits_{i=1}^k c^R_i
    \geq \sum\limits_{s \in C_R} \sum\limits_{x \in s} x$}.
    \item If $P$ receives (\textsc{virtualising}, $\dots$) by $\bar{P}$,
    subsequently transitions back to \textsc{open} (therefore moving to history
    $(F, C')$ where $C'$ is $C + [0]$) and finally receives \textsc{close} by
    \environment and (\textsc{closed}) by $\texttt{host}_P$ before any further
    change to its history, then \emph{eventually $P$'s \ledger state will
    contain $h$ $(c_i, \pk{P, \mathrm{out}})$ transaction outputs such that
    $\sum\limits_{i=1}^h c_i \geq \sum\limits_{s \in C} \sum\limits_{x \in s}
    x$}. Furthermore, given that $P$ moves to the \textsc{open} state after the
    (\textsc{virtualising}, $\dots$) message and in case it sends
    (\textsc{funded}, $\dots$) to some party $R$
    (Fig.~\ref{code:ln:virtualise:start-end},
    l.~\ref{code:ln:virtualise:start-end:helper-output-funded}), the latter
    party is the (local, trusted) \texttt{fundee} of a new virtual channel. If
    subsequently the state of $R$ transitions to \textsc{open} (therefore
    obtaining history $(F_R, C_R)$ where $F_R = F + C$ and $C_R = [[0]]$), and
    finally receives (\textsc{close}) by \environment and (\textsc{closed}) by
    $\texttt{host}_R$ ($\texttt{host}_R = \texttt{host}_P$ --
    Fig.~\ref{code:ln:bob}, l.~\ref{code:ln:bob:host}) before any further change
    to its history, then \emph{eventually $R$'s \ledger state will contain $k$
    $(c^R_i, \pk{R, \mathrm{out}})$ transaction outputs such that
    $\sum\limits_{i=1}^k c^R_i \geq \sum\limits_{s \in C_R} \sum\limits_{x \in
    s} x$}.
    \item If $P$ receives (\textsc{pay}, $d$) by \environment, subsequently
    transitions back to \textsc{open} (therefore moving to history $(F, C')$
    where $C'$ is $C$ with $-d$ appended to the last list of $C$) and finally
    receives \textsc{close} by \environment and (\textsc{closed}) by
    $\texttt{host}_P$ (the latter only if $\texttt{host}_P \neq \ledger$ or
    equivalently $F \neq []$) before any further change to its history, then
    \emph{eventually $P$'s \ledger state will contain $h$ $(c_i, \pk{P,
    \mathrm{out}})$ transaction outputs such that $\sum\limits_{i=1}^h c_i \geq
    \sum\limits_{s \in C} \sum\limits_{x \in s} x$}.
    \item If $P$ receives (\textsc{get paid}, $e$) by \environment, subsequently
    transitions back to \textsc{open} (therefore moving to history $(F, C')$
    where $C'$ is $C$ with $e$ appended to the last list of $C$) and finally
    receives \textsc{close} by \environment and (\textsc{closed}) by
    $\texttt{host}_P$ (the latter only if $\texttt{host}_P \neq \ledger$ or
    equivalently $F = []$) before any further change to its history, then
    \emph{eventually $P$'s \ledger state will contain $h$ $(c_i, \pk{P,
    \mathrm{out}})$ transaction outputs such that $\sum\limits_{i=1}^h c_i \geq
    \sum\limits_{s \in C} \sum\limits_{x \in s} x$}.
  \end{itemize}

  By the induction hypothesis, before the funding procedure started $P$ could
  close the channel and end up with a ($c_P$, $\pk{P, \mathrm{out}}$) output
  on-chain. When $P$ is in the \textsc{open} state and receives (\textsc{fund
  me}, $f$, $\dots$), it can only move again to the \textsc{open} state after
  doing the following state transitions: \textsc{open} $\rightarrow$
  \textsc{virtualising} $\rightarrow$ \textsc{waiting for revocation}
  $\rightarrow$ \textsc{waiting for inbound revocation} $\rightarrow$
  \textsc{waiting for hosts ready} $\rightarrow$ \textsc{open}. During this
  sequence of events, a new $\texttt{host}_P$ is defined
  (Fig.~\ref{code:ln:virtualise:start-end},
  l.~\ref{code:ln:virtualise:start-end:define}), new commitment transactions are
  negotiated with $\bar{P}$ (Fig.~\ref{code:ln:virtualise:start-end},
  l.~\ref{code:ln:virtualise:start-end:virtual-update}), control of the old
  funding output is handed over to $\texttt{host}_P$
  (Fig.~\ref{code:ln:virtualise:start-end},
  l.~\ref{code:ln:virtualise:start-end:host-me}), $\texttt{host}_P$ negotiates
  with its counterparty a new set of transactions and signatures that spend the
  aforementioned funding output and make available a new funding output with the
  keys $\pk{P, F}', \pk{\bar{P}, F}'$ as $P$ instructed
  (Fig.~\ref{code:virtual-layer:funder-sigs}
  and~\ref{code:virtual-layer:funding-sigs}) and the previous valid commitment
  transactions of both $P$ and $\bar{P}$ are invalidated
  (Fig.~\ref{code:ln:methods-for-virt},
  l.~\ref{code:ln:methods-for-virt:revoke-previous} and
  l.~\ref{code:ln:methods-for-virt:process-remote-revocation} respectively). We
  note that the use of the \texttt{ANYPREVOUT} flag in all signatures that
  correspond to transaction inputs that may spend various different transaction
  outputs ensures that this is possible, as it avoids tying each input to a
  specific, predefined output.  When $P$ receives (\textsc{close}) by
  \environment, it inputs (\textsc{close}) to $\texttt{host}_P$
  (Fig.~\ref{code:ln:close}, l.~\ref{code:ln:close:relay}). As per the Lemma
  conditions, $\texttt{host}_P$ will output (\textsc{closed}). This can only
  happen only when \ledger contains a suitable output for both $P$'s and $R$'s
  channel (Fig.~\ref{code:virtual-layer:check-chain-close},
  and~\ref{code:virtual-layer:check-chain-close:funder:output-funder}
  ll.~\ref{code:virtual-layer:check-chain-close:funder:output-virt}
  respectively).

  If the \texttt{host} of $\texttt{host}_P$ is \ledger, then the funding output
  $o_{1, 2} = (c_P + c_{\bar{P}}, 2/\{\pk{P, F}, \pk{\bar{P}, F}\})$ for the
  $A_1, A_2$ channel is already on-chain.  Regarding the case in which
  $\texttt{host}_P \neq \ledger$, after the funding procedure is complete, the
  new $\texttt{host}_P$ will have as its \texttt{host} the old $\texttt{host}_P$
  of $P$. If the (\textsc{close}) sequence is initiated, the new
  $\texttt{host}_P$ will follow the same steps that will be described below once
  the old $\texttt{host}_P$ succeeds in closing the lower layer
  (Fig.~\ref{code:virtual-layer:close},
  l.~\ref{code:virtual-layer:close:if-nested-host}). The old $\texttt{host}_P$
  however will see no difference in its interface compared to what would happen
  if $P$ had received (\textsc{close}) before the funding procedure, therefore
  it will successfully close by the induction hypothesis. Thereafter the process
  is identical to the one when the old $\texttt{host}_P = \ledger$.

  Moving on, $\texttt{host}_P$ is either able to publish its $\mathrm{TX}_{1,
  1}$ (it has necessarily received a valid signature
  $\mathrm{sig}(\mathrm{TX}_{1, 1}, \pk{\bar{P}, F})$
  (Fig.~\ref{code:virtual-layer:funding-sigs},
  l.~\ref{code:virtual-layer:funding-sigs:funder-check-sig}) by its counterparty
  before it moved to the \textsc{open} state for the first time), or the output
  $(c_P + c_{\bar{P}}, 2/\{\pk{P, F}, \pk{\bar{P}, F}\})$ needed to spend
  $\mathrm{TX}_{1, 1}$ has already been spent. The only other transactions that
  can spend it are $\mathrm{TX}_{2, 1}$ and any of $(\mathrm{TX}_{2, 2, k})_{k >
  2}$, since these are the only transactions that spend the aforementioned
  output and that $\texttt{host}_P$ has signed with $\sk{P, F}$
  (Fig.~\ref{code:virtual-layer:funding-sigs},
  ll.~\ref{code:virtual-layer:funding-sigs:funder-sign-first}-\ref{code:virtual-layer:funding-sigs:funder-sign-second-end}).
  The output can be also spent by old, revoked commitment transactions, but in
  that case $\texttt{host}_P$ would not have output (\textsc{closed}); $P$ would
  have instead detected this triggered by a (\textsc{check chain for closed})
  message by \environment (Fig.~\ref{code:ln:poll}) and would have moved to the
  \textsc{closed} state on its own accord (lack of such a message by
  \environment would lead $P$ to become \texttt{negligent}, something that
  cannot happen according to the Lemma conditions). Every transaction among
  $\mathrm{TX}_{1, 1}$, $\mathrm{TX}_{2, 1}$, $(\mathrm{TX}_{2, 2, k})_{k > 2}$
  has a ($c_P + c_{\bar{P}} - f$, $2/\{\pk{P, F}', \pk{\bar{P}, F}'\}$) output
  (Fig.~\ref{code:virtual-layer:endpoint-txs},
  l.~\ref{code:virtual-layer:endpoint-txs:new-fund} and
  Fig.~\ref{code:virtual-layer:mid-txs},
  ll.~\ref{code:virtual-layer:mid-txs:initiator:left-new-fund}
  and~\ref{code:virtual-layer:mid-txs:extend-interval-right:new-fund}) which
  will end up in \ledger{} -- call this output $o_P$. We will prove that at most
  $\sum\limits_{i=2}^{n-1}(t_i + p + s - 1)$ blocks after (\textsc{close}) is
  received by $P$, an output $o_R$ with $c_{\mathrm{virt}}$ coins and a
  $2/\{\pk{R, F}, \pk{\bar{R}, F}\}$ spending condition without or with an
  expired timelock will be included in \ledger. In case party $A_2$ is idle,
  then $o_{1, 2}$ is consumed by $\mathrm{TX}_{1, 1}$ and the timelock on its
  virtual output expires, therefore the required output $o_R$ is on-chain.  In
  case $A_2$ is active, exactly one of $\mathrm{TX}_{2, 1}$, $(\mathrm{TX}_{2,
  2, k})_{k > 2}$ or $(\mathrm{TX}_{2, 3, 1, k})_{k > 2}$ is a descendant of
  $o_{1, 2}$; if the transaction belongs to one of the two last transaction
  groups then necessarily $\mathrm{TX}_{1, 1}$ is on-chain in some block height
  $h$ and given the timelock on the virtual output of $\mathrm{TX}_{1, 1}$,
  $A_2$'s transaction can be at most at block height $h + t_2 + p + s - 1$. If
  $n=3$ or $k=n-1$, then $A_2$'s unique transaction has the required output
  $o_R$ (without a timelock). The rest of the cases are covered by the following
  sequence of events:

  \begin{center}
    \begin{notitlebox}{Closing sequence}
      \begin{algorithmic}[1]
        \State $\texttt{maxDel} \gets t_2 + p + s - 1$ \Comment{$A_2$ is active
        and the virtual output of $\mathrm{TX}_{1, 1}$ has a timelock of $t_2$}
        \State $i \gets 3$
        \State \textbf{loop}
        \Indent
          \If{$A_i$ is idle}
            \State The timelock on the virtual output of the transaction
            published by $A_{i-1}$ expires and therefore the required $o_R$ is
            on-chain
          \Else \: \Comment{$A_i$ publishes a transaction that is a descendant
          of $o_{1, 2}$}
            \State $\texttt{maxDel} \gets \texttt{maxDel} + t_i + p + s - 1$
            \State The published transaction can be of the form $\mathrm{TX}_{i,
            2, 2}$ or $(\mathrm{TX}_{i, 3, 2, k})_{k > i}$ as it spends the
            virtual output which is encumbered with a public key controlled by
            $R$ and $R$ has only signed these transactions
            \If{$i = n-1$ or $k \geq n-1$} \Comment{The interval contains all
            intermediaries}
              \State The virtual output of the transaction is not timelocked and
              has only a $2/\{\pk{R, F}, \pk{\bar{R}, F}\}$ spending method,
              therefore it is the required $o_R$
            \Else \: \Comment{At least one intermediary is not in the interval}
              \IfThenElse{the transaction is $\mathrm{TX}_{i, 3, 2, k}$}{$i
              \gets k$}{$i \gets i+1$}
            \EndIf
          \EndIf
        \EndIndent
        \State \textbf{end loop}
        \State \Comment{$\texttt{maxDel} \leq \sum\limits_{i=2}^{n-1}(t_i + p +
        s - 1)$}
      \end{algorithmic}
    \end{notitlebox}
    \captionof{figure}{}
    \label{code:settling-process}
  \end{center} \ \\

  In every case $o_P$ and $o_R$ end up on-chain in at most $s$ and
  $\sum\limits_{i=2}^{n-1}(t_i + p + s - 1)$ blocks respectively from the moment
  (\textsc{close}) is received. The output $o_P$ an be spent either by $C_{P,
  i}$ or $C_{\bar{P}, i}$. Both these transactions have a $(c_P - f, \pk{P,
  \mathrm{out}})$ output. This output of $C_{P, i}$ is timelocked, but the
  alternative spending method cannot be used as $P$ never signed a transaction
  that uses it (as it is reserved for revocation, which has not taken place yet
  in this virtualisation layer). We have now proven that if $P$ completes the
  funding of a new channel then it can close its channel for a ($c_P - f$,
  $\pk{P, \mathrm{out}}$) output, i.e. the first claim of the first bullet.

  We will now prove that the newly funded party $R$ can close its channel
  securely. After $R$ receives (\textsc{funded}, $\texttt{host}_P$, $\dots$) by
  $P$ and before moving to the \textsc{open} state, it receives
  $\mathrm{sig}_{\bar{R}, C, 0}$ which is a valid signature on $C_{R, 0}$ by
  $\sk{\bar{R}, F}$ and sends $\mathrm{sig}_{R, C, 0}$ which is a valid
  signature on $C_{\bar{R}, 0}$ by $\sk{R, F}$. Both these transactions spend
  $o_R$. As we showed before, if $R$ receives (\textsc{close}) by \environment
  then $o_R$ eventually ends up on-chain. After receiving (\textsc{closed}) from
  $\texttt{host}_P$, $R$ attempts to add $C_{R, 0}$ to \ledger, which may only
  fail if $C_{\bar{R}, 0}$ ends up on-chain instead.  Similar to the case of
  $P$, both these transactions have an $(f, \pk{R, \mathrm{out}})$ output. This
  output of $C_{R, 0}$ is timelocked, but the alternative spending method cannot
  be used as $R$ never signed a transaction that uses it (as it is reserved for
  revocation, which has not taken place yet) so the timelock will expire and the
  desired spending method will be available. We have now proven that if $R$'s
  channel is funded to completion then it can close its channel for a ($f$,
  $\pk{R, \mathrm{out}}$) output. We have therefore proven the first bullet.

%  May be needed for naming the txs and the intervals:
%
%  Call $n$ the number of base channels used for the virtual channel and
%  the following transactions ``virtual transactions'' collectively. $\forall i
%  \in [n]$, ``initiator'' $\mathrm{TX}_{i, 1}$ has an ``out-interval'' equal to
%  $\{i\}$. $\forall i \in \{2, \dots, n-1\}$, ``extend-interval''
%  $\mathrm{TX}_{i, 2, k}$ has an ``in-interval'' equal to $\{\min{(i+1, k)},
%  \dots, \max{(i-1, k)}\}$ and an ``out-interval'' equal to $\{\min{(i, k)},
%  \dots, \max{(i, k)}\}$. $\forall i \in [n]$ ``merge-intervals''
%  $\mathrm{TX}_{i, 3, k_1, k_2}$ has two ``in-intervals'' equal to $\{k_1,
%  \dots, i-1\}$ and $\{i+1, \dots, k_2\}$ respectively, and an ``out-interval''
%  equal to $\{k_1, \dots, k_2\}$. Observe that a party cannot add any other
%  party but itself to an interval, that if it extends a single already existing
%  interval (with a merge-interval transaction), it can only extend an iterval
%  that ends just before or just after $i$ and that if it merges two intervals
%  (with a merge-interval transaction), $i$ has to be the only missing link for
%  the two intervals to become one uninterrupted interval. Furthermore all
%  virtual transactions add the publishing party to the interval. Given that, if
%  all players are honest, all funding and interval-carrying (i.e.
%  ``\emph{all}'') outputs are unique and that for an intermediary to add itself
%  to a preexisting interval to its left/right it must be the case that its
%  funding output to its left/right respectively has been already spent, the only
%  way for a party that is implicated in a virtual channel (either as
%  intermediary or as endpoint) to publish more than one virtual transaction is
%  if a malicious party has published a suitable copy of a funding or
%  interval-carrying output (at the malicious party's own expense). By inspection
%  of the virtual transactions, we deduce that no honest party stands to lose
%  any funds if such copies are introduced. Indeed, an intermediary that
%  publishes a virtual transaction immediately gains $c_{\mathrm{virt}}$ coins,
%  generates a new base funding output for each old base funding output it
%  consumed, where the left party owns $c_{\mathrm{virt}}$ coins less than in the
%  old base output. If the latter party has already made its move, that means
%  that 

\end{proof}
