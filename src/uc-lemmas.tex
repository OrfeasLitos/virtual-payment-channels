\begin{theorem}[Simple Payment Channel Security]
  \label{theorem:security:simple}
  The protocol $\pchansup{1}$ UC-realises $\fchansup{1}$ in the presence of a
  global functionality $\ledger$ and assuming the security of the underlying
  digital signature:
  \begin{gather*}
    \forall \text{ PPT } \adversary, \exists \text{ PPT } \simulator: \forall
    \text{ PPT } \environment \text{ it is }\\
    \textsc{exec}^{\ledger}_{\pchansup{1}, \adversary, \environment} \approx
    \textsc{exec}^{\fchansup{1}, \ledger}_{\simulator, \environment} \enspace.
  \end{gather*}
\end{theorem}

  The corresponding proof stems from
  Lemma~\ref{lemma:no-halt}, the fact that \fchan is a simple relay and that
  \simulator faithfully simulates \pchan. Lastly we prove that $\forall n \geq 2, \pchansup{n}$ UC-realises $\fchansup{n}$
  in the presence of $\fchansup{1}, \dots, \fchansup{n-1}$
  (leveraging the relevant definition
  from~\cite{DBLP:conf/tcc/BadertscherCHTZ20}).

\begin{theorem}[Recursive Virtual Payment Channel Security]
  \label{theorem:security:virtual}
  $\forall n \in \mathbb{N}^* \setminus \{1\}$, the protocol $\pchansup{n}$
  UC-realises $\fchansup{n}$ in the presence of $\fchansup{1}, \dots,
  \fchansup{n-1}$ and \ledger, assuming the security of the underlying digital
  signature. Specifically,
  \begin{gather*}
    \forall n \in \mathbb{N}^* \setminus \{1\}, \forall \text{ PPT } \adversary,
    \exists \text{ PPT } \simulator: \forall \text{ PPT } \environment \text{ it
    is } \\
    \textsc{exec}^{\ledger, \fchansup{1}, \dots, \fchansup{n-1}}_{\pchansup{n},
    \adversary, \environment} \approx
    \textsc{exec}^{\fchansup{n}, \ledger, \fchansup{1}, \dots,
    \fchansup{n-1}}_{\simulator, \environment} \enspace.
  \end{gather*}
\end{theorem}

\makeatletter%
\@ifclassloaded{IEEEtran}%
{\begin{IEEEproof}}
{\begin{proof}}
\makeatother%
[Proof of Theorem~\ref{theorem:security:simple}]
  By inspection of Figures~\ref{code:functionality:rules}
  and~\ref{code:simulator:flow} we can deduce that for a particular
  \environment, in the ideal world execution $\textsc{exec}^{\fchansup{1},
  \ledger}_{\simulator_{\adversary}, \environment}$, $\simulator_{\adversary}$
  simulates internally the two $\pchansup{1}$ parties exactly as they would execute in
  $\textsc{exec}^{\ledger}_{\pchansup{1}, \adversary, \environment}$, the real world
  execution, in case $\fchansup{1}$ does not halt. Indeed, $\fchansup{1}$ only halts with
  negligible probability according to Lemma~\ref{lemma:no-halt}, therefore the
  two executions are computationally indistinguishable.
\makeatletter%
\@ifclassloaded{IEEEtran}%
{\end{IEEEproof}}
{\end{proof}}
\makeatother%

\makeatletter%
\@ifclassloaded{IEEEtran}%
{\begin{IEEEproof}}
{\begin{proof}}
\makeatother%
[Proof of Theorem~\ref{theorem:security:virtual}]
  The proof is exactly the same as that of
  Theorem~\ref{theorem:security:simple}, replacing superscripts $1$ for $n$.
\makeatletter%
\@ifclassloaded{IEEEtran}%
{\end{IEEEproof}}
{\end{proof}}
\makeatother%

Since we use a global setup, proving UC-emulation is not enough. We further
need to prove that all ideal global subroutines are \emph{replaceable}, i.e.,
they can be replaced with their real counterparts. This guarantees that a real
deployment will offer the same security guarantees as its idealized description.

For any $i \in [n]$, the individual global subroutines $\ledger, \fchansup{1},
\dots, \fchansup{i}$ can be merged (as per Def.~4.1
of~\cite{10.1007/978-3-030-90453-1_22}) into the ``global setup''
$\mathcal{G}^i$. Likewise, the realisation $\pledger$ of $\ledger$~\cite{BMTZ17}
and $\pchansup{1}, \dots, \pchansup{i}$ can be merged into $\Pi^i$.

\begin{lemma}
\label{lemma:merged-emulate}
For all $i \in [n]$, $\Pi^i$ UC-emulates $\mathcal{G}^i$.
\end{lemma}

\makeatletter%
\@ifclassloaded{IEEEtran}%
{\begin{IEEEproof}}
{\begin{proof}}
\makeatother%
[Proof of Lemma~\ref{lemma:merged-emulate}]
The following facts hold (note the inversion of the order of indices compared to
the formulation of Theorem~4.3): $\ledger$ and $\pledger$ are \emph{subroutine
respecting} (Def.~A.6~\cite{10.1007/978-3-030-90453-1_22}) as they do not
accept/pass inputs or outputs from/to parties outside their session (this can be
verified by inspection). $\fchansup{1}$ and $\pchansup{1}$ are $\ledger$- and
$\pledger$-subroutine respecting respectively
(Def.~2.3~\cite{10.1007/978-3-030-90453-1_22}), as they they do not accept/pass
inputs or outputs from/to parties outside their session apart from $\ledger$ and
$\pledger$ respectively. For any $j \in [i] \setminus \{1\}$, $\fchansup{j}$ and
$\pchansup{j}$ are $\mathcal{G}^{j-1}$- and $\Pi^{j-1}$-subroutine respecting
respectively, as they they do not accept/pass inputs or outputs from/to parties
outside their session apart from $\mathcal{G}^{j-1}$ and $\Pi^{j-1}$
respectively. Theorem~4.3 of~\cite{10.1007/978-3-030-90453-1_22} then implies
the required result.
\makeatletter%
\@ifclassloaded{IEEEtran}%
{\end{IEEEproof}}
{\end{proof}}
\makeatother%

\begin{theorem}[Full Replacement]
\label{theorem:replacement}
For all $i \in [n]$, the ideal global setup $\mathcal{G}^i$ can be replaced with
$\Pi^i$.
\end{theorem}

\makeatletter%
\@ifclassloaded{IEEEtran}%
{\begin{IEEEproof}}
{\begin{proof}}
\makeatother%
Simulator \simulator{} (Figs.~\ref{code:simulator:flow}--\ref{code:simulator})
is $\mathcal{G}$-\emph{agnostic} (Def.~3.4
of~\cite{10.1007/978-3-030-90453-1_22}) as \simulator{} is a relay between
parties and the environment \environment. This can be verified by inspection of
Fig.~\ref{code:simulator:flow}. Thus Theorem~3.5
of~\cite{10.1007/978-3-030-90453-1_22}, full replacement, applies inductively to
$\mathcal{G}^i$ and $\Pi^i$.
\makeatletter%
\@ifclassloaded{IEEEtran}%
{\end{IEEEproof}}
{\end{proof}}
\makeatother%
