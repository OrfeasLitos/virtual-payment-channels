\section{Model \& Construction}
  In this section we will examine the architecture and the details of our model,
  along with possible attacks and their mitigations. Following the UC
  framework~\cite{uc}, we define an ideal-world functionality \fchan
  (Figures~\ref{code:functionality:rules}-\ref{code:functionality:close}) and a
  simulator \simulator (Figures~\ref{code:simulator:flow}-\ref{code:simulator}),
  along with a real-world protocol \pchan
  (Figures~\ref{code:ln:init}-\ref{code:virtual-layer:punishment}) that
  UC-realizes \fchan (Theorem~\ref{theorem:security}).

  Our ideal world functionality \fchan represents a single channel, either
  simple or virtual. It acts as a relay between \adversary and \environment,
  leaking all messages. This simplifies the functionality and facilitates the
  indistinguishability argument by having \simulator simply running internally
  the real world protocols of the channel parties \pchan with no modifications.
  \fchan internally maintains a state machine (c.f. Figure~\TODO{state machine})
  that keeps track of which internal parties are corrupted or negligent, whether
  the channel has opened, whether a payment is underway, which external parties
  are to be considered trusted (as they correspond to other channels owned by
  the same player) and whether the channel has closed. The single security check
  performed is whether the on-chain coins are at least equal to the expected
  balance once the channel closes. If this check fails, \fchan halts.

  Our real world protocol \pchan, ran by party $P$, consists of two
  subprotocols: the Lightning-inspired part, dubbed \textsc{ln}
  (Figures~\ref{code:ln:init}-\ref{code:ln:used-revocation}) and the novel
  virtual layer subprotocol, named \textsc{virt}
  (Figures~\ref{code:virtual-layer:keys}-\ref{code:virtual-layer:punishment}).

  The \textsc{ln} subprotocol has two variations depending on whether $P$ is the
  channel funder (\alice) or the fundee (\bob). It performs a number of tasks:
  Initialisation takes a single step for fundees and two steps for funders.
  \textsc{ln} first receives a public key $\pk{P, \mathrm{out}}$ from
  \environment. This is the public key that should eventually own all $P$'s
  coins after the channel is closed. \textsc{ln} also initialises its internal
  variables. If $P$ is a funder, \textsc{ln} waits for a second activation to
  generate a keypair and then waits for \environment to endow it with some
  coins, which will be subsequently used to open the channel
  (Figure~\ref{code:ln:init}).

  After initialisation, the funder \alice is ready to open the channel. Once it
  is given by \environment \bob's identity, the initial channel balance $c$ and,
  in case it is a virtual, the identities of the base channel owners
  (Figure~\ref{code:ln:open}), \alice generates and sends \bob her funding and
  revocation public keys ($\pk{A, F}$, $\pk{A, R}$) along with $c$, $\pk{A,
  \mathrm{out}}$, and the base channel identities (if any). Given that \bob has
  been initialised, it generates funding and revocation keys and replies to
  \alice with $\pk{B, F}$, $\pk{B, R}$, and $\pk{B, \mathrm{out}}$
  (Figure~\ref{code:ln:exchange-open-keys}).

  The next step prepares the base channels (Figure~\ref{code:ln:prepare-base}).
  If our channel is a simple one, then \alice simply generates the funding tx.
  If it is a virtual and assuming all base parties (running \textsc{ln})
  cooperate, a chain of messages from \alice to \bob and back via all base
  parties is initiated (Figures~\ref{code:ln:virtualise:start-end}
  and~\ref{code:ln:open:virtualise:hops}). These messages let each successive
  neighbour know the identities of all the base parties. Furthermore each party
  instantiates a new ``host'' party that runs \textsc{virt}. It also generates
  new funding keys and communicates them, along with its out key and its
  leftward and rightward balances. If this circuit of messages completes, \alice
  delegates the creation of the new virtual layer transactions to its new
  \textsc{virt} host, which will be discussed later in detail. If the virtual
  layer is successful, each base party is informed by its host accordingly,
  intermediaries return to the \textsc{open} state and \alice and \bob continue
  the opening procedure. In particular, \alice and \bob exchange signatures on
  the initial commitment transactions, therefore ensuring that the funding
  output can be spent (Figure~\ref{code:ln:exchange-open-sigs}). After that, in
  case the channel is simple the funding transaction is put on-chain
  (Figure~\ref{code:ln:commit-base}) and finally \environment is informed of the
  successful channel opening.

  There are two facts that should be noted: Firstly, in case the opened channel
  is virtual, each intermediary base party necessarily partakes in two channels.
  However each protocol instance only represents a party in a single channel,
  therefore each intermediary is in practice realised by two mutually trusted
  \pchan instances that communicate locally, called ``siblings''. Secondly, our
  protocol is not designed to gracefully recover if other parties do not send an
  expected message at any point in the opening or payment procedure. Such
  anti-Denial-of-Service measures would greatly complicate the protocol and are
  left as a task for a real world implementation. It should be however stressed
  that an honest party with an open channel that has fallen victim to such an
  attack can still unilaterally close the channel, therefore no coins are lost
  in any case.

  Once the channel is open, \alice and \bob can carry out an unlimited number of
  payments in either direction with a speed that is bounded only by network
  delay. The payment procedure is identical for simple and virtual channels and
  crucially it does not implicate the intermediaries. For a payment to be
  carried out, the payee is first notified by \environment
  (Figure~\ref{code:ln:get-paid}) and subsequently the payer is instructed by
  \environment to commence the payment (Figure~\ref{code:ln:pay}).

  If the channel is virtual, each party also checks that its upcoming balance is
  lower than the balance of its sibling's counterparty and that the upcoming
  balance of the counterparty is higher than the balance of its own sibling,
  otherwise it rejects the payment. This is to mitigate an attack where a
  malicious counterparty uses an old commitment transaction to spend the base
  funding output, therefore blocking the honest party from using its initiator
  virtual transaction. This check ensures that the coins gained by the
  punishment are sufficient to cover the losses from the blocked initiator
  transaction. If the attack takes place, other local channels based directly or
  indirectly on it are informed and they moved to a failed state. Note that this
  does not bring a risk of losing any of the total coins of all local channels.
  We conjecture that this balance constraint can be lifted if the current
  Lightning-based payment method is replaced with an eltoo-based
  one~\cite{eltoo}.

  Subsequently each of the two parties builds the new commitment transaction of
  its counterparty, signs it and sends over the signature, then the revocation
  transactions for the previously valid commitment transactions are generated,
  signed and the signatures are exchanged. To reduce the number of messages, the
  payee sends the two signatures in one message. This does not put it at risk of
  losing funds, since the new commitment transaction (for which it has already
  received a signature and therefore can spend) gives it more funds than the
  previous one.

  \pchan also monitors the chain for outdated commitment transactions by the
  counterparty and publishes the corresponding revocation transaction in case
  one is found (Figure~\ref{code:ln:poll}). It also monitors whether the party
  is activated often enough and marks it as negligent otherwise
  (Figure~\ref{code:ln:init}). The need for explicit negligence marking stems
  from the fact that party activation is entirely controlled by \environment,
  therefore it can happen that an otherwise honest party is not activated in
  time to prevent a malicious counterparty from successfully using an old
  commitment transaction. Therefore at the beginning of every activation while
  the channel is open, \textsc{ln} checks if the party has been activated within
  the last $p$ blocks (where $p$ is an implementation-dependent global constant
  \TODO{decide if reference to Proposition is needed}). If a party is marked as
  negligent, no balance security guarantees are given (c.f.
  Lemma~\ref{lemma:real-balance-security}). Note that this does not affect
  indistinguishability with the ideal world, as \fchan is notified by our
  \simulator if a party becomes negligent and does not perform the balance
  security check.

  When either party is instructed by \environment to close the channel
  (Figure~\ref{code:ln:close}), it first asks its host to close (details on the
  exact steps are discussed later) and once that is done, the ledger is checked
  for any transaction spending the funding output. In case the latest remote
  commitment tx is on-chain, then the channel is already closed and no further
  action is necessary. If an old committment transaction is on-chain, the
  corresponding revocation transaction is used for punishment. If the funding
  output is still unspent, the party attempts to publish the latest commitment
  transaction after waiting for any relevant timelock to expire. Until the
  funding output is irrevocably spent, the party still has to periodically check
  the blockchain and again be ready to use a revocation transaction if an old
  commitment transaction spends the funding output after all
  (Figure~\ref{code:ln:poll}).

  \TODO{\textsc{virt}}
