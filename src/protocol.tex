\begin{figure}[H]
  \begin{protocolbox}{$\Pi_{\mathrm{Chan}}$}
    \begin{algorithmic}[1]
      \State On (\textsc{init}, \texttt{out\_keys}) by \environment:
      \Indent
        \State ensure $\mathit{State} = \bot$
        \State $(c_A, c_B) \gets (0, 0)$
        \State $\texttt{virtuals} \gets \emptyset$
        \State ensure \textsc{pcn.init}(\texttt{keys}, \alice) returns
        (\textsc{ok})
        \State $\mathit{State} \gets \textsc{init}$
      \EndIndent
      \Statex

      \State On \textsc{top up} by \environment, act like \fchan
      (Fig.~\ref{code:functionality:chan:skeleton:init},
      lines~\ref{code:functionality:chan:skeleton:init:top_up:start}-\ref{code:functionality:chan:skeleton:init:top_up:end})
      \Statex

      \State On (\textsc{open base}) by \environment:
      \TODO{\textsc{ln.openBase}: \texttt{keys} = $pk_{A, out}$, $pk_{B, out}$}
      \Indent
        \State ensure $\mathit{State} = \textsc{topped up}$
        \State ensure \textsc{pcn.openBase}(\texttt{keys}, \texttt{fundee})
        returns (\textsc{ok}, $c$)
        \State $c_A \gets c$; $c_B \gets 0$
        \State $\mathit{State} \gets \textsc{open base}$
        \State output (\textsc{open base success}) to \environment
      \EndIndent
      \Statex

      \State On (\textsc{pay}, $x$) by \environment:
      \Indent
        \State ensure $\mathit{State} \in \{\textsc{open base}, \textsc{open
        virtual}\}$
        \State ensure $c_A - \mathrm{locked}(A) \geq x$
        \State ensure \textsc{pcn.pay}(x) returns (\textsc{ok})
        \State $c_A \gets c_A - x$; $c_B \gets c_B + x$
        \State output (\textsc{pay success}) to \environment
      \EndIndent
      \Statex

      \State On (\textsc{balance}) by \environment, act like \fchan
      (Fig.~\ref{code:functionality:chan:skeleton:base},
      lines~\ref{code:functionality:chan:skeleton:base:balance:start}-\ref{code:functionality:chan:skeleton:base:balance:end})
      \Statex

      \State On (\textsc{close}) by \environment, act like \fchan
      (Fig.~\ref{code:functionality:chan:skeleton:base},
      lines~\ref{code:functionality:chan:skeleton:close:start}-\ref{code:functionality:chan:skeleton:close:end}):
    \end{algorithmic}
  \end{protocolbox}
  \caption{}
  \label{code:protocol:chan:skeleton}
\end{figure}

\begin{figure}[H]
  \begin{protocolbox}{$\Pi_{\mathrm{Chan}}$ -- virtual}
    \begin{algorithmic}[1]
      \State \Comment{notification to fundee}
      \State \Comment{trust that \charlie has $c$ in her channel}
      \State On input (\textsc{open virtual}, $c$, \bob, \texttt{host\_bob},
      \texttt{keys}) by \charlie:
      \label{code:protocol:chan:skeleton:vchan:open-virtual}
      \Indent
        \State ensure $\mathit{State} = \textsc{init}$
        \State ensure \textsc{pcn.openVirt}(keys, \bob, \texttt{host\_bob})
        returns (\textsc{ok}, $c$)
        \label{code:protocol:chan:skeleton:vchan:ln}
        \State $\mathtt{host\_alice} \gets \charlie$
        \State $c_A \gets c$; $c_B \gets 0$
        \State from now on, handle any (\textsc{relayed}, $m$) input by
        \texttt{host\_alice} as the input ($m$) by \environment
        \State from now on, transform any output ($m$) to \environment to
        output (\textsc{relay}, $m$) to \texttt{host\_alice}
        \State $\mathit{State} \gets \textsc{open virtual}$
        \State output (\textsc{open virtual success}) to \charlie
        \label{code:protocol:chan:skeleton:vchan:open-virtual:output}
      \EndIndent
      \Statex

      \State On (\textsc{fund}, $c$, hops, \texttt{inner\_parties} =
      (\texttt{intter\_fundee}, \texttt{inner\_peer}), \texttt{outer\_parties} =
      (\texttt{outer\_funder}, \texttt{outer\_peer}), \texttt{keys}) by
      \environment:
      \TODO{\texttt{keys} = $pk_{\mathit{VA}, out}$, $pk_{\mathit{VB}, out}$}
      \Indent
        \State ensure $\mathit{State} \in \{\textsc{open base}, \textsc{open
        virtual}\}$
        \State ensure $c_A - \mathrm{locked}(A) \geq c$
        \State do the same as in \fchan,
        Fig.~\ref{code:functionality:chan:skeleton:virtual},
        lines~\ref{code:functionality:chan:skeleton:virtual:fund:start}-\ref{code:functionality:chan:skeleton:virtual:fund:end},
        skipping line~\ref{code:functionality:chan:skeleton:virtual:fund:id} and
        replacing ``to \alice'' with ``to \environment''
        \Comment{``as \alice'' sender labels are applied anyway, since we
        \emph{are} \alice}
        \label{code:protocol:chan:skeleton:vchan}
      \EndIndent
      \Statex

      \State On (\textsc{relay}, $m$, \charlie) by \environment:
      \Indent
        \State do the same as in \fchan,
        Fig.~\ref{code:functionality:chan:skeleton:virtual},
        lines~\ref{code:functionality:chan:skeleton:virtual:relay:input:start}-\ref{code:functionality:chan:skeleton:virtual:relay:input:end}
      \EndIndent
      \Statex

      \State On output (\textsc{relay}, $m$) by \charlie:
      \Indent
        \State do the same as in \fchan,
        Fig.~\ref{code:functionality:chan:skeleton:virtual},
        lines~\ref{code:functionality:chan:skeleton:virtual:relay:output:start}-\ref{code:functionality:chan:skeleton:virtual:relay:output:end}
      \EndIndent
      \TODO{check that everything done in ideal wrt closing is also done here}
    \end{algorithmic}
  \end{protocolbox}
  \caption{}
  \label{code:protocol:chan:skeleton:virtual}
\end{figure}
