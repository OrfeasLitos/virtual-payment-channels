\begin{theorem}[Recursive Virtual Payment Channel Security]
  The protocol $\textsc{LN}$ realises $\fchan$ given a global functionality
  $\ledger$ and assuming the security of the underlying digital signature.
  Specifically,
  \begin{gather*}
    \forall \text{ PPT } \adversary, \exists \text{ PPT } \simulator: \forall
    \text{ PPT } \environment \text{ it is }
    \textsc{exec}^{\ledger}_{\textsc{LN}, \adversary, \environment} \approx
    \textsc{exec}^{\fchan, \ledger}_{\simulator, \environment}
  \end{gather*}
\end{theorem}

\begin{proof}
  By inspection of Figs.~\ref{code:functionality:rules}
  and~\ref{code:simulator:flow} we can deduce that for a particular
  \environment, in the ideal world execution $\textsc{exec}^{\fchan,
  \ledger}_{\simulator_{\adversary}, \environment}$, $\simulator_{\adversary}$
  simulates internally the two \textsc{ln} parties exactly as they would execute
  in the real world execution, $\textsc{exec}^{\ledger}_{\textsc{LN},
  \adversary, \environment}$ in case \fchan does not halt. Indeed, \fchan only
  halts with negligible probability according to Lemma~\ref{lemma:no-halt},
  therefore the two executions are computationally indistinguishable.
\end{proof}
