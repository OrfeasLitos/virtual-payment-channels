\section{Conclusion}

  In this work we presented Recursive Virtual Payment Channels for Bitcoin,
  a construction which enables the establishment of pairwise payment channels without the need for
  posting on-chain transactions. Such a channel can be opened over a path of consecutive base
  channels of arbitrary length, i.e., the virtual channel constructor is variadic. 

  The base channels themselves
  can be virtual, therefore the novel recursive nature of the construction. 
  A key performance characteristic of our construction is that it has optimal
  round complexity for channel closing: a single transaction is required
  by any participant, be it an end-point or an intermediary. 
  
  We formally described the protocol in the UC setting, provided a corresponding
  ideal functionality and simulator and finally proved the indistinguishability
  of the protocol and functionality, along with the balance security property
  that ensures no loss of funds for honest, non-negligent parties. This is
  achieved through the use of the \texttt{ANYPREVOUT} sighash flag, which is a
  proposed feature for Bitcoin, also required by the eltoo improvement to lightning, 
  \cite{eltoo}. 

  We also proved that any construction as performant as ours
  will require from channel intermediaries to be capable of accessing 
  an exponential number of different transactions in the number of base channels,
  unless a sighash flag such as \texttt{ANYPREVOUT} is available. Barring any other
  state compression technique that manages to compress this exponentially large
  set of transactions into a polynomial size private state, our
 work serves as further evidence for the usefulness of
  including this flag into the Bitcoin protocol. 

\newpage