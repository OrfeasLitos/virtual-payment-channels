\section{Future work}
  A number of features can be added to our protocol for additional efficiency,
  usability and flexibility. First of all, a new subprotocol for cooperatively
  closing a virtual channel can be created. In the optimistic case, a virtual
  channel would then be closed with no on-chain transactions and its base
  channels would become independent once again. To achieve this goal,
  cooperation is needed between all base parties of the virtual channel and
  possibly parties implicated in other virtual channels that use the same base
  channels.

  In our current construction, each time a particular channel $C$ acts as a base
  channel for a new virtual, one more ``virtualisation layer'' is added. When
  one of its owners wants to close $C$, it has to put on-chain as many
  transactions as there are virtualisation layers. Also the timelocks associated
  with closing a virtual channel increase with the number of virtualisation
  layers of its base channels. Both these issues can be alleviated by extending
  the opening subprotocol with the ability to cooperatively open multiple
  virtual channels in the same layer, either simultaneously or as an amendment
  to an existing virtualisation layer.

  Due to the possibility of the griefing attack discussed in
  Subection~\ref{construction:ln}, the range of balances a virtual channel can
  support is limited by the balances of neighbouring channels. We believe that
  this limitation can be lifted if instead of using a Lightning-based
  construction for the payment layer, we instead replace it with an
  eltoo-based~\cite{eltoo} construction. Since in eltoo a maliciously published
  old state can be simply re-spent by the honest latest state, the griefing
  attack is completely avoided. What is more, our protocol shares with eltoo the
  need for the \texttt{ANYPREVOUT} sighash flag, therefore no additional
  requirements from the Bitcoin protocol would be added by this change. Lastly,
  due to the separation of intermediate layers with the payment layer in our
  pseudocode implementation as found in the Appendix (i.e. the distinction
  between the \textsc{ln} and the \textsc{virt} protocols), this change should
  in principle not need extensive changes in all parts of the protocol.

  As it currently stands, the timelocks calculated for the virtual channels are
  based on $p$ (Figure~\ref{code:ln:init}) and $s$
  (Figure~\ref{code:ln:exchange-open-sigs}),
  which are global constants that are immutable and common to all parties. $s$
  stems from the liveness guarantees of Bitcoin, as discussed in
  Proposition~\ref{prop:liveness} and therefore cannot be tweaked. However, $p$
  represents the maximum time between two activations of a non-negligent party,
  so in principle it is possible for the parties to explicitly negotiate this
  value when opening a new channel and even renegotiate it after the channel has
  been opened if need be. We leave this usability-augmenting protocol feature as
  future work.

  As we mentioned earlier, our protocol is not designed to gracefully recover
  from a situation in which halfway through a subprotocol, one of the
  counterparties starts misbehaving. Currently the only solution is to
  unilaterally close the channel. This however means that DoS attacks (that
  still do not lead to financial losses) are possible. A practical
  implementation of our protocol would need to expand the available actions and
  states to be able to transparently and gracefully recover from such problems,
  avoiding closing the channel where possible, especially when the problem stems
  from network issues and not from malicious behaviour.

  Furthermore, a realistic implementation has to explicitly handle the issue of
  fees. These include miner fees for on-chain transactions and intermediary fees
  for the parties that own base channels and facilitate opening virtual
  channels. To incorporate these, an incentive analysis has to be conducted and
  subsequently the protocol has to be augmented with the necessary data and
  corresponding messages.

  In order to increase readability and to keep focus on the salient points of
  the construction, our protocol does not exploit a number of possible
  optimisations. These include a number of techniques employed in Lightning that
  drastically reduce storage requirements, along with a variety of possible
  improvements to our novel virtual subprotocol. Most notably, the
  Taproot~\cite{taproot} update that is planned for Bitcoin will allow for a
  drastic reduction in the size of transactions, as in the optimistic case only
  the hash of the Script has to be added to the blockchain and the $n$
  signatures needed to spend a virtual output can be replaced with their
  aggregate, which has constant size. As this work is mainly a proof of
  feasibility, we leave these optimisations as future work.

  Additionally, our protocol does not enable a one-off multi-hop payment without
  setting up a virtual channel first. This however is a useful feature in case
  two parties know that they will only transact once, as opening a virtual
  channel needs substantially more network messages than performing an one-of
  multi-hop payment. It would be therefore fruitful to also enable the multi-hop
  payment technique used in Lightning and allow users to choose which method to
  use in each case.

  \TODO{@Aggelos: check next}
  Moreover, the result of Theorem~\ref{theorem:anyprevout} excludes a large
  class of variadic recursive protocols that do not make use of
  \texttt{ANYPREVOUT} from achieving practical performance, but it does not
  preclude the existence of such protocols entirely. In particular, there may be
  some as of yet unknown (sufficiently secure) optimisation that allows parties
  to generate only the transactions that they need to put on-chain during the
  closing procedure, using only a small quantity of data that has been received
  when opening. This would permit parties to circumvent the need for exchanging
  and storing an exponential number of signatures and transactions. Further
  investigation for the existence of such an optimisation is left as future
  work.

  Last but not least, the current analysis gives no privacy guarantees for the
  protocol, as it does not employ onion packets~\cite{sphinx} like Lightning.
  Furthermore, \fchan leaks all messages to the ideal adversary therefore
  theoretically to privacy is offered at all. Nevertheless, onion packets can be
  incorporated in the current construction and intuitively our construction
  leaks less data than Lightning for the same multi-hop payments, as
  intermediaries in our case do not learn the new balance after every payment,
  contrary to Lightning. Therefore a future extension can improve the privacy of
  the construction and formally prove exact privacy guarantees.
