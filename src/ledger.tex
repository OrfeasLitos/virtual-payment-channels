\section{The Ledger and Clock Functionalities}
  \label{sec:ledger}
  We next provide the complete description of the ledger functionality as well as the clock and network functionalities  that are
drawn from the   UC formalisation of~\cite{BMTZ17,genesis}.

  The key characteristics of the functionality are as follows. The variable
  \texttt{state} maintains the current immutable state of the ledger. An honest,
  synchronised party considers finalised a prefix of \texttt{state} (specified by a
  pointer position $\pt_{i}$ for party $\stkh_i$ below). The functionality has a
  parameter \windowSize such that no finalised prefix of any player will be
  shorter than $|\texttt{state}| - \windowSize$. On any input originating from an honest
  party the functionality will run the \LFextend function that ensures that a
  suitable sequence of transactions will be ``blockified'' and added to \texttt{state}.
  Honest parties may also find themselves in a desynchronised state: this is
  when honest parties lose access to some of their resources. The resources that
  are necessary for proper ledger maintenance and that the functionality keeps
  track of are the global random oracle \Foracle and the clock \Fclock. If an honest party maintains registration with all the resources
  then after \vdelay clock ticks it necessarily becomes synchronised.

  The progress of the \texttt{state} variable is guaranteed via the \LFextend function
  that is executed when honest parties submit inputs to the functionality. While
  we do not specify \LFextend in our paper (we refer to the citations above for
  the full specification) it is sufficient to note that \LFextend guarantees the
  following properties:
  \begin{enumerate}
    \item in a period of time equal to \maxTime, a number of blocks at least
    \windowSize are
    added to \texttt{state}.
    \item in a period of time equal to $\minTime$, no more blocks may be added to $\texttt{state}$ if $\windowSize$ blocks have been already added. 
    \item each window of \windowSize blocks has at most \advBlocksinWindowSize\/
    adversarial blocks included in it.
    \item any transaction that (i) is submitted by an honest party earlier than
    $\frac{\vdelay}{2}$ rounds before the time that the block that is
    \windowSize positions before the head of the \texttt{state} was included, 
    and (ii) is valid with respect to an honest block that extends $\texttt{state}$,
    then it must be included in such block. 
  \end{enumerate}

  Given a synchronised honest party, we say that a transaction \tx is
  finalised when it becomes a part of \texttt{state} in its view.

  \begin{systembox}{$\Fledger$}\vspace{1ex}
    \small
    {\bf General:}
    The functionality is parameterized by four algorithms, \LFvalidate,
    \LFextend, \LFblockify, and \vsync, along with three parameters:
    $\windowSize, \vdelay\in\mathbb{N}$, and $\StakeHolderSet :=
    \{(\stkh_{1},\stake_{1}),\ldots,(\stkh_{n},\stake_{n})\}$.
    %
    The functionality manages variables $\texttt{state}$ (the immutable state of the
    ledger), $\cbuffer$ (a list of transaction identifiers to be added to the
    ledger), \buffer (the set of pending transactions), $\LFtau$ (the rules
    under which the state is extended), and \tlast (the time sequence where all
    immutable blocks where added). The variables are initialized as follows:
    $\texttt{state}:=\tlast:= \cbuffer:= \varepsilon$, $\buffer:=\emptyset$, $\LFtau=0$.
  %
    For each party $\party\in\PS$ the functionality maintains a pointer $\pt_i$
    (initially set to 1) and a current-state view $\texttt{state}_p:=\varepsilon$
    (initially set to empty). The functionality also keeps track of the  timed
    honest-input sequence in a vector $\TIH$ (initially $\TIH:=\varepsilon$).
    \medskip

    {\bf Party Management:}
    The functionality maintains the set of registered parties $\PS$, the
    (sub-)set of honest parties $\HO \subseteq \PS$, and the (sub-set) of
    de-synchronized honest parties $\DSyncPS\subset\HO$ (as discussed below).
    The sets $\PS, \HS, \DSyncPS$ are all initially set to $\emptyset$. When a
    (currently unregistered) honest party is registered at the ledger, {\em if
    it is registered with the clock and the global RO already,} then it is added
    to the party sets $\HS$ and $\PS$ and the current time of registration is
    also recorded; if the current time is $\LFtau > 0$, it is also added to
    $\DSyncPS$. Similarly, when a party is deregistered, it is removed from both
    $\PS$ (and therefore also from $\DSyncPS$ or $\HS$). The ledger maintains
    the invariant that it is registered (as a functionality) to the clock
    whenever $\HS \neq \emptyset$.

    \medskip

    {\bf Handling initial stakeholders:}
    If during round $\tau = 0$, the ledger did not received a registration from
    each initial stakeholder, i.e., $\party \in \StakeHolderSet$, the
    functionality halts.

    \medskip
    \hrule
    \medskip

    %\emph{To be executed upon any activation:}
    {\bf Upon receiving any input $I$} from any party or from the adversary,
    send $(\ClockRead, \sidClock)$ to \Fclock and upon receiving response
    $(\ClockRead, \sidClock, \current)$ set $\LFtau:=\current$ and do the
    following if $\current > 0$ (otherwise, ignore input):

    \begin{enumerate}\setlength\itemsep{1.5ex}
      \item Updating synchronized/desynchronized party set:
      \begin{enumerate}
        \item Let $\hat{\PS}\subseteq\DSyncPS$  denote the set of desynchronized
        honest parties that have been registered (continuously) to the ledger,
        the clock, and the GRO since time $\tau' < \LFtau-\vdelay$. Set
        $\DSyncPS:=\DSyncPS\setminus\hat{\PS}$.
        \item For any synchronized party $\party\in\HO\setminus\DSyncPS$, if
        $\party$ is not registered to the clock, then consider it
        desynchronized, i.e., set $\DSyncPS\cup\{\party\}$.
      \end{enumerate}
      \item If $I$ was received from an honest party $\party\in\PS$:\\[1ex]
      \begin{enumerate}\setlength\itemsep{1ex}
        \item Set $\TIH:=\TIH||(I,\party,\LFtau)$;
        \item Compute
        $\vec{N}=(\vec{N}_1,\ldots,\vec{N}_\ell):=\LFextend(\TIH,\texttt{state},\cbuffer,\buffer,\tlast)$
        and if $\vec{N}\neq\varepsilon$ set
        $\texttt{state}:=\texttt{state}||\LFblockify(\vec{N}_1)||\ldots||\LFblockify(\vec{N}_\ell)$
        and $ \tlast:=\tlast||\LFtau^\ell$, where
        $\LFtau^\ell=\LFtau||\ldots,||\LFtau$.
        \item For each $\BTX\in\buffer$: if $\LFvalidate(\BTX,\texttt{state},\buffer)=0$
        then delete  $\BTX$ from $\buffer$. Also, reset $\cbuffer:=
        \varepsilon$.
        \item If there exists $\stkh_{j}\in\HS\setminus\DSyncPS$ such that
        $|\texttt{state}| -\pt_{j} > \windowSize$ or $\pt_{j} < |\texttt{state}_{j}|$, then set
        $\pt_{k}:=|\texttt{state}|$ for all $\stkh_{k}\in\HS\setminus\DSyncPS$.
      \end{enumerate}

      \item If the calling party $\party$ is \emph{stalled or time-unaware}
      (according to the defined party classification), then no further actions
      are taken. Otherwise, depending on the above input $I$ and its sender's
      ID, $\Fledger$ executes the corresponding code from the following list:
      \begin{itemize}
        \let\labelitemi\labelitemii
        \item \emph{Submitting a transaction:}\\
        If $I=(\LFsubmit, \sid,\tx)$ and is received from a party $\party \in
        \PS$ or from \Adv (on behalf of a corrupted party $\party$) do the
        following%\\[0.5ex]
        \begin{enumerate}\setlength\itemsep{1ex}
          \item Choose a unique transaction ID $\txid$ and set
          $\BTX:=(\tx,\txid,\LFtau,\party)$
          \item If $\LFvalidate(\BTX,\texttt{state},\buffer)=1$, then
          $\buffer:=\buffer\cup\{\BTX\}$.
          \item Send $(\LFsubmit, \BTX)$ to \Adv.%\\[2ex]
        \end{enumerate}

        \item \emph{Reading the state:}\\ If $I=(\LFread, \sid)$ is received
        from a party $\party \in \PS$ then set $\texttt{state}_p:=
        \texttt{state}|_{\min\{\pt_p,|\texttt{state}|\}}$ and return $(\LFread, \sid, \texttt{state}_p)$
        to the requester. If the requester is \Adv then send
        $(\texttt{state},\buffer,\TIH)$ to~\Adv.%\\[2ex]

        \item \emph{Maintaining the ledger state:}\\
        If $I=(\LFmine,\sid, \minerID)$ is received by an honest party
        $\party\in\PS$ and  (after updating \TIH  as above)
        $\vsync(\TIH)=\hat{\tau}>\LFtau$ then send $(\ClockUp, \sidClock)$ to
        \Fclock. Else send $I$ to \Adv.%\\[2ex]


        \item \emph{The adversary proposing the next block:} \\
        If $I=(\LFmkcore, \honestyFlag, (\txid_1,\ldots,\txid_\ell))$ is sent
        from the adversary, update \cbuffer as follows:%\\[1ex]
        \begin{enumerate}\setlength\itemsep{1ex}
          \item Set $\mathrm{listOfTxid} \gets \epsilon$
          \item For $i=1,\ldots, \ell$ do: if there exists
          $\BTX:=(x,\txid,\minerID,\LFtau,\stkh_j)\in\buffer$ with ID
          $\txid=\txid_i$ then set
          $\mathrm{listOfTxid}:=\mathrm{listOfTxid}||\txid_i$.
          \item Finally, set $\cbuffer:= \cbuffer
          ||(\honestyFlag,\mathrm{listOfTxid})$ and output $(\LFmkcore, ok)$ to
          \Adv.%\\[2ex]
        \end{enumerate}
        \item \emph{The adversary setting state-slackness:}\\
        If $I=(\LFslack,(\stkh_{i_1},\hat{\pt}_{i_1}),\ldots,
        (\stkh_{i_\ell},\hat{\pt}_{i_\ell})),$ with
        $\{\party_{i_1},\ldots,\party_{i_\ell}\}\subseteq\HS\setminus\DSyncPS$
        is received from the adversary \Adv do the following:%\\[1ex]
        \begin{enumerate}\setlength\itemsep{1ex}
          \item If for all $j\in[\ell]:$  $|\texttt{state}|
          -\hat{\pt}_{i_j}\leq\windowSize$ and $\hat{\pt}_{i_j}\geq
          |\texttt{state}_{i_j}|$, set $\pt_{i_1}:=\hat{\pt}_{i_1}$ for every
          $j\in[\ell]$ and return $(\LFslack,ok)$ to \Adv.
          \item Otherwise set $\pt_{i_j}:=|\texttt{state}|$ for all $j\in[\ell]$.
        \end{enumerate}
        \item \emph{The adversary setting the state for desychronized
        parties:}\\
        If $I=(\LFdsstate,(\stkh_{i_1},\texttt{state}_{i_1}'),\ldots,
        (\stkh_{i_\ell},\texttt{state}_{i_\ell}')),$ with
        $\{\stkh_{i_1},\ldots,\stkh_{i_\ell}\}\subseteq\DSyncPS$ is received
        from the adversary \Adv, set $\texttt{state}_{i_j}:=\texttt{state}_{i_j}'$ for each
        $j\in[\ell]$ and return $(\LFdsstate,ok)$ to \Adv.
      \end{itemize}
    \end{enumerate}
  \end{systembox}

  \begin{systembox}{Functionality $\Fclock$}
    The functionality manages the set $\PS$ of registered identities, i.e.,
    parties $\party = (\pid,\sid)$. It also manages the set $F$ of
    functionalities (together with their session identifier). Initially,
    $\PS:=\emptyset$ and $F := \emptyset$.

    \smallskip
    For each session $\sid$ the clock maintains a variable $\current_\sid$. For
    each identity $\party := (\pid,\sid) \in \PS$ it manages variable
    $d_\party$. For each pair $(\Func,\sid) \in F$ it manages variable
    $d_{(\Func,\sid)}$ (all integer variables are initially $0$).

    \medskip

    \emph{Synchronization:}
    \begin{itemize}
      \item Upon receiving $(\ClockUp, \sidClock)$ from some party $\party \in
      \PS$ set $d_{\party}:=1$; execute \emph{Round-Update} and forward
      $(\ClockUp, \sidClock, \party)$  to $\Adv$.

      \item Upon receiving $(\ClockUp, \sidClock)$ from some functionality
      $\Func$ in a session $\sid$ such that $(\Func,\sid) \in F$ set
      $d_{(\Func,\sid)}:=1$, execute \emph{Round-Update} and return $(\ClockUp,
      \sidClock,\Func)$ to this instance of $\Func$.

      \item Upon receiving $(\ClockRead, \sidClock)$ from any participant
      (including the environment on behalf of a party, the adversary, or any
      ideal---shared or local---functionality) return $(\ClockRead,
      \sid,\current_\sid)$ to the requestor (where $\sid$ is the sid of the
      calling instance).
    \end{itemize}

    \emph{Procedure Round-Update:}
    For each session $\sid$ do:
    If $d_{(\Func,\sid)}:=1$ for all $\Func \in F$ and $d_{\party}=1$ for all
    honest parties $\party=(\cdot,\sid) \in \PS$, then set
    $\current_\sid:=\current_\sid+1$  and reset $d_{(\Func,\sid)}:=0$ and
    $d_{\party}:=0$ for all parties $\party=(\cdot,\sid) \in \PS$.
  \end{systembox}
