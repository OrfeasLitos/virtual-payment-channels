\section{On the necessity of \texttt{ANYPREVOUT}}
  \label{section:anyprevout}
  As our protocol relies on the \texttt{ANYPREVOUT} sighash flag, it cannot be
  deployed on Bitcoin until it is introduced. We here argue that any efficient
  protocol that achieves goals similar to ours and has parties maintain 
  Bitcoin transactions in their local state requires  the proposed sighash flag. 
  
  \begin{theorem}[\texttt{ANYPREVOUT} is necessary]
    \label{theorem:anyprevout}
    Consider $n$ independent ordered off-chain ``base'' protocols over Bitcoin
    (i.e. generalisations of pairwise channels to more than $2$ participants)
    such that every pair of consecutive protocols $(\Pi_{i-1}, \Pi_i)$ for $i
    \in \{2, \dots, n-1\}$ has a common party $P_i$. Also consider a protocol
    that establishes a virtual channel (i.e. a payment channel without any on-chain
    txs when opening) between two parties $P_1, P_n$ that take part in the first
    and last off-chain protocols respectively. If this protocol guarantees that
    each honest protocol party (both endpoints and intermediaries) needs to put
    at most $O(1)$ transactions on-chain for unilateral closure and needs to
    have at most a subexponential (in $n$) number of transactions  available off-chain,
    then the protocol needs the \texttt{ANYPREVOUT} sighash flag.
  \end{theorem}

  \begin{proof}[Proof of Theorem~\ref{theorem:anyprevout}]
    When an off-chain protocol is closed, there has to be some form of
    information and coin flow to its %$2$ or $2$
    neighbouring protocols in order
    to ensure that the virtual channel will be funded exactly once if at least
    one of its participants is honest and that no honest intermediary will be
    charged. Such flow can happen either with simultaneous closures (e.g. our
    initiator txs) or with special outputs that will be consumed when neighbours
    close (e.g. our virtual outputs). There is no other possible manner of
    on-chain enforceable information and coin flow that is compatible with the
    theorem requirements. This includes adaptor
    signatures~\cite{cryptoeprint:2020:476}, as they facilitate coin exchange
    only if the parties and all base protocols for this particular virtual
    channel were known when the off-chain protocols were opened (contradicting
    off-chain protocol independence) or if new on-chain transactions are
    introduced when opening the virtual channel (contradicting off-chain
    opening).

    Therefore each party must have different transactions available to close its
    off-chain protocol(s), each corresponding to a different order of actions
    taken by participants of other off-chain protocols. This is true because if
    a party could close its protocol in an identical way whether one of its
    neighbouring protocols had already closed or not, it would then fail to make
    use of and possibly propagate to the other side the relevant coins and
    information. We will now prove by induction in the number $m = n - 1$ of
    base protocols that the number of these transactions $T_m$ is exponential if
    \texttt{ANYPREVOUT} is not available, by calculating a lower bound, specifically,
    that $T_m \geq 2^{m-1}$. 

    If $m = 2$, then there is a single intermediary $P_2$. It needs at least $2$
    different transactions: one if it moves first and one if it moves second,
    after a member in the off-chain protocol to its right, e.g. $P_3$.
    From this it follows immediately that $T_2 \geq 2$. 

    If $m = k > 2$, then assume that $P_2$ needs to have $f\geq 2^{m-1}$ transactions
    available to be able to unilaterally close its protocols in all scenarios in
    which all parties $P_i$ for $i \in \{3, \dots, k+1\}$ act before $P_2$. Each
    of those transactions corresponds to one or more orderings of the closing
    actions of the parties of the other base protocols. No two transactions
    correspond to the same ordering.

    For the induction step, consider a virtual channel over $m = k + 1$ base
    protocols. $P_2$ would still need $f$ different transactions, each
    corresponding to the same orderings of parties' actions as in the induction
    hypothesis. These transactions are possibly different to the ones they
    correspond to in the case of the induction hypothesis, but their total
    number is the same. For each of these orderings we produce two new
    orderings: one in which the new party $P_{k+2}$ acts right before and one in
    which it acts right after $P_{k+1}$.  Given such an ordering $o$, consider the
    neighbor relation between the set of parties that have been activated 
    and take its reflexive and transitive closure $\sim_o$. 
    Now consider any party $P_i$ with the following properties: 
    (i) it acts after $P_{k+2}$ and $P_{k+1}$ (e.g., $P_2$ is such a party), and
    (ii)  at least one of its neighbours belongs to the equivalence class of 
    $\sim_o$ that contains $P_{k+1}$. Observe that such party $P_i$ is always
    well defined. 
    %
    Since $P_{k+1}$ must necessarily use a different transaction for each of the
    two orderings with $P_{k+2}$, and since there is a continuous chain of
    parties between $P_{k+1}$ and $P_i$ that have already acted, it is the case
    that $P_i$ must have a different transaction for each of these two cases as
    well, as without \texttt{ANYPREVOUT}, an input of a transaction can only
    spend a specific output of a specific transaction.  Finally, 
    given that $P_2$ will have to act in response to at least as many of the above
    options, we deduce that $P_2$ needs to have at
    least $2f\geq 2^{m}$ transactions available. This completes the induction step.

    As a result, we conclude that party $P_2$ needs at least
    $2^{m-1} \in O(2^n)$ transactions to be
    able to unilaterally close its protocol.  
  \end{proof}

  Note that in case of a protocol that resembles ours but does not make use of
  \texttt{ANYPREVOUT}, the situation is further complicated in two distinct
  ways: First, virtual channel parties would have to generate and sign an at
  least exponential number of new commitment transactions on each update, one
  for each possible virtual output, therefore making virtual channel payments
  unrealistic. Second, if one of the base channels of a virtual channel is
  itself virtual, then the new channel needs a different set of virtual
  transactions for each of the (exponentially many) possible funding outputs of
  the base virtual channel, thus further compounding the issue.
 
