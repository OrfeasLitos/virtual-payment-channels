\section{Security}
  \label{section:security}
  Elmo provides concrete security
  properties regarding the conservation of funds as described in
  Lemma~\ref{lemma:real-balance-security-informal}. Informally, it establishes
  that if an honest, non-negligent party was implicated in a channel that it
  then unilaterally closes, then the party will have at least the expected funds
  on-chain. The formal statements
  (Lemmas~\ref{lemma:real-balance-security},~\ref{lemma:ideal-balance},
  and~\ref{lemma:no-halt}) along with all proofs are deferred to
  Appx.~\ref{sec:proofs}.

\begin{lemma}[Real world balance security (informal)]
\label{lemma:real-balance-security-informal}
  Consider a real world execution with $P \in \{\alice, \bob\}$ honest,
  non-negligent \textsc{ln} ITI. Assume that all of the following are true:
  \begin{itemize}
    \item $P$ opened the channel, with initial balance $c$,
    \item $P$ is the host of $n$ channels, each funded with $f_i$ coins,
    \item $P$ has cooperatively closed $k$ channels, where the $i$-th channel
    transferred $r_i$ coins from the hosted virtual channel to $P$,
    \item $P$ has sent $m$ payments, each involving $d_i$ coins,
    \item $P$ has received $l$ payments, each involving $e_i$ coins.
  \end{itemize}
  If $P$ closes unilaterally, eventually there will be $h$ outputs on-chain
  spendable only by $P$ or a kindred party, each of value $c_i$, such that
  \begin{equation}
    \sum\limits_{i=1}^h c_i \geq c - \sum\limits_{i=1}^n f_i -
    \sum\limits_{i=1}^m d_i + \sum\limits_{i=1}^l e_i + \sum\limits_{i=1}^k r_i
    \enspace.
  \end{equation}
\end{lemma}

  The expected funds are [initial balance - funds for hosted
  virtuals + funds returned from hosted virtuals - outbound payments + inbound
  payments]. Note that the funds for hosted virtuals only refer to those funds
  used by the funder of the virtual channel, not the rest of the base parties.
  The proof follows all possible execution paths, keeping track of the
  resulting balance in each case.
  %and coming to the conclusion that balance is
  %secure in all cases, except if signatures are forged.

  %This is proven by first arguing that if
  %the conditions of Lemma~\ref{lemma:ideal-balance} for the ideal world hold,
  %then the conditions of Lemma~\ref{lemma:real-balance-security} also hold for
  %the equivalent real world execution, therefore in this case \fchan does not
  %halt. We then argue that also in case the conditions of
  %Lemma~\ref{lemma:ideal-balance} do not hold, \fchan may never halt as well,
  %therefore concluding the proof.

  Complementarily to the direct security properties above, in this work we
  embrace the Universal Composition (UC) framework~\cite{uc}
  together with its global subroutines extension,
  UCGS~\cite{DBLP:conf/tcc/BadertscherCHTZ20}, to
  model parties, network interactions, adversarial influence and corruptions, as
  well as formalise and prove security.
  A salient observation
  regarding our UC protocol \pchan (Appx.~\ref{construction:real-world}) is
  that, in order to
  open a virtual channel, it passes inputs to another \pchan instance that
  belongs to a different extended session, therefore \pchan is not
  \emph{subroutine respecting}, as defined in~\cite{uc}. To
  address this, we first add a superscript to \pchan, i.e.,
  $\pchansup{n}$. $\pchansup{1}$ is always a simple channel.
  This is done by ignoring instructions to \textsc{open} on top of other
  channels. As for higher superscripts, $\forall n \in
  \mathbb{N}^*, \pchansup{n+1}$ is the same as \pchan but with
  base channels of a maximum superscript $n$. It then holds that $\forall
  n \in \mathbb{N}^*, \pchansup{n}$ is $(\ledger, \pchansup{1}, \dots,
  \pchansup{n-1})$-subroutine respecting, as defined
  in~\cite{DBLP:conf/tcc/BadertscherCHTZ20}. The same superscript trick is done
  to \fchan. To
  the best of the authors' knowledge, this recursion-based proof technique for
  UC security is novel. It is of independent interest and can be reused to prove
  UC security in protocols that may use copies of themselves as subroutines.
  Theorems~\ref{theorem:security:simple} and~\ref{theorem:security:virtual}
  (Appx.~\ref{sec:proofs}) state
  that $\forall n \geq 1, \pchansup{n}$ UC-realises $\fchansup{n}$.
