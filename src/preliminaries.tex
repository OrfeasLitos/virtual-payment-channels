\section{Preliminaries \& Notation}
  In this work we embrace the Universal Composition (UC) framework~\cite{uc} to
  model parties, network interactions, adversarial influence and corruptions, as
  well as formalise and prove security.

  UC closely follows and expands upon the simulation-based security
  paradigm~\cite{DBLP:books/sp/17/Lindell17}. For a particular real world
  protocol, the main goal of UC is allow us to provide a simple ``interface'',
  the ideal world functionality, that describes what the protocol achieves in an
  ideal way. The functionality takes the inputs of all protocol parties and
  knows which parties are corrupted, therefore it normally can achieve the
  intention of the protocol in a much more straightforward manner. At a high
  level, once we have the protocol and the functionality defined, our goal is to
  prove that no probabilistic polynomial-time (PPT) ITM can distinguish whether
  it is interacting with the real world protocol or the ideal world
  functionality. If this is true we then say that the protocol UC-realises the
  functionality.

  The principal contribution of UC is the following: Once a functionality that
  corresponds to a particular protocol is found, any other higher level protocol
  that internally uses the former protocol can instead use the functionality.
  This allows cryptographic proofs to compose and obviates the need for
  re-proving the security of every underlying primitive in every new application
  that uses it, therefore vastly improving the efficiency and scalability of the
  effort of cryptographic proofs.

  In UC, a number of interactive Turing Machines (ITMs) execute and send
  messages to each other. At each moment only one ITM is executing (has the
  ``execution token'') and when it sends a message to another ITM, it transfers
  the execution token to the receiver. Messages can be sent either locally
  (inputs, outputs) or over the network.

  The first ITM to be activated is the environment \environment. This can be any
  PPT ITM. This ITM encompasses everything that happens around the protocol
  under scrutiny, including the players that send instructions to the protocol.
  It also is the ITM that tries to distinguish whether it is in the real or the
  ideal world. Put otherwise, it plays the role of the distinguisher.

  After activating and executing some code, \environment may input a message to
  any party. If this execution is in the real world, then each party is an ITM
  running the protocol \prot. Otherwise if the execution takes place in the
  ideal world, then each party is a dummy that simply relays messages to the
  functionality \func. An activated real world party then follows its code,
  which may instruct it to parse its input and send a message to another party
  via the network.

  In UC the network is fully controlled by the so called adversary \adversary,
  which may be any PPT ITM. Once activated by any network message, this machine
  can read the message contents and act adaptively, freely communicate with
  \environment bidirectionally, choose to deliver the message right away, delay
  its delivery arbitrarily long, even corrupt it or drop it entirely. Crucially,
  it can also choose to corrupt any protocol party (in other words, UC allows
  adaptive corruptions). Once a party is corrupted, its internal state, inputs,
  outputs and execution comes under the full control of \adversary for the rest
  of the execution. Corruptions take place covertly, so other parties do not
  necessarily learn which parties are corrupt. Furthermore, a corrupted party
  cannot become honest again.

  The fact that \adversary controls the network in the real world is modelled by
  providing direct communication channels between \adversary and every other
  machine. This however poses an issue for the ideal world, as \func is a single
  party that replaces all real world parties, so the interface has to be adapted
  accordingly. Furthermore, if \func is to be as simple as possible, simulating
  internally all real world parties is not the way forward. This however may
  prove necessary in order to faithfully simulate the messages that the
  adversary expects to see in the real world. To solve these issues an ideal
  world adversary, also known as simulator \simulator, is introduced. This party
  can communicate freely with \func and completely engulfs the real world
  \adversary. It can therefore internally simulate real world parties and
  generate suitable messages so that \adversary remains oblivious to the fact
  that this is the ideal world. Normally it just relays messages between
  \adversary and \environment.

  From the point of view of the functionality, \simulator is untrusted,
  therefore any information that \func leaks to \simulator has to be carefully
  monitored by the designer. Ideally it has to be as little as possible so that
  \simulator does not learn more than what is needed to simulate the real world.
  This facilitates modelling privacy.

  At any point during one of its activations, \environment may return a binary
  value. The entire execution then halts. Informally, we say that \prot
  UC-realises \func, or equivalently that the ideal and the real worlds are
  indistinguishable, if $\forall \text{ PPT } \adversary, \exists \text{ PPT }
  \simulator: \forall \text{ PPT } \environment$, the distance of the
  distributions over the machines' random tapes of the outputs of \environment
  in the two worlds is negligibly small. Note the order of quantifiers:
  \simulator depends on \adversary, but not on \environment.
