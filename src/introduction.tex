\section{Introduction}
  The popularity of blockchains in recent years has stretched their
  performance to its limits. Due to their need for synchronisation their latency
  is large (e.g. Bitcoin has a latency of 1h~\cite{bitcoin}) and due to the need
  for massive redundancy their throughput is low (Bitcoin can handle at most 7
  transactions per second~\cite{scaling}). To circumvent these inherent
  limitations of blockchains, a prominent solution is to optimistically handle
  payments off-chain via a Payment Channel Network
  (PCN)~\cite{DBLP:conf/fc/GudgeonMRMG20} and only use the blockchain as an
  arbiter in case of dispute.

  The most popular PCN is the Lightning Network (LN)~\cite{lightning}, which
  works on top of Bitcoin. With this, parties can open a pairwise channel with a
  single on-chain transaction and subsequently pay each other an unlimited
  number of times, only limited by the speed of their internet connection. What
  is more, a party can pay another even if they do not have a direct channel.
  They can instead leverage a path of channels for a fee and perform a so-called
  multi-hop payment in an atomic manner. Unfortunately a multi-hop payment needs
  active cooperation by all intermediaries, therefore increasing the latency and
  the probability of failure of the payment.

  To mitigate this issue, virtual payment channels have been proposed
  \TODO{cite}. These enable two parties, say Alice and Bob, to open a payment
  channel over two preexisting channels, one between Alice and Charlie and
  another between Charlie and Bob. \TODO{check if recursive channels exist}

  However, due to the limited scripting language of Bitcoin, it has proved
  challenging to build a secure protocol that allows virtual channels to be
  opened over more than two underlying channels, \TODO{delete following phrase
  if the previous's TODO answer is affirmative} as well as to make this
  construction recursive in the sense that further virtual channels can be
  opened on top of other virtual channels.

  This work fills this gap by providing a concrete protocol that allows for
  arbitrarily many channels to be opened on top of arbitrarily long channel
  paths, where the underlying channels may themselves be virtual. This is
  achieved using standard Bitcoin script and an elaborate transaction
  configuration. We formally prove the security of the protocol in the
  UC~\cite{uc} setting. The construction relies on the \texttt{ANYPREVOUT}
  signature type, which does not sign the hash of the transaction it spends,
  therefore allowing for a single pre-signed transaction to spend any output
  with a suitable script. We conjecture that this primitive cannot be achieved
  without \texttt{ANYPREVOUT}.
