\section{High Level Explanation}
  Conceptually, our protocol is split into three main actions: channel opening,
  payments and closing. A channel $(P_1, P_n)$ between parties $P_1$ and $P_n$
  may be opened directly on-chain, in which case the two parties follow an
  opening procedure similar to that of LN, or it can be opened on top of a path
  of preexisting channels $(P_2, P_3), (P_3, P_4), \dots, (P_{n-3}, P_{n-2}),
  (P_{n-2}, P_{n-1})$. In the latter case all parties $P_i$ on the path follow
  our novel protocol, setting aside funds in their channels as collateral for
  the new virtual channel that is being opened. Once all intermediaries are
  committed, $P_1$ and $P_n$ finally create (and keep off-chain) their
  ``commitment'' transaction, following a logic similar to Lightning and thus
  their channel is open.

  A payment over an established channel follows a procedure heavily inspired by
  LN, but without the use of HTLCs. To be completed, a payment needs three
  messages to be exchanged by the two parties.

  Finally, the closing procedure of a channel $C$ can be completed unilaterally
  and consists of signing and publishing a number of transactions on-chain. As
  we will discuss later, the exact transactions that a party will publish vary
  depending on the actions of the parties controlling the channels that form the
  ``base'' of $C$ and the channels that are based on $C$. Our protocol can be
  augmented with a more efficient optimistic collaborative closing procedure,
  which however is left as future work.

  In more detail, to open a channel (c.f.~\ref{code:ln:open}) the two
  counterparties (a.k.a. ``endpoints'') first create new keypairs and exchange
  the resulting public keys ($2$ messages), then prepare the underlying base
  channels if the new channel is virtual ($12 \cdot (n-1)$ total messages, i.e.
  $6$ outgoing messages per endpoint and $12$ outgoing messages per
  intermediary, for $n-2$ intermediaries), next they exchange signatures for
  their respective initial commitment transactions ($2$ messages) and lastly, if
  the channel is simple (i.e. not virtual), the ``funder'' signs and publishes
  the ``funding'' transaction on-chain.  We here note that like LN, only one of
  the two parties, the funder, provides coins for a new channel.  This
  limitation simplifies the execution model and the analysis, but can be lifted
  at the cost of additional protocol complexity.

\subsection{Simple Channels}
  In a similar vein to earlier PCN proposals, having an open channel essentially
  means having very specific keys, transactions and signatures at hand, as well
  as checking the ledger periodically and being ready to take action if
  misbehaviour is detected. Let us first consider a simple channel that has been
  established between \alice and \bob where the former owns $c_A$ and the latter
  $c_B$ coins. There are three sets of transactions at play: A ``funding''
  transaction that is put on-chain, off-chain ``commitment'' transactions that
  spend the funding output on channel closure and off-chain ``revocation''
  transactions that spend commitment outputs in case of misbehaviour (c.f.
  Figure~\ref{figure:payment-layer}).

  \begin{figure}
    \subimport{./figures/manual-tikz/}{payment-layer.tex}
    \caption{Funding, commitment and revocation transactions}
    \label{figure:payment-layer}
  \end{figure}

  In particular, there is a single on-chain funding transaction that spends $c_A
  + c_B$ funder's coins, with a single output that is encumbered with a
  $2/\{\pk{A, F}, \pk{B, F}\}$ multisig and carries $c_A + c_B$ coins.

  Next, there are two commitment transactions, each of which can spend the
  funding tx and produce two outputs with $c_A$ and $c_B$ coins each. The two
  txs differ in the outputs' spending conditions: The $c_A$ output in \alice's
  commitment tx can be spent either by \alice after it has been on-chain for a
  pre-agreed period (i.e. it is encumbered with a ``timelock''), or by a
  ``revocation'' transaction (discussed below) via a $2$-of-$2$ multisig between
  the counterparties, whereas the $c_B$ output can be spent only by \bob without
  a timelock. \bob's commitment tx is symmetric: the $c_A$ output can be spent
  only by \alice without timelock and the $c_B$ output can be spent either by
  \bob after the timelock expiration or by a revocation tx. When a new pair of
  commitment txs are created (either during channel opening or on each update)
  \alice signs \bob's commitment tx and sends him the signature (and
  vice-versa), therefore \alice can unilaterally sign and publish her commitment
  tx but not \bob's (and vice-versa).

  Last, there are $2m$ revocation transactions, where $m$ is the total number of
  updates of the channel. The $j$th revocation tx held by an endpoint spends the
  output carrying the counterparty's funds in the counterparty's $j$th
  commitment tx. It has a single output spendable immediately by the
  aforementioned endpoint. Each endpoint stores $m$ revocation txs, one for each
  superseded commitment tx. This creates a disincentive for an endpoint to cheat
  by using any other commitment transaction than its most recent one to close
  the channel: the timelock on the commitment output permits its counterparty to
  use the corresponding revocation transaction and thus claim the cheater's
  funds.  Endpoints do not have a revocation tx for the last commitment
  transaction, therefore these can be safely published. For a channel update to
  be completed, the endpoints must exchange the signatures for the revocation
  txs that spend the commitment txs that just became obsolete.

  Observe that the above logic is essentially a simplification of LN.

\subsection{Virtual Channels}
  In order to gain intuition on how virtual channels function, consider $n-1$
  simple channels established between $n$ honest parties as before. $P_1$ (the
  funder) and $P_n$ want to open a virtual channel over these base channels.
  Before opening the virtual, each base channel is entirely independent, having
  different unique keys, separate on-chain funding outputs, a possibly different
  balance and number of updates. After the $n$ parties follow our novel virtual
  channel opening protocol, they will all hold off-chain a number of new,
  ``virtual'' transactions that spend their respective funding transactions and
  can themselves be spent by new commitment transactions in a manner that
  ensures fair funds allocation for all honest parties.

  In particular, apart from the transactions of simple channels, each of the two
  endpoints also has an ``initiator'' transaction that spends the funding output
  of its only base channel and produces two outputs: one new funding output for
  the base channel and one ``virtual'' output (c.f.
  Figures~\ref{figure:virtual-layer-endpoint},~\ref{code:virtual-layer:endpoint-txs}).
  If the
  initiator 
  transaction ends up on-chain, the latter output carries coins that
  will directly or indirectly fund the funding output of the virtual channel.
  This virtual funding output can in turn be spent by a commitment transaction
  that is negotiated and updated with direct communication between the two
  endpoints in exactly the same manner as the payments of simple channels.

  \begin{figure}
    \subimport{./figures/manual-tikz/}{virtual-layer-endpoint.tex}
    \caption{$A$ - $E$ virtual channel: $A$'s initiator transaction. Spends the
    funding output of the $A$-$B$ channel. Can be used if $B$ has not published
    a virtual transaction yet.}
    \label{figure:virtual-layer-endpoint}
  \end{figure}

  Intermediaries on the other hand store three sets of virtual transactions
  (Figure~\ref{code:virtual-layer:mid-txs}): ``initiator''
  (Figure~\ref{figure:virtual-layer-initiator}), ``extend-interval''
  (Figure~\ref{figure:virtual-layer-extend-interval}) and ``merge-intervals''
  (Figure~\ref{figure:virtual-layer-merge-intervals}). Each intermediary has one
  initiator tx, which spends the party's two funding outputs and produces four:
  one funding output for each base channel, one output that directly pays the
  intermediary coins equal to the total value in the virtual channel, and one
  virtual output. If both funding outputs are still unspent, publishing its
  initiator tx is the only way for an intermediary to close either of its
  channels.

  \begin{figure}
    \subimport{./figures/manual-tikz/}{intermediary-initiator.tex}
    \caption{$A$ - $E$ virtual channel: $B$'s initiator transaction. Spends the
    funding outputs of the $A$-$B$ and $B$-$C$ channels. Can be used if neither
    $A$ nor $C$ have published a virtual transaction yet.}
    \label{figure:virtual-layer-initiator}
  \end{figure}

  \begin{figure}
    \subimport{./figures/manual-tikz/}{intermediary-extend-interval.tex}
    \caption{$A$ - $E$ virtual channel: One of $B$'s extend interval
    transactions. $\sigma$ is the signature. Spends the virtual output of $A$'s
    initiator transaction and the funding output of the $B$-$C$ channel. Can be
    used if $A$ has already published its initiator transaction and $C$ has not
    published a virtual transaction yet.}
    \label{figure:virtual-layer-extend-interval}
  \end{figure}

  \begin{figure}
    \subimport{./figures/manual-tikz/}{intermediary-merge-intervals.tex}
    \caption{$A$ - $E$ virtual channel: One of $B$'s merge intervals
    transactions. Spends the virtual outputs of $A$'s and $C$'s virtual
    transactions. Can be used if both $A$ and $C$ have already published their
    initiator transactions. Notice that the interval of $C$'s virtual output
    only contains $C$, which can only happen if $C$ has published its initiator
    and not any other of its virtual transactions.}
    \label{figure:virtual-layer-merge-intervals}
  \end{figure}


  Furthermore, each intermediary has $O(n)$ extend-interval transactions. If
  exactly one of the party's two base channels' funding outputs is unspent,
  publishing an extend-interval transaction is the only way for the party to
  close that base channel. Such a transaction consumes two outputs: the only
  available funding output and a suitable virtual output, as discussed below. An
  extend-interval tx has three outputs: A funding output replacing the one just
  spent, one output that directly pays the intermediary coins equal to the total
  value of the virtual channel, and one virtual output.

  Last, each intermediary has $O(n^2)$ merge-intervals transactions. If both
  party's base channels' funding outputs are spent, publishing a merge-intervals
  transaction is the only way for the party to close either base channel. Such a
  transaction consumes two suitable virtual outputs, as discussed below. It has
  two outputs: One that directly pays the intermediary coins equal to the total
  value of the virtual channel, and one virtual output.

  To understand why this multitude of virtual transactions is needed, we now
  zoom out from the individual party and discuss the dynamic of the system as a
  whole. The first party $P_i$ that wishes to close a base channel observes that
  its funding output(s) remain(s) unspent and publishes its initiator
  transaction. First, this allows $P_i$ to use its commitment transaction to
  close the channel. Second, in case $P_i$ is an intermediary, it directly
  regains the coins it has locked for the virtual channel. Third, it produces a
  virtual output that can only be consumed by $P_{i-1}$ and $P_{i+1}$, the
  parties adjacent to $P_i$ (if any) with specific extend-interval transactions.
  The virtual output of this extend-interval transaction can in turn be spent by
  specific extend-interval transactions of $P_{i-2}$ or $P_{i+2}$ that have not
  published a transaction yet (if any) and so on for the next neighbours. The
  idea is that each party only needs to publish a single virtual transaction to
  ``collapse'' the virtual layer and each virtual output uniquely defines the
  continuous interval of parties that have already published a virtual
  transaction and only allow parties at the edges of this interval to extend it.
  This prevents malicious parties from indefinitely replacing a virtual output
  with a new one. As the name suggests, merge-intervals transactions are
  published by parties that are adjacent to two parties that have already
  published their virtual transactions an in effect joins the two intervals into
  one.

  Each virtual output can also be used as the funding output for the virtual
  channel after a timelock, to protect from unresponsive parties blocking the
  virtual channel indefinitely. This in turn means that if an intermediary
  observes either of its funding outputs being spent, it has to publish its
  suitable virtual transaction before the timelock expires to avoid losing
  funds. What is more, all virtual outputs need the signature of all parties to
  be spent before the timelock (i.e. they have an $n$-of-$n$ multisig) in order
  to prevent colluding parties from faking the intervals progression. To ensure
  that parties have enough time to react, the timelock of a virtual output is
  the maximum of the required timelocks of the intermediaries that can spend it,
  where the required timelock of a party is $p + s$ if its channel is simple
  ($p$ being a global constant representing the maximum number of blocks between
  activations of non-negligent parties and $s = (2 + q)\texttt{windowSize}$,
  stemming from Proposition~\ref{prop:liveness} of Appendix --
  \texttt{windowSize} is the maximum number of blocks a party may be missing
  from the latest blockchain state at any given time) \orfeas{explained
  windowSize, PTAL},  or $p +
  \sum\limits_{j = 2}^{n - 1}(s - 1 + t_j)$ if the channel is virtual, where $t_j$ is the required timelock of the $j$th base channel
  of the intermediary's channel. The only exception are virtual outputs that
  correspond to an interval that includes all parties, which can only be used as
  funding outputs for the virtual channel as its interval cannot be further
  extended, therefore the two separate spending methods and the associated
  timelock are dropped.

  Many extend-interval and merge-intervals transactions have to be able to spend
  different outputs, depending on the order other base parties publish their
  virtual transactions. For example, $P_3$'s extend-interval tx that extends the
  interval $\{P_1, P_2\}$ to $\{P_1, P_2, P_3\}$ must be able to spend both
  the virtual output of $P_2$'s initiator transaction and $P_2$'s
  extend-interval transaction which has spent $P_1$'s initiator transaction. The
  same issue is faced by commitment transactions of a virtual channel, as any
  virtual output can potentially be used as the funding ouput for the channel.
  In order for the received signatures for virtual and commitment txs to be
  valid for multiple previous outputs, the previously proposed
  \texttt{ANYPREVOUT} sighash flag~\cite{anyprevout} is needed to be added to
  Bitcoin. We show in Theorem~\ref{theorem:anyprevout} that variadic recursive
  virtual channels with $O(1)$ on-chain and subexponential number of off-chain
  transactions for each party cannot be constructed in Bitcoin without this
  flag. We hope this work provides additional motivation for this flag to be
  included in the future.

  Note also that the newly established virtual channel can itself act as a base
  for further virtual channels, as its funding output can be unilaterally put
  on-chain in a pre-agreed maximum number of blocks. This in turn means that, as
  we discussed above, a further virtual channel must take the delay of its
  virtual base channels into account to determine the timelocks needed for its
  own virtual outputs.

  As for the actual protocol needed to establish a virtual channel, $6$ rounds
  of communication are needed
  \TODO{you mean $6$ rounds of communication?}\orfeas{fixed}, starting from the funder and hop by hop reaching
  the fundee and back (c.f.~\ref{code:ln:prepare-base}). The first communicates
  parties' identities, their funding keys and their neighbours' channel
  balances, the second creates new commitment transactions, the third circulates
  virtual keys, all parties' coins and desired timelocks, the fourth and the
  fifth circulate signatures for the virtual transactions (signatures for
  virtual outputs and funding outputs respectively) and the sixth circulates
  revocation signatures for the old channel states.

  In order to better grasp the manner in which the construction described
  achieves its intended goals, let us now turn to an example. Consider an
  established virtual channel on top of $4$ preexisting simple base channels.
  Let $A$, $B$, $C$, $D$ and $E$ the relevant parties, which control the $(A,
  B)$, $(B, C)$, $(C, D)$ and $(D, E)$ base channels, along with the $(A, E)$
  virtual channel. After carrying out some payments, $A$ decides to close the
  virtual channel. It therefore publishes its initiator transaction, thus
  consuming the funding output of $(A, B)$ and producing (among others) a
  virtual output with the interval $\{A\}$. $B$ notices this before the timelock
  of the virtual output expires and publishes its extend-interval interval
  transaction that consumes the aforementioned virtual output and the funding
  output of $(B, C)$, producing a virtual output with the interval $\{A, B\}$.
  $C$ in turn publishes the corresponding extend-interval transaction, consuming
  the virtual output of $B$ and the funding output of $(C, D)$ while producing a
  virtual output with the interval $\{A, B, C\}$. Finally $D$ publishes the last
  extend-interval transaction, thus producing an interval with all players.
  Instead of a virtual output, it produces the funding output for the virtual
  channel $(A, E)$. Now $A$ can spend this funding output with its latest
  commitment transaction. The entire process is depicted schematically in
  Figure~\ref{figure:example-start-end}. Note that if any of $B$, $C$ or $D$ do
  not act within the timelock prescribed in their consumed virtual output, then
  $A$ or $E$ can spend the virtual output with their latest commitment
  transaction, thus eventually $A$ can close its virtual channel in all cases.

  \begin{figure*}
    \subimport{./figures/manual-tikz/}{example-start-end.tex}
    \caption{$4$ simple channels supporting a virtual. $A$ initiates the closing
    procedure and no party is negligent.}
    \label{figure:example-start-end}
  \end{figure*}
