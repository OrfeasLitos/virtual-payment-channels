\section{Introduction}
  The popularity of blockchain protocols in recent years has stretched their
  performance exposing a number of scalability considerations. In particular,
  Bitcoin and related blockchain protocols exhibit very high latency (e.g.,
  Bitcoin has a latency of 1h~\cite{bitcoin}) and a very low throughput (e.g.,
  Bitcoin can handle at most 7 transactions per second~\cite{scaling}), both
  significant shortcomings that jeopardize wider use and adoption and are to a
  certain extent inherent~\cite{scaling}. To address these considerations a
  prominent approach is to optimistically handle payments via a \emph{Payment
  Channel Network} (PCN) (see, e.g.,~\cite{DBLP:conf/fc/GudgeonMRMG20} for a
  survey). Payments over a PCN happen \emph{off-chain}, i.e., without adding any
  transactions to the underlying blockchain. They only use the blockchain as an
  arbiter in case of disputes.

The key primitive of PCN protocols is a payment channel. Two parties (the
\emph{endpoints}) initiate
the channel by locking some funds on-chain and subsequently exchange direct
messages to update the state of the channel. The key feature is that state
updates are not posted on-chain and hence they remain unencumbered by the
performance limitations of the underlying blockchain protocol, making them a
natural choice for parties that interact often. Multiple overlapping payment
channels can be combined and form a PCN.

Closing a non-virtual channel is an operation that involves posting the channel state
on-chain. Closing should be efficient, i.e., needing $O(1)$ on-chain
transactions, independent of how many payments have occured off-chain.
It is also essential that any party can unilaterally close a channel as
otherwise a malicious counterparty (i.e., the other channel participant) could
prevent an honest party from accessing their funds. This functionality however
raises an important design consideration: how to prevent malicious parties from
posting old states of the channel.
%
Addressing this issue can be done with some suitable use of transaction
\emph{timelocks}, a feature that prevents a transaction or a specific script from
being processed on-chain prior to a specific time (measured in block height).
%
For instance, diminishing transaction timelocks facilitated the Duplex
Micropayment Channels (DMC)~\cite{decker} at the expense of bounding the overall
lifetime of a channel. Using script timelocks, the Lightning Network
(LN)~\cite{lightning} provided a better solution that enabled channels to stay
open for an arbitrary duration: the key idea was to duplicate the state of the
channel between the two counterparties, say Alice and Bob, and facilitate a
punishment mechanism that can be triggered by Bob whenever Alice posts an old
state update and vice-versa. The script timelocking is essential to allow an
honest counterparty some time to act.

Interconnecting channels in LN enables endpoints to transmit funds
to each other as long as there is a route of payment channels that connects
them. The downside of this mechanism is that it requires the active involvement
of all parties along the path for each payment. Instead, \emph{virtual payment
channels} suggest the more attractive approach of an one-time off-chain
initialization step to set up a virtual payment channel over the preexisting
channels, which subsequently can
be used for direct payments with complexity ---in the optimistic case---
independent of the path length. When the virtual channel has exhausted
its usefulness, it can be closed off-chain if the involved parties cooperate.
Initial constructions for virtual channels capitalized on the extended functionality of Ethereum, e.g.,
Perun~\cite{perun} and GSCN~\cite{DBLP:conf/ccs/DziembowskiFH18}, while more
recent work~\cite{9519487} brought them closer to
Bitcoin-compatibility (by leveraging adaptor
signatures~\cite{DBLP:journals/iacr/AumayrEEFHMMR20}).

We call the parties of the underlying channels \emph{intermediaries}.
A virtual channel constructor can be thought of as an \emph{operator} over the
underlying payment channel primitive. We identify three natural
desiderata for it.

\begin{itemize}
\item Recursive. A recursive virtual channel constructor can operate over
channels that themselves could be the results of previous applications of the
operator. This is important in the context of PCNs since it allows building
virtual channels on top of pre-existing virtual channels, allowing the channel
structure to evolve dynamically.
\item Variadic. A variadic virtual channel constructor can virtualize any number
of input payment channels directly, i.e., without leveraging recursion, contrary to a \emph{binary} constructor. This is
important in the context of PCNs since it enables applying the operator to build
virtual channels of arbitrary length, without the undue overhead of opening,
managing and closing multiple virtual channels only to use the one at the
``top'' of the recursion.
\item Symmetric. A symmetric virtual channel constructor offers setup and
closing operations that are symmetric in terms of computation, network and storage cost between the two
endpoints or the intermediaries (but not necessarily a mix of both) for the
optimistic and pessimistic execution paths. Furthermore, payments by the two
endpoints are encumbered with the same delay. Importantly, this ensures that no
party is worse-off or better-off after an application of the operator in terms
of accessing the basic channel functionality.
\end{itemize}

We note that recursiveness, while identified already as an important design
property~\cite{DBLP:conf/ccs/DziembowskiFH18}, has not been achieved for Bitcoin-compatible channels
(it was achieved only for DMC-like fixed lifetime channels in~\cite{10.1007/978-3-030-65411-5_18} and left as an open question for LN-type channels in~\cite{9519487}).
This is because of the severe limitations imposed by the scripting language of Bitcoin-compatible systems.
With respect to the other two properties, observe that successive applications
of a recursive binary virtual channel operator to connect distant endpoints will
break symmetry (since the sequence of operator applications will impact the
participants' functions with respect to the resulting channel). This is of
particular concern since most previous virtual channel constructors proposed are
binary~\cite{DBLP:conf/ccs/DziembowskiFH18,9519487,10.1007/978-3-030-65411-5_18}.

The primary motivation for recursive channels is adding flexibility in
moving off-chain coins quickly, with minimal interaction and at a low cost, even
under consistently
congested ledger conditions. Without recursiveness and when facing
unresponsive channel parties, one would have to first
close its channel on-chain in order to then use some of its coins with
another party, which is as slow as any on-chain transaction and in case of high
congestion prohibitively expensive. Even if parties collaboratively
close the original channel off-chain, the entire
channel closes even if only some of its coins are needed. On the other hand, a
recursive
virtual channel permits using some of its coins with other parties by
opening off-chain a new virtual channel on top without involvement of the base parties of
the original channel, and even keeping the remaining coins in
the latter. Importantly, users can decide to open a recursive virtual channel
long after having established their underlying one. This flexibility can inspire confidence in virtual
channels, prompting users to transfer more coins off-chain and
reduce on-chain congestion.

A scenario only possible with both the recursive and the variadic
properties is as follows (Fig.~\ref{figure:graph}): Initially Alice has a channel with
Bob, Bob one with Charlie and Charlie one with Dave. Alice opens a virtual
channel with Dave over the $3$ channels -- this needs the variadic property.
After a while she realises she has to pay Eve a few
times, who happens to have a channel with Dave. Alice interacts just with
Dave and Eve to move half of her coins from her virtual channel with Dave to a
new one with Eve -- this needs the recursive property.

\begin{figure}[!htbp]
  \centering
  \subimport{./figures/manual-tikz/}{graph.tex}
  \caption{An example Elmo network with $5$ nodes, $4$ black \emph{simple}
  (i.e., on-chain) channels, $1$ blue virtual channel built only on simple ones,
  and $1$ green virtual channel built on the blue virtual and a simple one. Each
  virtual channel is connected with its \emph{base} channels with dashed lines
  of the same colour. The variadic and recursive properties of Elmo are
  showcased.}
  \label{figure:graph}
\end{figure}

\makeatletter%
\@ifclassloaded{IEEEtran}%
  {\paragraph{Our Contributions}}%
  {\paragraph{Our Contributions.}}%
\makeatother%
 Elmo (named after St.
Elmo's fire) is the first Bitcoin-suitable
recursive virtual channel constructor that supports channels
of indefinite lifetime. In addition, our constructor is variadic and symmetric. Both optimistic and
pessimistic execution paths are optimal in terms of round complexity: issuing
payments between two endpoints requires just three messages of size
independent of the channel length, closing a channel cooperatively
needs at most three messages from each party while
closing a channel unilaterally demands up to two on-chain transactions for
any involved party (endpoint or intermediary) that can be submitted
simultaneously, also independent of the channel length. We build Elmo on top
of Bitcoin, as this means it can be adapted for any blockchain that
supports Turing-complete smart contracts such as
Ethereum~\cite{wood2014ethereum}. The latter provides additional tools to
increase Elmo efficiency. Furthermore, Elmo can inspire future
blockchain designs that maintain minimal scripting capabilities while
providing robust off-chain functionality.

We achieve the above by leveraging a sophisticated virtual channel setup
protocol which, on the one hand, enables endpoints to use an interface that is
invariant between on-chain and off-chain (i.e., virtual) channels,
while on the other, parties can securely close the channel cooperatively
off-chain, or instead close unilaterally on-chain, following an arbitrary
activation sequence. The latter is achieved by enabling anyone to
start closing the channel, while subsequent respondents, following the activation sequence, can choose the right action to complete the closure process by posting a single transaction each.

  We formally prove the security of our protocol in the Universal
  Composition (UC)~\cite{uc} setting (Appx.~\ref{sec:uc}); our functionality
  \fchan (Appx.~\ref{subsec:fchan}) represents
  a single channel. It is a global functionality,
  as defined in~\cite{DBLP:conf/tcc/BadertscherCHTZ20} (cf.
  % splncs
  %Appx.~\ref{sec:globality}).
  Sec.~\ref{section:security}).
%
  Elmo requires the \texttt{ANYPREVOUT} signature
  type (candidate for
  inclusion in the next Bitcoin update\footnote{\url{https://anyprevout.xyz/}}), which does not sign the hash of the
  transaction it spends, thus enabling a single pre-signed transaction
  to spend any output with a suitable script. We leverage \texttt{ANYPREVOUT} to
  avoid exponential storage.
  %and ensure off-chain payments of base channels are practical.
  We further conjecture that without
  \texttt{ANYPREVOUT} no efficient off-chain virtual channel constructor
  over Bitcoin can be built. In particular, if any such protocol
  (i) offers an efficient closing operation (i.e., with $O(1)$ on-chain
  transactions), (ii) has parties locally store the channel state as transactions and
  signatures and (iii) does not require locking on-chain
  coins (unlike~\cite{donner}), then each party needs exponential
  space in the number of intermediaries. Note that the second protocol
  requirement is natural, since, to our knowledge, all trustless layer $2$
  protocols over Bitcoin require all implicated protocol parties to actively
  sign off every state transition and locally store the relevant transactions
  and signatures of their counterparties to ensure they can
  unilaterally exit later.
