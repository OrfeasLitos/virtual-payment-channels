\section{Further Future Work}
\label{sec:further-future-work}
  Here we provide additional future work directions which pertain to improving
  the usability and reliability of the protocol. As it currently stands, the
  timelocks calculated for the virtual channels are based on $p$
  (Figure~\ref{code:ln:init}) and $s$ (Figure~\ref{code:ln:exchange-open-sigs}),
  which are global constants that are immutable and common to all parties. The
  parameter $s$ stems from the liveness guarantees of Bitcoin, as discussed in
  Proposition~\ref{prop:liveness} and therefore cannot be tweaked. However, $p$
  represents the maximum time (in blocks) between two activations of a
  non-negligent party, so in principle it is possible for the parties to
  explicitly negotiate this value when opening a new channel and even
  renegotiate it after the channel has been opened if the counterparties agree.
  We leave this usability-augmenting protocol feature as future work.

  Our protocol is not designed to ``gracefully'' recover from a situation in
  which halfway through a subprotocol, one of the counterparties starts
  misbehaving. Currently the only solution is to unilaterally close the channel.
  This however means that DoS attacks (that still do not lead to channel fund
  losses) are possible. A practical implementation of our protocol would need to
  expand the available actions and states to be able to transparently and
  gracefully recover from such problems, avoiding closing the channel where
  possible, especially when the problem stems from network issues and not from
  malicious behaviour.

  Additionally, our protocol does not feature one-off multi-hop payments like
  those possible in Lightning. This however is a useful feature in case two
  parties know that they will only transact once, as opening a virtual channel
  needs substantially more network communication than performing an one-off
  multi-hop payment. It would be therefore fruitful to also enable the multi-hop
  payment technique and allow human users to choose which method to use in each
  case. Likewise, optimistic cooperative on-chain closing of simple channels
  could be done just like in Lightning, obviating the need to wait for the
  revocation timelock to expire and reducing on-chain costs if the counterparty
  is cooperative.
