\section{Model \& Construction}
\label{sec:construction}
  \subsection{Model}
  In this section we will examine the architecture and the details of our model,
  along with possible attacks and their mitigations. We follow the UCGS
  framework~\cite{DBLP:conf/tcc/BadertscherCHTZ20} to formulate the protocol and
  its security. We list the
  ideal-world global functionality \fchan in Appx.~\ref{sec:functionalities}
  (Figures~\ref{code:functionality:rules}-\ref{code:functionality:close}) and a
  simulator \simulator (Figures~\ref{code:simulator:flow}-\ref{code:simulator}),
  along with a real-world protocol \pchan
  (Figures~\ref{code:ln:init}-\ref{code:virtual-layer:punishment}) that
  UC-realises \fchan (Theorem~\ref{theorem:security:virtual}). We give a
  self-contained description in this section, while pointing to figures in
  Appx.~\ref{sec:functionalities} and~\ref{sec:protocol}, in case the reader is
  interested in a pseudocode style specification.

  As in previous formulations, (e.g.,~\cite{DBLP:conf/csfw/KiayiasL20}), the
  role of \environment corresponds to two distinct actors in a real world
  implementation. On the one hand \environment passes inputs that correspond to
  the desires of human users (e.g., open a channel, pay, close), on the other hand
  \environment is responsible with periodically waking up parties to check the
  ledger and act upon any detected counterparty misbehaviour, similar to an
  always-on ``daemon'' of real-life software that periodically nudges the
  implementation to perform these checks.

  Since it is possible that \environment fails to wake up a party often enough,
  \pchan explicitly checks whether it has become ``negligent'' every time it is
  activated and all security guarantees are conditioned on the party not being
  negligent. A party is deemed negligent if more than $p$ blocks have been added
  to \ledger between any consecutive pair of activations. The need for explicit
  negligence checking stems from the fact that party activation is entirely
  controlled by \environment and no synchrony limitations are imposed (e.g., via
  the use of \Fclock), therefore it can happen that an otherwise honest party is
  not activated in time to prevent a malicious counterparty from successfully
  using an old commitment transaction. If a party is marked as negligent, no
  balance security guarantees are given (cf.
  Lemma~\ref{lemma:real-balance-security-informal}). Note that in realistic
  software the aforementioned daemon is local and trustworthy, therefore it
  would never allow \pchan to become negligent, as long as the machine is
  powered on and in good order.

  \subsection{Ideal world functionality \fchan}
  \label{subsec:func-desc}
  Our ideal world functionality \fchan represents a single channel, either
  simple or virtual.
  It acts as a relay between \adversary and \environment, leaking all
  messages. This simplifies the functionality and facilitates the
  indistinguishability argument by having \simulator simply running internally
  the real world protocols of the channel parties \pchan with no modifications.
  Furthermore, the communication of parties with \ledger is handled by \fchan:
  when a simulated honest party in \simulator needs to send a message to
  \ledger, \simulator instructs \fchan to send this message to \ledger on this
  party's behalf.
  \fchan internally maintains two state machines, one per channel party (cf.
  Figures~\ref{figure:fchan-state-init},~\ref{figure:fchan-state-open-funder},~\ref{figure:fchan-state-open-fundee},~\ref{figure:fchan-state-pay},~\ref{figure:fchan-state-fund},~\ref{figure:fchan-state-close},~\ref{figure:fchan-state-corruption})
  that keep track of whether the parties are corrupted or negligent, whether the
  channel has opened, whether a payment is underway, which ITIs are to be
  considered \emph{kindred} parties (as they correspond to other channels
  owned by the same human user, discussed below) and whether the channel is
  currently closing collaboratively or has already
  closed. The single security check performed is \emph{whether the on-chain
  coins are at least equal to the expected balance once the channel closes}. If
  this check fails, \fchan halts. Since the protocol \pchan (which realises
  \fchan, cf. Theorems~\ref{theorem:security:simple}
  and~\ref{theorem:security:virtual}) never halts, this ideal world check
  corresponds to the security guarantee offered by \pchan. Note that this check
  is not performed for negligent parties, as \simulator notifies \fchan if a
  party becomes negligent and the latter omits the check. Thus
  indistinguishability between the real and the ideal world is not violated in
  case of negligence.

  Observe that a human user often participates in various channels, therefore it
  corresponds to more than one ITMs. This is the case for example for the funder
  of a virtual channel and the corresponding party of the first base channel.
  Such parties are called \emph{kindred}. They communicate locally (i.e., via
  inputs and outputs, without using the adversarially controlled network) and
  balance guarantees concern their aggregate coins. Formally this communication
  is modelled by having a virtual channel using its base channels as global
  subroutines, as defined in~\cite{DBLP:conf/tcc/BadertscherCHTZ20}.

  If we were using plain UC, the above would constitute a violation of the
  subroutine respecting property that functionalities have to fulfill. We
  leverage the concept of global functionalities put forth
  in~\cite{DBLP:conf/tcc/BadertscherCHTZ20} to circumvent the issue. More
  specifically, we say that a simple channel functionality is of ``level'' $1$,
  which is written as $\fchansup{1}$. Inductively, a virtual channel
  functionality that is based on channels of any ``level'' up to and including
  $n-1$ (but no further) has a ``level'' $n$, which we write as $\fchansup{n}$. Then $\fchansup{n}$
  is $(\ledger, \fchansup{1}, \dots, \fchansup{n-1})$-subroutine respecting,
  according to the definition of~\cite{DBLP:conf/tcc/BadertscherCHTZ20}. The
  same structure is used in the real world between protocols. This
  technique ensures that the necessary conditions for the validity of the
  functionality and the protocol are met and that the realisability proof can go
  through, as we will see in Section~\ref{section:security} in more detail.

  We could instead contain all the channels in a single, monolithic
  functionality (following the approach of~\cite{DBLP:conf/csfw/KiayiasL20}) and
  we believe that we could still carry out the security proof. Nevertheless,
  having the functionality correspond to a single channel has no drawbacks, as
  all desired security guarantees are provided by our modular architecture, and
  instead brings two benefits. Firstly, the functionality is easier to
  intuitively grasp, as it handles less tasks. Having a simple and intuitive
  functionality aids in its reusability and is an informal goal of the
  simulation-based paradigm. Secondly, this approach permits our functionality
  to be global, as defined in~\cite{DBLP:conf/tcc/BadertscherCHTZ20}.
  We note that the ideal functionality defined in~\cite{DBLP:journals/iacr/AumayrEEFHMMR20}
  is unsuitable for our case, as it requires direct access to the ledger, which
  is not the case for a \fchan corresponding to a virtual channel.

  \subsection{Real world protocol \pchan}
  \label{construction:real-world}
  Our real world protocol \pchan, ran by party $P$, consists of two
  subprotocols: the Lightning-inspired part, dubbed \textsc{ln}
  (Figures~\ref{code:ln:init}-\ref{code:ln:used-revocation}) and the novel
  virtual layer subprotocol, named \textsc{virt}
  (Figures~\ref{code:virtual-layer:keys}-\ref{code:virtual-layer:punishment}). A
  simple channel that is not the base of any virtual channel leverages only
  \textsc{ln}, whereas a simple channel that is the base of at least one virtual
  channel does leverage both \textsc{ln} and \textsc{virt}. A virtual channel
  uses both \textsc{ln} and \textsc{virt}.

\makeatletter%
\@ifclassloaded{IEEEtran}%
  {\subsubsection{\textsc{ln} subprotocol}}%
  {\subsubsection{\textsc{ln} subprotocol.}}%
\makeatother%
  \label{construction:ln}
  The \textsc{ln} subprotocol has two variations depending on whether $P$ is the
  channel funder (\alice) or the fundee (\bob). It performs a number of tasks:
  Initialisation takes a single step for fundees and two steps for funders.
  \textsc{ln} first receives a public key $\pk{P, \mathrm{out}}$ from
  \environment. This is the public key that should eventually own all $P$'s
  coins after the channel is closed. \textsc{ln} also initialises its internal
  variables. If $P$ is a funder, \textsc{ln} waits for a second activation to
  generate a keypair and then waits for \environment to endow it with some
  coins, which will be subsequently used to open the channel
  (Figure~\ref{code:ln:init}).

  After initialisation, the funder \alice is ready to open the channel. Once
  \environment gives to \alice the identity of \bob, the initial channel balance
  $c$ and, (looking forward to the \textsc{virt} subprotocol description)
  in case it is a virtual channel, the identities of the base channel owners
  (Figure~\ref{code:ln:open}), \alice generates and sends \bob her funding and
  revocation public keys ($\pk{A, F}$, $\pk{A, R}$, used for the funding and
  revocation outputs respectively) along with $c$, $\pk{A,
  \mathrm{out}}$, and the base channel identities (only for virtual channels).
  Given that \bob has been initialised, it generates funding and revocation keys
  and replies to \alice with $\pk{B, F}$, $\pk{B, R}$, and $\pk{B,
  \mathrm{out}}$ (Figure~\ref{code:ln:exchange-open-keys}).

  The next step prepares the base channels (Figure~\ref{code:ln:prepare-base})
  if needed.
  If our channel is a simple one, then \alice simply generates the funding tx.
  If it is a virtual and assuming all base parties (running \textsc{ln})
  cooperate, a chain of messages from \alice to \bob and back via all base
  parties is initiated (Figures~\ref{code:ln:virtualise:start-end}
  and~\ref{code:ln:open:virtualise:hops}). These messages let each successive
  neighbour know the identities of all the base parties. Furthermore each party
  instantiates a new ``host'' party that runs \textsc{virt}. It also generates
  new funding keys and communicates them, along with its ``out'' key $\pk{P,
  \mathrm{out}}$ and its
  leftward and rightward balances. If this circuit of messages completes, \alice
  delegates the creation of the new virtual layer transactions to its new
  \textsc{virt} host, which will be discussed later in detail. If the virtual
  layer is successful, each base party is informed by its host accordingly,
  intermediaries return to the \textsc{open} state (i.e., they have completed
  their part and are in standby, ready to accept instructions for, e.g., new
  payments) and \alice and \bob continue
  the opening procedure. In particular, \alice and \bob exchange signatures on
  the initial commitment transactions, therefore ensuring that the funding
  output can be spent (Figure~\ref{code:ln:exchange-open-sigs}). After that, in
  case the channel is simple the funding transaction is put on-chain
  (Figure~\ref{code:ln:commit-base}) and finally \environment is informed of the
  successful channel opening.

  There are two facts that should be noted: Firstly, in case the opened channel
  is virtual, each intermediary necessarily partakes in two channels.
  However each protocol instance only represents a party in a single channel,
  therefore each intermediary is in practice realised by two kindred
  \pchan instances that communicate locally, called ``siblings''. Secondly, our
  protocol is not designed to gracefully recover if other parties do not send an
  expected message at any point in the opening or payment procedure. Such
  anti-Denial-of-Service measures would greatly complicate the protocol and are
  left as a task for a real world implementation. It should however be stressed
  that an honest party with an open channel that has fallen victim to such an
  attack can still unilaterally close the channel, therefore no coins are lost
  in any case.

  Once the channel is open, \alice and \bob can carry out an unlimited number of
  payments in either direction, only needing to exchange $3$ direct network
  messages with each other per payment, therefore avoiding the slow and costly
  on-chain validation. The payment procedure is identical for simple and virtual
  channels and crucially it does not implicate the intermediaries (and therefore
  \alice and \bob do not incur any delays such an interaction with
  intermediaries would introduce). For a payment to be carried out, the payee is
  first notified by \environment (Figure~\ref{code:ln:get-paid}) and
  subsequently the payer is instructed by \environment to commence the payment
  (Figure~\ref{code:ln:pay}).

  If the channel is virtual, each party also checks that its upcoming balance is
  lower than the balance of its sibling's counterparty and that the upcoming
  balance of the counterparty is higher than the balance of its own sibling,
  otherwise it rejects the payment. This is to mitigate a ``griefing'' attack (i.e.,
  one that does not lead to financial gain) where a malicious counterparty
  uses an old commitment transaction to spend the base funding output, therefore
  blocking the honest party from using its initiator virtual transaction. This
  check ensures that the coins gained by the punishment are sufficient to cover
  the losses from the blocked initiator transaction. If the attack takes place,
  other local channels based directly or indirectly on it are informed and are
  moved to a failed state. Note that this does not bring a risk of losing any of
  the total coins of all local channels. We conjecture that this balance
  constraint can be lifted if the current Lightning-inspired payment method is
  replaced with an eltoo-inspired one~\cite{eltoo}.

  Subsequently each of the two parties builds the new commitment transaction of
  its counterparty and signs it. It also generates a new revocation keypair for
  the next update and sends over the generated signature and public key. Then
  the revocation transactions for the previously valid commitment transactions
  are generated, signed and the signatures are exchanged. To reduce the number
  of messages, the payee sends the two signatures and the public key in one
  message. This does not put it at risk of losing funds, since the new
  commitment transaction (for which it has already received a signature and
  therefore can spend) gives it more funds than the previous one.

  \pchan also checks the chain for outdated commitment transactions by the
  counterparty and publishes the corresponding revocation transaction in case
  one is found (Figure~\ref{code:ln:poll}). It also keeps track of whether the
  party is activated often enough and marks it as negligent otherwise
  (Figure~\ref{code:ln:init}). In particular, at the beginning of every activation
  while the channel is open, \textsc{ln} checks if the party has been activated
  within the last $p$ blocks (where $p$ is an implementation-dependent global
  constant) by reading from \ledger and comparing the current block height with
  that of the last activation.

  Cooperative closing involves both \textsc{ln}
  (Figures~\ref{code:ln:coop-close}-\ref{code:ln:coop-close-funder}) and
  \textsc{virt} (Figure~\ref{code:virtual-layer:coop-close-intermediary})
  subprotocols. Any party can initiate it by asking the virtual channel fundee.
  The latter signs the last coin balance and sends it to the funder, who first
  ensures the fundee signed the correct balance, then signs it as well. Its
  enabler (i.e., the kindred party that is a member of the $1$st base channel)
  generates and signs a new commitment tx in which it adds the funder's coins to
  its own and the fundee's coins to its counterparty's, while using the funding
  keys that were used before opening the virtual channel. It also generates a
  new revocation keypair for the next channel update and sends the revocation
  public key with the signature and the final virtual channel balance to its
  counterparty. The latter verifies the signature and that the two virtual
  channel parties agree on their final balance. If all goes well, it passes
  control to its kindred party that is a member of the next channel in sequence.
  If no verification fails, the process repeats until the fundee is reached. Now
  a backwards sequence of messages begins, in which each party that previously
  did verification now provides a signature for the new commitment tx, along
  with a revocation signature for the old commitment tx and a new revocation
  public key for the next update. Each receiver verifies the signatures and
  ``passes the baton'' to its kindred party closer to the funder. When the
  funder is reached, the last series of messages begins. Now each party that has
  not yet sent a revocation does so. Once the chain of messages reaches the
  fundee, the channel has successfully closed cooperatively. In total, each
  \textsc{ln} party sends and stores $2$ signatures, $1$ private key and $1$
  public key. The associated behaviour of the \textsc{virt} subprotocol is
  discussed later.

  Alternatively, when either party is instructed by \environment to unilaterally close the
  channel (Figure~\ref{code:ln:close}), it first asks its host to unilaterally
  close (details on the exact steps are discussed later) and once that is done,
  the ledger is checked for any transaction spending the funding output. In case
  the latest remote commitment tx is on-chain, then the channel is already
  closed and no further action is necessary. If an old committment transaction
  is on-chain, the corresponding revocation transaction is used for punishment.
  If the funding output is still unspent, the party attempts to publish the
  latest commitment transaction after waiting for any relevant timelock to
  expire. Until the funding output is irrevocably spent, the party still has to
  periodically check the blockchain and again be ready to use a revocation
  transaction if an old commitment transaction spends the funding output after
  all (Figure~\ref{code:ln:poll}).

\makeatletter%
\@ifclassloaded{IEEEtran}%
  {\subsubsection{\textsc{virt} subprotocol}}%
  {\subsubsection{\textsc{virt} subprotocol.}}%
\makeatother%
  \label{construction:virt}
  This subprotocol acts as a mediator between the base channels and the
  Lightning-based logic. Put simply, its main responsibility is putting on-chain
  the funding output of the channel when needed. When first initialised by a
  machine that executes the \textsc{ln} subprotocol
  (Figure~\ref{code:virtual-layer:keys}), it learns and stores the identities,
  keys, and balances of various relevant parties, along with the required
  timelock and other useful data regarding the base channels. It then generates
  a number of keys as needed for the rest of the base preparation. If the
  initialiser is also the channel funder, then the \textsc{virt} machine
  initiates $4$ ``circuits'' of messages. Each circuit consists of one message
  from the funder $P_1$ to its neighbour $P_2$, one message from each
  intermediary $P_i$ to the ``next'' neighbour $P_{i+1}$, one message from the
  fundee $P_n$ to its neighbour $P_{n-1}$ and one more message from each
  intermediary $P_i$ to the ``previous'' neighbour $P_{i-1}$, for a total of
  $2\cdot(n-1)$ messages per circuit.

  The first circuit (Figure~\ref{code:virtual-layer:circulate-keys-and-coins})
  communicates all ``out'', virtual, revocation and funding keys (both old and new), all
  balances and all timelocks among all parties. In the second circuit
  (Figure~\ref{code:virtual-layer:virtual-sigs}) every party receives and
  verifies all signatures for all inputs of its virtual and bridge transactions
  that spend a virtual output. It also produces and sends its own such
  signatures to the other parties. Each party generates and circulates
  $S = 2(n-2) + (i-3)(n-i) +(i-1)(n-i-2) + \chi_{i=3}(2(n-i)-1) +
  \chi_{i=n-2}(2i-3) + 3 + \sum\limits_{i = 2}^{n-2} (n-3 + \chi_{i = 2} +
  \chi_{i = n - 1} + 2(i - 2 + \chi_{i = 2})(n - i - 1 + \chi_{i = n - 1})) \in
  O(n^3)$ signatures (where $\chi_A$ is the characteristic function that equals
  $1$ if $A$ is true and $0$ else), which is derived by calculating the total
  number of bridge transactions and virtual outputs of all parties' virtual
  transactions -- we remind that each virtual output can be spent either by a
  $n$-of-$n$ multisig via a new virtual transaction, or by a $4$-of-$4$ multisig
  via its bridge transaction.
%  , for a   total of $nS \in O(n^4)$ signatures in this phase. 
  On a related note, the total number of virtual and bridge transactions for
  which each party needs to store signatures is $2$ for the two endpoints
  (Figure~\ref{code:virtual-layer:endpoint-txs}) and $2(n - 2 + \chi_{i = 2} +
  \chi_{i = n - 1} +  (i - 2 + \chi_{i = 2}) (n - i - 1 + \chi_{i = n-1})) \in
  O(n^2)$ for the $i$-th intermediary
  (Figure~\ref{code:virtual-layer:mid-txs}). The latter is derived by counting
  the number of extend-interval and merge-intervals transactions held by the
  intermediary, which are equal to the number of distinct intervals that the
  party can extend and the number of distinct pairs of intervals that the party
  can merge respectively, plus $1$ for the unique initiator transaction of the
  party.
  The third circuit concerns sharing signatures for the funding outputs
  (Figure~\ref{code:virtual-layer:funding-sigs}). Each party signs all
  transactions that spend a funding output relevant to the party, i.e., the
  initiator transaction and some of the extend-interval transactions of its
  neighbours. The two endpoints send $2$ signatures each when $n = 3$ and $n -
  2$ signatures each when $n > 3$, whereas each intermediary sends $2 + \chi_{i
  + 1 < n}(n - 2 + \chi_{i = n - 2}) + \chi_{i - 1 > 1}(n - 2 + \chi_{i = 3})
  \in O(n)$ signatures each. The last circuit of messages
  (Figure~\ref{code:virtual-layer:revocation}) carries the revocations of the
  previous states of all base channels. After this, base parties can only use
  the newly created virtual transactions to spend their funding outputs. In this
  step each party exchanges a single signature with each of its neighbours.

  In case of a cooperative closing, \textsc{virt} orchestrates the hosted
  \textsc{ln} ITIs, instructing them to perform the actions discussed
  previously. It also is responsible for sending the actual messages to the host
  of the next counterparty and receiving its responses. Apart from controlling
  the flow of messages, a \textsc{virt} ITI also generates revocation signatures
  to invalidate its virtual and bridge transactions and verifies the respective
  revocation signatures generated by its counterparty \textsc{virt} ITI, thereby
  ensuring that, moving forward, the use of an old virtual or bridge transaction
  can be punished.

  On the other hand, when \textsc{virt} is instructed to unilaterally close by
  party $R$
  (Figure~\ref{code:virtual-layer:close}), it first notifies its \textsc{virt}
  host (if any) and waits for it to unilaterally close. After that, it signs and
  publishes the unique valid virtual transaction. It then repeatedly checks the
  chain to see if the transaction is included
  (Figure~\ref{code:virtual-layer:check-chain-close}). If it is included, the
  virtual layer is closed and \textsc{virt} informs (i.e., outputs
  (\textsc{closed}) to) $R$. The instruction to close has to be received
  potentially many times, because a number of virtual transactions (the ones
  that spend the same output) are mutually exclusive and therefore if another
  base party publishes an incompatible virtual transaction contemporaneously and
  that remote transaction wins the race to the chain, then our \textsc{virt}
  party has to try again with another, compatible virtual transaction.
