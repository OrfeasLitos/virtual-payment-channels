
\paragraph{Related work}
 The first proposal for PCNs was due to~\cite{spilman} which only enabled
 unidirectional payment channels. As mentioned previously, DMCs~\cite{decker} with their decrementing timelocks have the shortcoming of limited channel lifetime. This was ameliorated by LN~\cite{lightning} which as become the dominant paradigm for designing PCNs for bitcoin-compatible systems. 
 Lightning is now implemented and functional for
  Bitcoin. It has also been adapted for Ethereum~\cite{wood2014ethereum}, where
  it is known as the Raiden Network~\cite{raiden}.

A number of attacks have been identified against LN. The wormhole
  attack~\cite{DBLP:conf/ndss/MalavoltaMSKM19} against Lightning allows
  colluding parties in a multi-hop payment to steal the fees of the
  intermediaries between them and Flood \& Loot~\cite{10.1145/3419614.3423248}
  analyses the feasibility of an attack in which too many channels are forced to
  close in a short amount of time, reducing the blockchain liveness and enabling
  a malicious party to steal off-chain funds.

  Payment routing~\cite{spider,prihodko2016flare,lee2020routee} is another research area that aims to improve the network efficiency without sacrificing  privacy. Actively rebalancing channels~\cite{DBLP:conf/ccs/KhalilG17} can
  further increase network efficiency by preventing routes from becoming   unavailable due to lack of well-balanced funds.

  An alternantive payment channel construction that aspires to be the successor
  of Lightning is eltoo~\cite{eltoo}. It has a conceptually simpler
  construction, smaller on-chain footprint and a more forgiving attitude towards
  submitting an old channel state than Lightning, but it needs the
  \texttt{ANYPREVOUT} sighash flag to be added to Bitcoin. Generalized
  Bitcoin-Compatible Channels~\cite{cryptoeprint:2020:476} enable the creation
  of state channels on Bitcoin, extending channel functionality from simple
  payments to arbitrary Bitcoin scripts.

  Sprites~\cite{sprites} leverages the scripting language of Ethereum to
  decrease the time collateral is locked up compared to Lightning.
  Perun~\cite{perun} and GSCN~\cite{DBLP:conf/ccs/DziembowskiFH18} exploit the
  Turing-complete scripting language of Ethereum to provide virtual state
  channels, i.e. channels that can open without an on-chain transaction and that
  allow for arbitrary scripts to be executed off-chain. Similar features are
  provided by Celer~\cite{dong2018celer}. Hydra~\cite{cryptoeprint:2020:299}
  provides state channels for the Cardano~\cite{cardano} blockchain which
  combines a UTXO type of model with general purpose smart contract
  functionality that are also isomorphic, i.e. Hydra channels can accommodate
  any script that is compatible with the Cardano blockchain.

  BDW~\cite{scalable-funding} shows how pairwise channels over Bitcoin can be   funded with no on-chain transactions by allowing parties to form groups that   can pool their funds together off-chain and then use those funds to open   channels. ACMU~\cite{10.1145/3319535.3345666} allows for multi-path atomic   payments with reduced collateral, enabling new applications such as   crowdfunding conditional on reaching a funding target.

  TEE-based~\cite{zhao2019sok}
solutions~\cite{teechan,10.1145/3341301.3359627,liao2021speedster,lee2020routee}
  improve the throughput and efficiency of PCNs by an order of magnitude or
  more, at the cost of having to trust TEEs. Brick~\cite{avarikioti2020brick}
  uses a partially trusted committee to extend PCNs to fully asynchronous
  networks.

  Solutions alternative to PCNs include sidechains~\cite{cryptoeprint:2020:175}
  and partially centralised payment networks that entirely avoid using a blockchain~\cite{DBLP:conf/trust/ArmknechtKMYZ15,stellar,silentwhispers,DBLP:conf/ndss/RoosMKG18}. \TODO{Mention also non-custodial chains}

  Last but not least, a number of works propose virtual channel constructions
  for Bitcoin. Lightweight Virtual Payment
  Channels~\cite{10.1007/978-3-030-65411-5_18} enables a virtual channel to be
  opened on top of two preexisting channels and uses a technique similar to DMC.
  Bitcoin-Compatible Virtual Channels~\cite{cryptoeprint:2020:554} also enables
  virtual channels on top of two preexisting simple (i.e. non-virtual) channels
  and offers two protocols, the first of which guarantees that the channel will
  stay off-chain for an agreed period, while the second allows the single intermediary
  to turn the virtual into a simple channel. 
We remark that the above strategy has the shortcoming that even if it is made recursive (a direction left open in \cite{cryptoeprint:2020:554}) after $k$ applications of the constructor there will be $k-1$ specific intermediaries that have to act in a specific order to close the channel. \TODO{double check the previous sentence}. 
  We refer the reader to Table~\ref{table:comparison} for a comparison of the
  features and limitations of virtual channel protocols, including the one put
  forth in the current work.

  \begin{table*}
    \caption{Comparison of virtual channel protocols}
    \label{table:comparison}
    \begin{minipage}{\textwidth}
    \begin{center}
    \begin{tabular}{|l|c|c|c|c|}
    \hline
              & Unlimited lifetime & Recursive & Multi-hop & no need for \texttt{ANYPREVOUT} \\
    \hline
    LVPC~\cite{10.1007/978-3-030-65411-5_18}
              & ✗                  & \LEFTcircle\footnote{lacks security analysis}
                                               & ✗         & ✓ \\
    \hline
    BCVC~\cite{cryptoeprint:2020:554}
              & ✓                  & ✗         & ✗         & ✓ \\
    \hline
    GBCC~\cite{cryptoeprint:2020:476}
              & ✓                  & ✓         & ✗         & ✓ \\
    \hline
    this work & ✓                  & ✓         & ✓         & ✗ \\
    \hline
    \end{tabular}
    \end{center}
    \end{minipage}
  \end{table*}

  \cite{DBLP:conf/fc/GudgeonMRMG20}, \cite{spider}, \cite{lightning},
  \cite{eltoo}, \cite{raiden}, \cite{bitcoin}, \cite{scaling}, \cite{decker},
  \cite{DBLP:conf/trust/ArmknechtKMYZ15}, \cite{stellar}, \cite{silentwhispers},
  \cite{DBLP:conf/ndss/RoosMKG18}, \cite{DBLP:conf/ccs/DziembowskiFH18},
  \cite{perun}, \cite{teechan}, \cite{sprites}, \cite{prihodko2016flare},
  \cite{DBLP:conf/ndss/MalavoltaMSKM19}, \cite{cryptoeprint:2020:175},
  \cite{cryptoeprint:2020:554}, \cite{cryptoeprint:2020:476},
  \cite{avarikioti2020brick}, \cite{10.1145/3319535.3345666},
  \cite{10.1007/978-3-030-65411-5_18}, \cite{cryptoeprint:2020:299},
  \cite{dong2018celer}, \cite{10.1145/3341301.3359627}, \cite{liao2021speedster}
