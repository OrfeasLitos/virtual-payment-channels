\section{Future Work}
\label{sec:extra-future-work}
  A number of features can be added to our protocol for additional efficiency,
  usability and flexibility. First of all, in our current construction, each
  time a particular channel $C$ acts as a base
  channel for a new virtual channel, one more ``virtualisation layer'' is added. When
  one of its owners wants to close $C$, it has to put on-chain as many
  transactions as there are virtualisation layers. Also the timelocks associated
  with closing a virtual channel increase with the number of virtualisation
  layers of its base channels. Both these issues can be alleviated by extending
  the opening and cooperative closing subprotocol with the ability to
  cooperatively open and close multiple virtual channels in the same layer,
  either simultaneously or by amending an existing virtualisation layer.

  In this work we only allow a channel to be funded by one of the two endpoints.
  This limitation simplifies the execution model and analysis, but can be lifted
  at the cost of additional protocol complexity.

  Furthermore, as it currently stands, the
  timelocks calculated for the virtual channels are based on $p$
  (Figure~\ref{code:ln:init}) and $s$ (Figure~\ref{code:ln:exchange-open-sigs}),
  which are global constants that are immutable and common to all parties. The
  parameter $s$ stems from the liveness guarantees of Bitcoin, as discussed in
  Proposition~\ref{prop:liveness} (Appx.~\ref{subsec:liveness}) and therefore cannot be tweaked. However, $p$
  represents the maximum time (in blocks) between two activations of a
  non-negligent party, so in principle it is possible for the parties to
  explicitly negotiate this value when opening a new channel and even
  renegotiate it after the channel has been opened if the counterparties agree.
  We leave this usability-augmenting protocol feature as future work.

  Our protocol is not designed to ``gracefully'' recover from a situation in
  which halfway through a subprotocol, one of the counterparties starts
  misbehaving. Currently the only solution is to unilaterally close the channel.
  This however means that DoS attacks (that still do not lead to channel fund
  losses) are possible. A practical implementation of our protocol would need to
  expand the available actions and states to be able to transparently and
  gracefully recover from such problems, avoiding closing the channel where
  possible, especially when the problem stems from network issues and not from
  malicious behaviour.

  Additionally, our protocol does not feature one-off multi-hop payments like
  those possible in Lightning. This however is a useful feature in case two
  parties know that they will only transact once, as opening a virtual channel
  needs substantially more network communication than performing an one-off
  multi-hop payment. It would be therefore fruitful to also enable the multi-hop
  payment technique and allow human users to choose which method to use in each
  case. Likewise, optimistic cooperative on-chain closing of simple channels
  could be done just like in Lightning, obviating the need to wait for the
  revocation timelock to expire and reducing on-chain costs if the counterparty
  is cooperative.

  What is more, any deployment of the protocol has to explicitly handle the issue
  of tx fees. These include miner fees for on-chain txs and
  intermediary fees for the parties that own base channels and facilitate
  opening virtual channels. These fees should take into account the fact that
  each intermediary has quadratic storage requirements, whereas endpoints only
  need constant storage, creating an opportunity for amplification attacks.
  Additionally, a fee structure that takes into account the opportunity cost of
  base parties locking collateral for a potentially long time is needed. A
  straightforward mechanism is for parties to agree on a time-based fee schedule
  and periodically update their base channels to reflect contingent payments by
  the endpoints. We leave the relevant incentive analysis as future work.

  In order to increase readability and to keep focus on the salient points of
  the construction, our protocol does not exploit various possible
  optimisations. These include allowing parties to stay offline for
  longer~\cite{DBLP:conf/ccs/AumayrTMMM22}, and some techniques employed in Lightning that
  drastically reduce storage requirements, such as storage of per-update secrets
  in $O(\log n)$
  space\footnote{\url{https://github.com/lightning/bolts/blob/master/03-transactions.md\#efficient-per-commitment-secret-storage}},
  and other improvements to our novel virtual subprotocol.

  As mentioned before, we conjecture that a variadic virtual
  channel protocol with unlimited lifetime needs each party to store an
  exponential number of signatures if \texttt{ANYPREVOUT} is not available. We
  leave proof of this as future work. Furthermore, the formal verification of
  the UC security proof is deferred to such a time when a practical framework
  for mechanised UC proofs becomes available.

  Last but not least, the current analysis gives no privacy guarantees for the
  protocol, as it does not employ onion packets~\cite{sphinx} like Lightning.
  Furthermore, \fchan leaks all messages to the ideal adversary therefore
  theoretically no privacy is offered at all. Nevertheless, onion packets can be
  incorporated in the current construction. Intuitively our construction
  leaks less data than Lightning for the same multi-hop payments, as
  intermediaries in our case are not notified on each payment, contrary to
  multi-hop payments in Lightning. Therefore a future extension can improve the
  privacy of the construction and formally demonstrate exact privacy guarantees.

