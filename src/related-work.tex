\makeatletter%
\@ifclassloaded{IEEEtran}%
  {\paragraph{Related work}}%
  {\paragraph{Related work.}}%
\makeatother%
 The first proposal for PCNs~\cite{spilman} only enabled
 unidirectional payment channels. As mentioned previously, DMCs~\cite{decker}
 with their decrementing timelocks have the shortcoming of limited channel
 lifetime. This was ameliorated by LN~\cite{lightning} which has become the dominant paradigm for designing Bitcoin-compatible PCNs.
 LN is currently implemented and operational for
  Bitcoin. It has also been adapted for Ethereum, named
  Raiden Network. Compared to Elmo, LN is more lightweight in terms of
  storage and communication when setting up, but suffers from increased latency
  and communication for payments, as intermediaries have to actively participate
  in multi-hop payments. Its privacy also suffers, as intermediaries
  learn the exact time and value of each payment.

%Various attacks have been identified against LN. The wormhole
%  attack~\cite{DBLP:conf/ndss/MalavoltaMSKM19} against LN allows
%  colluding parties in a multi-hop payment to steal the fees of the
%  intermediaries between them and Flood \& Loot attacks~\cite{10.1145/3419614.3423248}
%  analyses an attack in which too many channels are forced to
%  close in a short amount of time, harming blockchain liveness and enabling
%  a malicious party to steal off-chain funds.
%
%  To the best of our knowledge, no formal treatment of the privacy of LN exists.
%  Nevertheless, it intuitively improves upon the privacy of on-chain Bitcoin
%  transactions, as LN payments do not leave a permanent record: only
%  intermediaries of each payment are informed. It can be argued that Elmo
%  further improves privacy, as payments are hidden from
%  the intermediaries of a virtual channel.
%
%  Payment routing~\cite{spider,prihodko2016flare,lee2020routee} is another research area that aims to improve network efficiency without sacrificing  privacy. Actively rebalancing channels~\cite{DBLP:conf/ccs/KhalilG17} can
%  further increase network efficiency by reducing unavailable routes due to lack of well-balanced funds.

  An alternantive payment channel system for Bitcoin that aspires to
  succeed LN is eltoo~\cite{eltoo}. It is conceptually simpler,
  has smaller on-chain footprint and a more forgiving attitude towards
  submitting an old channel state than LN (the old state is superseded without punishment), but it needs
  \texttt{ANYPREVOUT}. Since eltoo and LN function similarly, the previous comparison of
  Elmo with LN applies to eltoo as well. On a related note, the payment
  logic of Elmo could also be designed based on the eltoo mechanism instead of
  the currently used LN.

%  Bolt~\cite{10.1145/3133956.3134093} constructs privacy-preserving payment
%  channels enabling both direct payments and payments with a single untrusted
%  intermediary. Sprites~\cite{sprites} leverages the scripting language of
%  E\-the\-re\-um to decrease the time collateral is locked compared to LN.
%
%  State channels are a generalisation of payment channels, which enable
%  off-chain execution of any smart contract supported by the underlying
%  blockchain, not just payments. Generalized Bitcoin-Compatible
%  Channels~\cite{DBLP:journals/iacr/AumayrEEFHMMR20} enable the creation of
%  state channels on Bitcoin, extending channel functionality from simple
%  payments to arbitrary Bitcoin scripts. Since Elmo only pertains to payment,
%  not state, channels, we choose not to build it on top
%  of~\cite{DBLP:journals/iacr/AumayrEEFHMMR20}. State channels can also be
%  extended to more than two
%  parties~\cite{DBLP:conf/asiaccs/LiaoZSS22,DBLP:conf/eurocrypt/DziembowskiEFHH19}.

  Perun~\cite{perun} and GSCN~\cite{DBLP:conf/ccs/DziembowskiFH18} exploit the
  Turing-complete scripting language of Ethereum to provide virtual state
  channels.
  % GSCN also uses a per-channel functionality and a similar recursive argument
  % as we use in our UC-security analysis. Their security argument is however
  % flawed, as they incorrectly argue that every level is subroutine respecting
  % with respect to the same environment and subroutines.
  We believe that, given the versatile scripting of Ethereum, GSCN could be
  straightforwardly extended to support variadic channels. Similar
  features are provided by Celer~\cite{dong2018celer}.
  Hydra~\cite{cryptoeprint:2020:299} provides state channels for
  Cardano~\cite{cardano}.

  %BDW~\cite{scalable-funding} shows how pairwise channels over Bitcoin can be
  %funded with no on-chain transactions by allowing parties to form groups that
  %can pool their funds together off-chain and then use those funds to open
  %channels. Such proposals are complementary to virtual channels and, depending
  %on the use case, could be more efficient. In comparison to Elmo, BDW is less
  %flexible: coins in a BDW pool can only be exchanged with members of that
  %pool. ACMU~\cite{10.1145/3319535.3345666} allows for multi-path atomic
  %payments with reduced collateral, enabling new applications such as
  %crowdfunding conditional on reaching a funding target.

%  TEE-based~\cite{zhao2019sok}
%solutions~\cite{teechan,10.1145/3341301.3359627,DBLP:conf/asiaccs/LiaoZSS22,lee2020routee}
%  improve the throughput and efficiency of PCNs by an order of magnitude or
%  more, at the cost of having to trust TEEs. Brick~\cite{avarikioti2020brick}
%  uses a partially trusted committee to extend PCNs to fully asynchronous
%  networks.

  Solutions alternative to PCNs include side\-chains
  (e.g.,~\cite{BCDF+14,sidechains,KiaZin18}), commit-chains
  (e.g.,~\cite{plasma}) and non-cu\-sto\-di\-al chains
  (e.g.,~\cite{plasma,konstantopoulos2019plasma,plasma-lower-bounds}). These
  approaches offer more efficient payment methods, at the cost of
  requiring a distinguished mediator, additional tust, or on-chain
  checkpointing. Furthermore, they do not enable payments between different instances
  of the same protocol. Due to their conceptual and interface differences as
  well as differing levels of software maturity, dedicated user studies need to
  be carried out in order to compare the usability and overall costs of each
  approach under various usage patterns. Rollups~\cite{ZKRollup,Optimism} are
  incompatible with Bitcoin, as they only optimise computation, not storage,
  whereas Bitcoin has by design minimal computation needs.
%  and partially centralised payment networks that entirely avoid using a
%  blockchain~\cite{DBLP:conf/trust/ArmknechtKMYZ15,stellar,silentwhispers,DBLP:conf/ndss/RoosMKG18}.

  Last but not least, a number of works propose virtual channel constructions
  for Bitcoin. LVPC~\cite{10.1007/978-3-030-65411-5_18} enables a virtual channel to be
  opened on top of two preexisting channels and uses a technique similar to DMC,
  unfortunately inheriting the fixed lifetime limitation.
  Let \emph{simple channels} be those built directly on-chain, i.e., channels that are not
  virtual.
  Bitcoin-Compatible Virtual Channels~\cite{9519487} also enables
  virtual channels on top of two preexisting simple channels
  and offers two protocols, the first of which guarantees that the channel will
  stay off-chain for an agreed period, while the second allows the single intermediary
  to turn the virtual into a simple channel.
  This strategy has the shortcoming that even if it is made
  recursive (a direction left open in~\cite{9519487}) after $k$
  applications of the constructor the virtual channel participant will have to
  publish on-chain $k$ transactions in order to close the channel if all
  intermediaries actively monitor the blockchain.

  Donner~\cite{donner}
    (released originally concurrently with the first technical report of our work) 
%  is the first work 
%  to 
also achieves variadic
  virtual channels, but without recursion nor future Bitcoin features. This is
  achieved by having the funder lock as
  collateral twice the amount of the desired channel funds: once on-chain with
  funds that are external to the \emph{base channels} (i.e., the channels that the
  virtual channel is based on) and once off-chain within its base channel. Thus
  the required collateral for the funder is double that of other protocols and
  a party lacking sufficient on-chain coins cannot fund a Donner channel;
  additionally, we conjecture that using external coins precludes variadic
  virtual channels that are not encumbered with limited lifetime. This
  design choice further means that Donner is not symmetric. Donner also uses
  placeholder outputs which, due to the minimum coins they need to
  exceed Bitcoin's \emph{dust limit}, may skew the incentives of rational players
  and adds to the
  opportunity cost of channel maintenance. Further, its design complicates
  future iterations that lift its current restriction that only one of the two
  channel parties can fund the virtual channel. The aforementioned incentives
  together with its lack of recursiveness mean that if a party with coins in a
  Donner channel decides to use them with another party, it first has to close
  its channel either off-chain, which needs cooperation of all base parties, or
  else on-chain, with all the delays and fees this entails.
  On the positive side, Donner is
  more efficient than Elmo in terms of storage, computation and communication
  complexity, and boasts a simpler design. 
  %, but has less room for optimisations and is not recursive. 
  Their work also introduces the \emph{Domino attack},
  which we adress in Section~\ref{sec:future-work}.

  Furthermore, as described in~\cite{donner}, Donner
  is insecure since any state update to a base
  channel invalidates the corresponding $\mathsf{tx}^r$. There is a
  straightforward fix, which however adds an overhead to each payment over
  a base channel: On every payment, the two base channel parties must update
  their $\mathsf{tx}^r$ to spend the $\alpha$ output of the new state. Potential
  intermediaries must consider this overhead and possibly increase the fees they
  require from the endpoints. This per-payment overhead can be avoided by using
  \texttt{ANYPREVOUT} in the $\alpha$ output.

  Table~\ref{table:comparison-features} contains a comparison of the
  features and limitations of virtual channel protocols, including Elmo.

  \begin{table*}
    \caption{Features \& requirements comparison of virtual channel protocols}
    \label{table:comparison-features}
    \begin{minipage}{\textwidth}
    \begin{center}
    \begin{tabular}{|l|c|c|c|c|c|}
    \hline
              & Unlimited lifetime & Recursive & Variadic & Symmetric & Script requirements \\
    \hline
    LVPC~\cite{10.1007/978-3-030-65411-5_18}
              & ✗                  & \LEFTcircle\footnote{lacks security analysis}
                                               & ✗         & ✓         & Bitcoin \\
    \hline
    BCVC~\cite{9519487}
              & ✓                  & ✗         & ✗         & ✓         & Bitcoin \\
    \hline
    Perun~\cite{perun}
              & ✓                  & ✗         & ✗         & ✓         & Ethereum \\
    \hline
    GSCN~\cite{DBLP:conf/ccs/DziembowskiFH18}
              & ✓                  & ✓         & ✗         & ✓        & Ethereum \\
    \hline
    Donner~\cite{donner}
              & ✗                  & ✗         & ✓         & ✗         & Bitcoin \\
    \hline
    this work & ✓                  & ✓         & ✓         & ✓         & Bitcoin + \texttt{ANYPREVOUT} \\
    \hline
    \end{tabular}
    \end{center}
    \end{minipage}
  \end{table*}
