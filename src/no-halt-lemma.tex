\begin{lemma}[No halt]
\label{lemma:no-halt}
  In an ideal execution with \fchan and \simulator, the functionality halts with
  negligible probability in the security parameter (i.e.
  l.~\ref{code:functionality:state-machine:halt} of
  Fig.~\ref{code:functionality:state-machine} is executed negligibly often).
\end{lemma}

\begin{proof}
  The only way for \fchan to halt is if either $\texttt{check}_A$ or
  $\texttt{check}_B$ fails. For these checks to take place, \fchan must have
  received (\textsc{close}, \alice) or (\textsc{close}, \bob) by \simulator
  while in the \textsc{open} \textit{State}
  (Fig.~\ref{code:functionality:state-machine},
  l.~\ref{code:functionality:state-machine:close:prev-state}). Additionally,
  \fchan can only reach the \textsc{open} \textit{State} if all honest simulated
  parties transition to the \textsc{open} \textit{State} as well
  (Fig.~\ref{code:simulator}, l.~\ref{code:simulator:open}), which in turn, if
  simulated \alice is honest, happens only if she has received (\textsc{open},
  \dots) by \environment (Fig.~\ref{code:ln:open}). Observe further that
  \simulator notifies \fchan right away when either simulated party becomes
  corrupted or negligent (Fig.~\ref{code:simulator},
  ll.~\ref{code:simulator:corrupted} and~\ref{code:simulator:negligent}
  respectively) and the balance of each party is checked by \fchan only if this
  party is not corrupted nor negligent. These facts in combination mean that for
  each party, whenever the prerequisites for Lemma~\ref{lemma:ideal-balance} are
  true the prerequisites for Lemma~\ref{lemma:real-balance-security} are also
  true and therefore the check for this party will succeed
  (Fig~\ref{code:functionality:state-machine},
  l.~\ref{code:functionality:state-machine:close:coins}). Furthermore, when the
  first prerequisite for Lemma~\ref{lemma:ideal-balance} does not hold (i.e.
  $\texttt{ignore}_P = \mathrm{False}$), the check for the respective party will
  be true, when the second prerequisite does not hold (i.e. $\textit{State} \neq
  \textsc{open}$ when (\textsc{close}, \alice/\bob) is received) the check will
  not take place at all and the other three prerequisites hold always, therefore
  the check cannot fail if the prerequisites for Lemma~\ref{lemma:ideal-balance}
  do not hold. On aggregate, when an ideal execution of \fchan and \simulator
  take place, \fchan will halt with negligible probability in the security
  parameter.
\end{proof}
