Both \pchan and \fchan are parametrized by the stateful processes \textsc{pcn}
(payment channel network) and \textsc{virt} (virtual layer).
\TODO{if the 2 processes share too much state, merge into 1 process}

If:
\begin{itemize}
  \item \pchan (\fchan) is activated by \environment ($\dave \in \{\alice,
  \bob\}$),
  \item \pchan (\fchan) then calls a method of either process (expecting some
  value to be returned by it),
  \item and subsequently the method gives up the execution token to another ITI
  (before it returns),
\end{itemize}
then \pchan (\fchan) repeatedly relays any input by \environment (\dave) to the
method until the latter returns.

The following functions are available for both \pchan and \fchan.
\begin{figure}[H]
  \begin{titlebox}{\normalfont locked coins in nested virtual
  channels}{commonbox}{normal}
    \begin{algorithmic}[1]
      \State $\mathrm{locked}(\alice)$:
      \Indent
        \State $\mathrm{res} \gets 0$
        \For{$((c_1, c_2), \_, \_, (\alice, \_), \_) \in \texttt{virtuals}$}
          \State $\mathrm{res} \gets \mathrm{res} + c_1 + c_2$
        \EndFor
        \State \Return res
      \EndIndent
    \end{algorithmic}
  \end{titlebox}
  \caption{}
  \label{code:locked}
\end{figure}
